\documentclass[mat1, tisk]{fmfdelo}
% \documentclass[fin1, tisk]{fmfdelo}
% Če pobrišete možnost tisk, bodo povezave obarvane,
% na začetku pa ne bo praznih strani po naslovu, …

\usepackage{bussproofs}
\usepackage{amsfonts}
\usepackage{amsmath}
\usepackage{amssymb}
\usepackage{cmll}
\usepackage{graphicx}
\usepackage{wrapfig}
\graphicspath{ {./Slike/} }
%%%%%%%%%%%%%%%%%%%%%%%%%%%%%%%%%%%%%%%%%%%%%%%%%%%%%%%%%%%%%%%%%%%%%%%%%%%%%%%
% METAPODATKI
%%%%%%%%%%%%%%%%%%%%%%%%%%%%%%%%%%%%%%%%%%%%%%%%%%%%%%%%%%%%%%%%%%%%%%%%%%%%%%%

% - vaše ime
\avtor{Jona Koltaj}

% - naslov dela v slovenščini
\naslov{Eliminacija rezov v linearni logiki}

% - naslov dela v angleščini
\title{Cut elimination in linear logic}

% - ime mentorja/mentorice s polnim nazivom:
%   - doc.~dr.~Ime Priimek
%   - izr.~prof.~dr.~Ime Priimek
%   - prof.~dr.~Ime Priimek
%   za druge variante uporabite ustrezne ukaze
\mentor{prof.~dr.~Andrej Bauer}
% \somentor{...}
% \mentorica{...}
% \somentorica{...}
% \mentorja{...}{...}
% \somentorja{...}{...}
% \mentorici{...}{...}
% \somentorici{...}{...}

% - leto diplome
\letnica{2025}

% - povzetek v slovenščini
%   V povzetku na kratko opišite vsebinske rezultate dela. Sem ne sodi razlaga
%   organizacije dela, torej v katerem razdelku je kaj, pač pa le opis vsebine.
\povzetek{Vpeljemo sekventni račun, njegova strukturna pravila ter pravili vpeljave za veznik $\land$, nato pa se omejimo na linearno logiko ter veznik $\land$ razdelimo na dva. Vpeljemo še vse ostale veznike v linearni logiki ter razložimo njihov pomen, nato vpeljemo pravilo reza. Formuliramo izrek o eliminaciji reza ter ga dokažemo z dvojno strukturno indukcijo, zunanjo na strukturi drevesa izpeljave, notranjo na strukturi rezane formule. Definiramo glavni rez in ga eliminiramo za vsak veznik posebej, pri eksponentih pa definiramo in nato eliminiramo še posplošeni rez. Rez (in posplošeni rez) eliminiramo tudi, ko ni glaven, nato pa opravimo še korak baze indukcije. Po končanem dokazu izreka dokažemo še Grishinovo vložitev nelinearnega sekventnega računa v linearni sekventni račun.}

% - povzetek v angleščini
\abstract{We introduce sequent calculus, its structural rules and the introduction rules for the logical connective $\land$. We then limit the sequent calculus to linear logic nad split $\land$ into two connectives. We further introduce all other connectives in linear logic and explain their interpretations, then we introduce the cut rule. We formulate the cut elimination theorem and prove it with double structural induction, the outer induction on the structure of the proof tree and the inner induction on the structure of the cut formula. We define the principal cut and eliminate it for each connective seperately. We define and eliminate the generalised cut rule for the exponentials. We eliminate the cut rule (and the generalised cut rule) when it is not principal as well and then proceed to the induction base. After the proof of the cut eliminaiton theorem we also prove the Grishin embedding of the non-linear sequent calculus into the linear one.}

% - klasifikacijske oznake, ločene z vejicami
%   Oznake, ki opisujejo področje dela, so dostopne na strani https://www.ams.org/msc/
\klasifikacija{03B47,03F05}

% - ključne besede, ki nastopajo v delu, ločene s \sep
\kljucnebesede{sekventni račun\sep linearna logika\sep eliminacija rezov\sep Grishinova vložitev}

% - angleški prevod ključnih besed
\keywords{sequent calculus\sep linear logic\sep cut elimination\sep Grishin embedding} % angleški prevod ključnih besed

% - angleško-slovenski slovar strokovnih izrazov
% \slovar{
% \geslo{sequent}{sekvent}
% \geslo{term}{term}
% \geslo{atomic formula}{osnovna formula}
% \geslo{introduction rule}{pravilo vpeljave}
% \geslo{axiom}{aksiom}
% \geslo{exchange}{menjava}
% \geslo{weakening}{ošibitev}
% \geslo{contraction}{skrčitev}
% \geslo{intuitionistic logic}{intuicionistična logika}
% \geslo{exponentials}{eksponenti}
% \geslo{connective ,,of course''}{veznik ,,seveda''}
% \geslo{connective ,,why not''}{veznik ,,zakaj ne''}
% \geslo{cut rule}{pravilo reza}
% \geslo{cut elimination}{eliminacija rezov}
% \geslo{proof tree}{drevo izpeljave}
% \geslo{leaf}{list}
% \geslo{subtree}{poddrevo}
% \geslo{principal cut}{glavni rez}
% \geslo{generalised cut rule}{posplošeno pravilo reza}
% \geslo{embedding}{vložitev}
% }

% - ime datoteke z viri (vključno s končnico .bib), če uporabljate BibTeX
% \literatura{....bib}

%%%%%%%%%%%%%%%%%%%%%%%%%%%%%%%%%%%%%%%%%%%%%%%%%%%%%%%%%%%%%%%%%%%%%%%%%%%%%%%
% DODATNE DEFINICIJE
%%%%%%%%%%%%%%%%%%%%%%%%%%%%%%%%%%%%%%%%%%%%%%%%%%%%%%%%%%%%%%%%%%%%%%%%%%%%%%%
\newenvironment{bprooftree}{\leavevmode\hbox\bgroup}{\DisplayProof\egroup}

\newcommand{\pravilo}[1]{\RightLabel{\footnotesize{#1}}}
\newcommand{\levopravilo}[1]{\LeftLabel{\footnotesize{#1}}}
\newcommand{\negacija}{\mathord{\sim}}
\newcommand{\enota}{\mathbf{1}}
\newcommand{\nicla}{\mathbf{0}}
\newcommand{\dol}{\begin{center}$\downarrow$\end{center}}
\newcommand{\R}{\mathfrak{R}}
\newcommand{\h}{\mathfrak{h}}
\newcommand{\D}{\mathcal{D}}

\theoremstyle{definition}
\newtheorem*{summary}{Oris dokaza}
%%%%%%%%%%%%%%%%%%%%%%%%%%%%%%%%%%%%%%%%%%%%%%%%%%%%%%%%%%%%%%%%%%%%%%%%%%%%%%%
% ZAČETEK VSEBINE
%%%%%%%%%%%%%%%%%%%%%%%%%%%%%%%%%%%%%%%%%%%%%%%%%%%%%%%%%%%%%%%%%%%%%%%%%%%%%%%

\begin{document}

\section{Uvod}
Ko preučujemo trditve in izreke, nas velikokrat bolj od izrekov samih zanimajo njihovi dokazi, saj ti držijo bistvo izreka samega. Ker bi torej želeli bolje preučiti strukturo teh dokazov, bi pomagalo, če bi lahko kako formalizirali to dokazovanje. Tu nastopijo formalni sistemi dokazovanja, ki -- kot pove že ime -- formalizirajo dokaze. Takih sistemov je več, najbolj poznani izmed njih pa so Hilbertov sistem, naravna dedukcija ter sekventni račun. Prvi Hilbertov sistem dokazovanja je bil predstavljen v letu 1879 s strani Gottloba Frege, nemškega filozofa in matematika. V tem sistemu je vsak korak dokaza ali aksiom, ali pa je dobljen iz aksioma z enim izmed dveh pravil sklepanja. Karakteriziran je tudi s tem, da uporablja le dva veznika, namreč implikacijo in negacijo, kar ga sicer naredi minimalističnega in neodvečnega, a je velikokrat bistvo dokaza težko izluščiti. Sekventni račun in naravna dedukcija pa sta bila vpeljana v istem članku, leta 1934, ki ga je objavil nemški matematik Gerhard Gentzen. Te dva sistema vključujeta več veznikov in tudi več pravil sklepanja. Sekventni račun temelji na levih in desnih pravilih vpeljave veznikov, naravna dedukcija pa ima pravila vpeljave ter pravila eliminacije. Oba sistema sta torej veliko manj minimalistična od Hilbertovega sistema, a sta v nekaterih pogledih bolj berljiva. V tem delu bomo preučevali sekventni račun, ki ga bomo bolj natančno vpeljali v poglavju \ref{seq_calc}.

V članku, kjer sta bila sekventni račun in naravna dedukcija vpeljana, je poleg tega Gentzen dokazal enega izmed pomembnejših izrekov, kar se tiče sistemov dokazovanja, namreč izrek o eliminaciji rezov. Ta sistemu dokazovanja na formalen način zagotavlja konsistentnost. V poglavju \ref{ier} bomo tudi mi ta izrek formulirali ter dokazali, le da bomo to naredili za podzvrst sekventnega računa, imenovano linearna logika. Slednjo je prvič v članku iz leta 1987 opisal Jean-Yves Girard, francoski matematik, ki je ugotovil, da z omejitvijo določenih strukturnih pravil v formalnem sistemu dokazovanja lahko bolj natančno preučujemo \emph{koliko} predpostavk smo porabili v dokazu. Linearno logiko je možno obravnavati tako v naravni dedukciji kot sekventnem računu, a se bomo v tem delu omejili na sekventni račun.

Glavna motivacija za linearno logiko je zavedanje, koliko ,,surovin'' smo porabili in pridelali, torej kolikokrat smo tekom dokaza predpostavko uporabili, katere predpostavke smo zavrgli ter katere sklepe smo dokazali večkrat. Omejitev podvajanja in odvečnih predpostavk pomembno vpliva tudi na veznike, ki jih uporabljamo. Več o tem bomo povedali v poglavju \ref{ll}. Želimo pa tudi vedeti, kako linearna logika modelira nelinarno logiko in kako se ti dve primerjata med seboj, kar pa bomo nazadnje obravnavali v poglavju \ref{cl v cll}.


\section{Sekventni račun} \label{seq_calc}
Sekventni račun je formalni sistem dokazovanja, ki sestoji iz t.~i.\ sekventov in vnaprej določenih pravil, kako jih smemo preoblikovati. Vsak korak dokaza torej uporabi enega izmed teh pravil, dokler začetnega sekventa ali sekventov ne preoblikujemo v tistega, ki smo ga želeli dokazati.

Korake ločimo s horizontalno črto, nad katero so vsi sekventi, ki jih pravilo uporabljeno na tem koraku sprejme, velikokrat pomimenovani \emph{hipoteze}, pod njo pa je novo dobljeni sekvent, navadno imenovan \emph{sklep}. Označimo hipoteze s $\mathcal{H}_0, \mathcal{H}_1, \ldots, \mathcal{H}_n$, sklep pa s $\mathcal{C}$. Korak izpeljave bo torej izgledal takole:

\begin{prooftree}
    \AxiomC{$\mathcal{H}_0$}
    \AxiomC{$\mathcal{H}_1$}
    \AxiomC{$\dots$}
    \AxiomC{$\mathcal{H}_n$}
    \pravilo{Pravilo}
    \QuaternaryInfC{$\mathcal{C}$}
\end{prooftree}

Na desni ponavadi označimo, katero pravilo smo uporabili na tem koraku, zavoljo preglednosti.

\subsection{Sekvent in formula}
Sekvent sestoji iz \emph{logičnih formul}, ki jih je torej potrebno definirati preden lahko definiramo sekvent. Definicija je induktivna, kar pomeni, da se formule gradijo iz podformul, te iz svojih podformul in tako dalje, na dnu pa so t.~i.\ osnovne formule. Te pa so zgrajene iz termov, ki so prav tako induktivno definirani.
\begin{definicija}
    \emph{Term} je izraz, ki je lahko oblike:
    \begin{itemize}
        \item spremenljivka,
        \item konstanta,
        \item $f(t_1,\ldots,t_n)$, kjer je $t_i$ term za vsak $i\in [n]$ in je $f$ nek funkcijski simbol, ki sprejme $n\in \mathbb{N}$ termov.
    \end{itemize}
\end{definicija}
\begin{primer*}
    Na voljo imamo denimo naravna števila, na katerih je definiran funkcijski simbol $+$, ki sprejme dva terma. Možni termi, ki jih lahko tvorimo, so torej lahko npr.\ spremenljivka $x$, konstanta $3$ ali pa izraz $x+3$.
\end{primer*}

\begin{definicija} \label{formula}
	\emph{Formula} je izraz oblike:
	\begin{itemize}
        \item \emph{osnovna formula}; $R(t_1, ..., t_n)$, kjer je $t_i$ term za vsak $i\in [n]$ in je $R$ nek \emph{relacijski simbol}, ki sprejme $n\in\mathbb{N}$ termov,
        \item \emph{sestavljena formula}, ki je -- kot pove ime -- sestavljena iz ene ali več podformul, med seboj povezanih z veznikom ali kvantifikatorjem.
	\end{itemize}
\end{definicija}
\begin{primer*}
    Imamo denimo terma $t_1$ in $t_2$ in relacijski simbol $\geq$, ki sprejme dva terma. Tvorimo torej lahko osnovno formulo $t_1\geq t_2$.
\end{primer*}
\begin{primer*}
    Če imamo formuli $A$ in $B$, so npr.\ $A \land B, \neg A, A \lor B$ ali $\forall x A$, kjer je $x$ neka prosta spremenljivka v $A$, tudi formule.
\end{primer*}
Katere sestavljene formule lahko tvorimo je odvisno od tega, s kakšnimi vezniki želimo delati. Če na primer nimamo veznika $\land$, $A \land B$ ne more biti formula. Specifične veznike, ki jih bomo uporabljali pri linearni logiki, bomo natančneje definirali v poglavju \ref{ll}.

\begin{definicija}
    Naj bodo $A_0,\ldots,A_n$ ter $B_0,\ldots,B_m$ neke logične formule. \emph{Sekvent} je izraz oblike $A_0,\ldots,A_n \Rightarrow B_0,\ldots,B_m$.
\end{definicija}

Formulam na levi strani sekventa navadno pravimo \emph{predpostavke}, formulam na desni pa \emph{sklepi}, obojemu pa lahko rečemo tudi \emph{kontekst}. Predpostavke navadno označujemo z $\Gamma$, sklepe pa z $\Delta$, kjer tako $\Gamma$ kot $\Delta$ predstavljata \emph{multimnožici} logičnih formul.

\begin{definicija}
    \emph{Multimnožica} je funkcija $f:A\to\mathbb{N}$, ki vsakemu elementu iz končne množice $A$ priredi naravno število. Drugače povedano je to množica, kjer se elementi lahko ponavljajo.
\end{definicija}

Multimnožice kontekstov še vedno zapišemo kot sezname, le da vrstni red ni pomemben, število pojavitev pa je. Tako sta na primer sekventa $A,A,B \Rightarrow C$ in $A,B,A\Rightarrow C$ pravzaprav isti sekvent, ki pa je različen od sekventa $A,B \Rightarrow C$.
Pomembno je omeniti še, da simbol $\Rightarrow$ v sekventu ne predstavlja običajne implikacije in ga raje beremo kot ,,dokaže''. Vejice na levi strani sekventa se bere kot ,,in'', vejice na desni strani pa kot ,,ali''. Sekvent $A,B \Rightarrow C,D$ bi se torej razumel kot ,,predpostavki $A$ in $B$ dokažeta sklep $C$ ali sklep $D$''.


\subsection{Pravila pri sekventnem računu}
Pravila pri sekventnem računu delimo na \emph{strukturna pravila}, ki nam povedo kako ravnati s poljubnimi zaporedji formul, \emph{logična pravila} ali \emph{pravila vpeljave}, ki nam povedo kako z različnimi vezniki tvorimo nove formule, in pa pravilo aksioma.

\begin{definicija}
    \emph{Aksiom} je vsak sekvent oblike $A \Rightarrow A$, kar lahko interpretiramo kot ,,formula dokaže sama sebe''. To je seveda vedno res, zato pravilo aksioma, skrajšano $Ax$, pravi, da aksiome lahko vedno tvorimo, t.~j.\ zanje ne potrebujemo predhodnih sekventov. Zapisano v sekventnem računu torej:
    \begin{prooftree}
        \AxiomC{}
        \pravilo{Ax}
        \UnaryInfC{$A \Rightarrow A$}
    \end{prooftree}
\end{definicija}

\subsubsection{Pravila vpeljave}

Pravila vpeljave pri sekventnem računu načeloma sestojijo iz \emph{levega pravila vpeljave} ter \emph{desnega pravila vpeljave}. Prvo nam pove kako veznik uporabiti med predpostavkami, drugo pa kako dani veznik dokazati.

Oglejmo si kot primer pravilo vpeljave za veznik $\land$. Več veznikov bomo vpeljali in si podrobneje pogledali v poglavju ??.

\begin{definicija} \label{inl}
	\emph{Levo pravilo vpeljave veznika $\land$}, krajše $L\land$, pravi, da če znamo nekaj dokazati iz formule $A$, znamo isto dokazati iz $A \land B$ za poljubno formulo $B$. Ker je veznik $\land$ simetričen, je tudi to pravilo simetrično.
	\begin{center}
        \begin{bprooftree}
            \AxiomC{$\Gamma, A \Rightarrow \Delta$}
            \pravilo{L$\land$}
            \UnaryInfC{$\Gamma,A \land B \Rightarrow \Delta$}
        \end{bprooftree}\qquad
        in\qquad
        \begin{bprooftree}
            \AxiomC{$\Gamma, B \Rightarrow \Delta$}
            \pravilo{L$\land$}
            \UnaryInfC{$\Gamma,A \land B \Rightarrow \Delta$}
        \end{bprooftree}
    \end{center}
\end{definicija}

\begin{definicija} \label{inr}
	\emph{Desno pravilo vpeljave veznika $\land$}, krajše $R\land$, pa pravi, da če znamo iz nekih predpostavk dobiti formulo $A$ ter iz istih predpostavk dobiti formulo $B$, znamo iz teh predpostavk dobiti tudi formulo $A \land B$.
	\begin{prooftree}
        \AxiomC{$\Gamma \Rightarrow A,\Delta$}
        \AxiomC{$\Gamma \Rightarrow B,\Delta$}
        \pravilo{R$\land$}
        \BinaryInfC{$\Gamma \Rightarrow A \land B,\Delta$}
    \end{prooftree}
\end{definicija}

\subsubsection{Strukturna pravila}

Običajno je sekventni račun opremljen s tremi strukturnimi pravili.

\begin{definicija}
	\emph{Pravilo menjave}, krajše $Ex$, nam pove, da lahko vrstni red predpostavk in sklepov med seboj poljubno menjamo.
	\begin{center}
        \begin{bprooftree}
            \AxiomC{$\Gamma,A,B,\Gamma' \Rightarrow \Delta$}
            \pravilo{Ex}
            \UnaryInfC{$\Gamma,B,A,\Gamma' \Rightarrow \Delta$}
        \end{bprooftree}\qquad
        in \qquad
        \begin{bprooftree}
            \AxiomC{$\Gamma \Rightarrow \Delta,A,B,\Delta'$}
            \pravilo{Ex}
            \UnaryInfC{$\Gamma \Rightarrow \Delta,B,A,\Delta'$}
        \end{bprooftree}
	\end{center}
\end{definicija}

\begin{opomba}
	Do sedaj smo na $\Gamma$ in $\Delta$ gledali kot zaporedji formul. Če ju namesto tega definiramo kot \emph{multimnožici}, torej množici, kjer je vsakemu elementu prirejeno število pojavitev, lahko pravilo menjave zavržemo, saj sledi že iz same strukture predpostavk in sklepov. Zavoljo enostavnosti bomo torej na $\Gamma$ in $\Delta$ v nadaljevanju gledali kot multimnožici.

	Tu je pomembno, da to ni le množica, saj nas še vedno zanima koliko formul, četudi istih, nastopa v sekventu.
\end{opomba}

\begin{definicija} \label{weakening}
	\emph{Ošibitev}, krajše $W$, nam pove, da lahko tako predpostavke kot sklepe ,,ošibimo'' z dodatno formulo.
    \begin{center}
        \begin{bprooftree}
            \AxiomC{$\Gamma \Rightarrow \Delta$}
            \pravilo{W}
            \UnaryInfC{$\Gamma,A \Rightarrow \Delta$}
        \end{bprooftree} \qquad
        in \qquad
        \begin{bprooftree}
            \AxiomC{$\Gamma \Rightarrow \Delta$}
            \pravilo{W}
            \UnaryInfC{$\Gamma \Rightarrow \Delta, A$}
        \end{bprooftree}
    \end{center}
\end{definicija}

Kar to pomeni na levi je, da če znamo že iz $\Gamma$ dokazati $\Delta$, potem lahko med predpostavke dodamo kakršnokoli odvečno formulo in bomo $\Delta$ še vedno znali dokazati. Odvečne predpostavke nam torej ne škodujejo.

Na desni pa, ker tam vejico beremo kot ,,ali'' velja podobno. Če znamo iz $\Gamma$ dokazati $\Delta$, potem znamo iz $\Gamma$ dokazati tudi $\Delta$ ali $A$.

\begin{definicija} \label{contraction}
    \emph{Skrčitev}, krajše $C$, nam pove, da število ponovitev formule tako med predpostavkami kot sklepi ni pomembno.
    \begin{center}
        \begin{bprooftree}
            \AxiomC{$\Gamma,A,A \Rightarrow \Delta$}
            \pravilo{C}
            \UnaryInfC{$\Gamma,A \Rightarrow \Delta$}
        \end{bprooftree} \qquad
        in \qquad
        \begin{bprooftree}
            \AxiomC{$\Gamma \Rightarrow \Delta,A,A$}
            \pravilo{C}
            \UnaryInfC{$\Gamma \Rightarrow \Delta,A$}
        \end{bprooftree}
    \end{center}
\end{definicija}

Če torej znamo dokazati $\Delta$ iz dveh ponovitev formule $A$, znamo isto dokazati iz le ene kopije. Prav tako, če znamo dvakrat dokazati $A$, znamo to seveda narediti tudi enkrat.

Preden preidemo specifično na linearno logiko je potrebno še omeniti, da pri sekventnem računu dokazujemo ,,od spodaj navzgor''. Začnemo torej s sekventom, ki bi ga želeli dokazati in poiščemo katera pravila, strukturna ali logična, so nam na voljo. Analogija pri dokazovanju v vsakdanji matematiki je, da začnemo s problemom, ki ga želimo dokazati, in ga razčlenimo na manjše podprobleme, dokler ne dobimo nečesa, za kar gotovo vemo da je res. Prav tako poskušamo pri sekventnem računu sekvente postopoma prevesti na aksiom, ki pa bo vedno veljal.

Tudi pravila si zato lahko interpretiramo drugače. Levo pravilo za veznik $\land$ iz definicije \ref{inl} lahko sedaj razumemo kot; če želimo iz $A \land B$ dokazati neke sklepe $\Delta$ je dovolj da $\Delta$ dokažemo iz formule $A$ ali iz formule $B$. Desno pravilo za $\land$ iz definicije \ref{inr} pa razumemo kot; če želimo iz predpostavk $\Gamma$ dokazati $A \land B$, je dovolj da iz $\Gamma$ dokažemo $A$ ter da iz $\Gamma$ dokažemo $B$.

Za primer dokaza v sekventnem računu si oglejmo skoraj trivialen dokaz komutativnosti veznika $\land$.

\begin{prooftree}
    \AxiomC{}
    \pravilo{Ax}
    \UnaryInfC{$B \Rightarrow B$}
    \pravilo{L$\land$}
    \UnaryInfC{$A \land B \Rightarrow B$}

    \AxiomC{}
    \pravilo{Ax}
    \UnaryInfC{$A \Rightarrow A$}
    \pravilo{L$\land$}
    \UnaryInfC{$A \land B \Rightarrow A$}

    \pravilo{R$\land$}
    \BinaryInfC{$A \land B \Rightarrow B \land A$}
\end{prooftree}

V besedah lahko ta dokaz razumemo sledeče. Želimo izpeljati, da $A \land B$ dokaže $B \land A$. Dovolj je, da dokažemo, da $A \land B$ dokaže $A$ ter da dokaže $B$. Na levi je potem dovolj pokazati, da že $B$ dokaže $B$, kar pa je vedno res. Na desni se zgodi podobno.

Od sedaj naprej bomo vsa pravila in dokaze interpretirali od spodaj navzgor.



\section{Linearna logika} \label{ll}
Linearna logika je različica logike sekventnega računa, kjer zavržemo pravili ošibitve in skrčitve iz definicij \ref{weakening} in \ref{contraction}. To pomeni, da moramo vsako predpostavko uporabiti natanko enkrat ter da ne smemo imeti odvečnih predpostavk. Prav tako moramo vsak sklep dokazati natanko enkrat, brez odvečnih sklepov.

Za primer si oglejmo še eno možno definicijo veznika $\land$, poleg \ref{inl} ter \ref{inr} , označimo ga v tem primeru z $\land'$.
\begin{definicija} \label{in'l}
    \emph{Levo pravilo vpeljave veznika $\land'$} pravi, če želimo iz $A \land' B$ dokazati $\Delta$, lahko veznik na levi enostavno prevedemo nazaj v vejico in iz $A$ ter $B$ dokazujemo $\Delta$.
    \begin{prooftree}
        \AxiomC{$\Gamma,A,B \Rightarrow \Delta$}
        \pravilo{L$\land'$}
        \UnaryInfC{$\Gamma,A \land' B \Rightarrow \Delta$}
    \end{prooftree}
\end{definicija}

\begin{definicija} \label{in'r}
	\emph{Desno pravilo vpeljave veznika $\land'$} pa pravi, da če želimo $A \land' B$ dokazati, lahko predpostavke (in preostale sklepe, ki niso povezani z $A$ in $B$) ločimo na dva dela ter z enim delom dokažemo $A$, z drugim pa $B$. Obratno gledano, če znamo iz $\Gamma$ dokazati $A$ ter iz $\Gamma'$ dokazati $B$, lahko predpostavke združimo in dokažemo $A \land' B$.
    \begin{prooftree}
        \AxiomC{$\Gamma \Rightarrow A,\Delta$}
        \AxiomC{$\Gamma' \Rightarrow B,\Delta'$}
        \pravilo{R$\land'$}
        \BinaryInfC{$\Gamma,\Gamma' \Rightarrow A \land' B,\Delta,\Delta'$}
    \end{prooftree}
\end{definicija}

\begin{lema}
    Če dopustimo uporabo ošibitve in skrčitve, sta si levi in desni pravili vpeljave za $\land$ ter $\land'$ ekvivalentni.
\end{lema}
\begin{dokaz}
    Začnimo z dokazom ekvivalence levih pravil za veznika $\land$ ter $\land'$. Dokaz ekvivalence v tem kontekstu pomeni, da sta pravili medsebojno izpeljivi.
    Izpeljava levega pravila za $\land'$ iz definicije \ref{in'l} na podlagi levega pravila za $\land$ iz definicije \ref{inl} poteka s pomočjo skrčitve.
    \begin{prooftree}
        \AxiomC{$\Gamma,A,B \Rightarrow \Delta$}
        \pravilo{L$\land$}
        \UnaryInfC{$\Gamma,A \land B,B \Rightarrow \Delta$}
        \pravilo{L$\land$}
        \UnaryInfC{$\Gamma,A \land B, A \land B \Rightarrow \Delta$}
        \pravilo{C}
        \UnaryInfC{$\Gamma,A \land B \Rightarrow \Delta$}
    \end{prooftree}
    Najprej torej predpostavko $A \land B$ ,,podvojimo'', nato pa dvakrat uporabimo levo pravilo za $\land$, vsakič ne eni izmed podvojenih predpostavk.

    Obratna izpeljava pa poteka s pomočjo ošibitve. Tu najprej uporabimo levo pravilo za $\land'$, nato pa odvečno izmed predpostavk odstranimo s pomočjo ošibitve.
    \begin{prooftree}
        \AxiomC{$\Gamma,A \Rightarrow \Delta$}
        \pravilo{W}
        \UnaryInfC{$\Gamma,A,B \Rightarrow \Delta$}
        \pravilo{L$\land'$}
        \UnaryInfC{$\Gamma,A \land' B \Rightarrow \Delta$}
    \end{prooftree}

    Podobno dokažemo ekvivalenco desnega pravila za $\land$ iz definicije \ref{inr} ter desnega pravila za $\land'$ iz definicije \ref{in'r}.
    \begin{prooftree}
        \AxiomC{$\Gamma \Rightarrow A,\Delta$}
        \levopravilo{W$\times |\Gamma' \cup \Delta'|$}
        \UnaryInfC{$\Gamma,\Gamma' \Rightarrow A,\Delta,\Delta'$}

        \AxiomC{$\Gamma' \Rightarrow B,\Delta'$}
        \pravilo{W$\times |\Gamma \cup \Delta|$}
        \UnaryInfC{$\Gamma,\Gamma' \Rightarrow B,\Delta,\Delta'$}

        \pravilo{R$\land$}
        \BinaryInfC{$\Gamma,\Gamma' \Rightarrow A \land B,\Delta,\Delta'$}
    \end{prooftree}
    Ko zgoraj iz desnega pravila za $\land$ izpeljujemo desno pravilo za $\land'$, najprej uporabimo desno pravilo za $\land$, torej predpostavk (in sklepov) ne razdelimo na dva dela, zato se s pomočjo ošibitve na levi ,,znebimo'' (saj pravila beremo od spodaj navzgor) predpostavk $\Gamma'$ in sklepov $\Delta'$. To naredimo tako, da ošibitev iteriramo tolikokrat, kolikor je velikost multimnožice $\Gamma'\cup\Delta'$. Podobno naredimo na desni strani.

    Ko pa iz desnega pravila za $\land'$ izpeljujemo desno pravilo za $\land$, najprej ,,podvojimo'' vse predpostavke v $\Gamma$ in vse sklepe v $\Delta$, nato pa uporabimo desno pravilo za $\land'$ in podvojene predpostavke spet razpolovimo.
    \begin{prooftree}
        \AxiomC{$\Gamma \Rightarrow A,\Delta$}
        \AxiomC{$\Gamma \Rightarrow B,\Delta$}
        \pravilo{R$\land'$}
        \BinaryInfC{$\Gamma, \Gamma \Rightarrow A \land' B, \Delta, \Delta$}
        \pravilo{C$\times |\Gamma \cup \Delta|$}
        \UnaryInfC{$\Gamma \Rightarrow A \land' B, \Delta$}
    \end{prooftree}
\end{dokaz}
\begin{opomba}
	V zgornjem dokazu smo se malce podrobneje spustili v intuicijo posameznega dela dokaza, saj je to prvi formalen dokaz v sekventnem računu v tem delu. V nadaljnem je intuicija za posamezne vrstice dokaza načeloma prepuščena bralcu.
\end{opomba}

Kot smo v zgornjem dokazu lahko videli, se za dokaz ekvivalence $\land$ ter $\land'$ na bistven način uporabi tako skrčitev kot ošibitev. Slutimo lahko, da brez teh dveh pravil veznika pravzaprav nista ekvivalentna, kar se tudi izkaže za resnično ??cite??. Zato sta v linearni logiki to dva ločena veznika.

\subsection{Propozicijski vezniki}
\begin{definicija} \label{in}
    Veznik $\land$, s pravili iz definicij \ref{inl} in \ref{inr}, se še vedno glasi \emph{in}, zapišemo pa ga s simbolom $\sqcap$. Zapišimo še enkrat njegovo levo in desno pravilo, tokrat s pravilnim pojmovanjem.
    \begin{center}
        \begin{bprooftree}
            \AxiomC{$\Gamma, A \Rightarrow \Delta$}
            \pravilo{L$\sqcap$}
            \UnaryInfC{$\Gamma,A \sqcap B \Rightarrow \Delta$}
        \end{bprooftree}
        \begin{bprooftree}
            \AxiomC{$\Gamma, B \Rightarrow \Delta$}
            \pravilo{L$\sqcap$}
            \UnaryInfC{$\Gamma,A \sqcap B \Rightarrow \Delta$}
        \end{bprooftree}
        \begin{bprooftree}
            \AxiomC{$\Gamma \Rightarrow A,\Delta$}
            \AxiomC{$\Gamma \Rightarrow B,\Delta$}
            \pravilo{R$\sqcap$}
            \BinaryInfC{$\Gamma \Rightarrow A \sqcap B,\Delta$}
        \end{bprooftree}
    \end{center}
\end{definicija}

\begin{definicija} \label{tenzor}
    Veznik $\land'$, s pravili iz definicij \ref{in'l} in \ref{in'r} pa preimenujemo v \emph{tenzor} ter ga zapišemo s simbolom $\star$.
    \begin{center}
        \begin{bprooftree}
            \AxiomC{$\Gamma,A,B \Rightarrow \Delta$}
            \pravilo{L$\star$}
            \UnaryInfC{$\Gamma,A \star B \Rightarrow \Delta$}
        \end{bprooftree}
        \begin{bprooftree}
            \AxiomC{$\Gamma \Rightarrow A,\Delta$}
            \AxiomC{$\Gamma' \Rightarrow B,\Delta'$}
            \pravilo{R$\star$}
            \BinaryInfC{$\Gamma,\Gamma' \Rightarrow A \star B,\Delta,\Delta'$}
        \end{bprooftree}
    \end{center}
\end{definicija}

Zakaj te dva veznika v kontekstu linearne logike nista enaka je razvidno že če primerjamo njuni levi in desni pravili. Kot smo omenili na začetku tega poglavja je pomembno, da vsako predpostavko uporabimo natanko enkrat. Veznik $\sqcap$ med predpostavkami na nek način vsebuje le eno izmed predpostavk, ki ju združuje, medtem ko veznik $\star$ vsebuje obe. Ko torej uporabimo $A \sqcap B$¸ da dokažemo neki $\Delta$, uporabimo le $A$ ali $B$, medtem ko pri $A \star B$ uporabimo tako $A$ kot $B$. Če pa želimo dokazati, da velja $A \sqcap B$, pa je dovolj, da iz istih predpostavk dokažemo $A$ ter $B$, prav tako ostale sklepe na desni strani sekventa pustimo pri miru. To spet implicira, da vsebuje $A\sqcap B$ enako število informacij kot le $A$ ali $B$. Če pa dokazujemo $A\star B$, pa moramo posebej dokazati $A$, nato pa iz ločenega sklopa predpostavk dokazati $B$. Ostale sklepe poleg $A\star B$ je tudi potrebno posebej dokazati. Vse to spet implicira, da vsebuje tenzor informacij tako za $A$ kot $B$.

Oglejmo si sedaj še preostale veznike, začenši z veznikom $\lor$. V linearni logiki se ta spet razdeli na dvoje, intuicija za to pa je simetična intuiciji za veznik $\land$, zato jo prepustimo bralcu.

\begin{definicija} \label{ali}
	Veznik \emph{ali}, označen z $\sqcup$, ima sledeče levo in desno pravilo vpeljave.
	\begin{center}
        \begin{bprooftree}
            \AxiomC{$\Gamma,A \Rightarrow \Delta$}
            \AxiomC{$\Gamma,B \Rightarrow \Delta$}
            \pravilo{L$\sqcup$}
            \BinaryInfC{$\Gamma,A \sqcup B \Rightarrow \Delta$}
        \end{bprooftree}
        \begin{bprooftree}
            \AxiomC{$\Gamma \Rightarrow A,\Delta$}
            \pravilo{R$\sqcup$}
            \UnaryInfC{$\Gamma \Rightarrow A \sqcup B,\Delta$}
        \end{bprooftree}
        \begin{bprooftree}
            \AxiomC{$\Gamma \Rightarrow B, \Delta$}
            \pravilo{R$\sqcup$}
            \UnaryInfC{$\Gamma \Rightarrow A \sqcup B,\Delta$}
        \end{bprooftree}
    \end{center}
    Kot vidimo sta obe pravili popolnoma simetrični praviloma za veznik $\sqcap$.
\end{definicija}

\begin{definicija} \label{plus}
	Veznik \emph{plus}, označen z + pa je analogno simetričen vezniku $\star$.
	\begin{center}
        \begin{bprooftree}
            \AxiomC{$\Gamma,A \Rightarrow \Delta$}
            \AxiomC{$\Gamma',B \Rightarrow \Delta'$}
            \pravilo{L+}
            \BinaryInfC{$\Gamma,\Gamma',A + B \Rightarrow \Delta,\Delta'$}
        \end{bprooftree}
        \begin{bprooftree}
            \AxiomC{$\Gamma \Rightarrow A,B,\Delta$}
            \pravilo{R+}
            \UnaryInfC{$\Gamma \Rightarrow A + B,\Delta$}
        \end{bprooftree}
    \end{center}
\end{definicija}

Vsi nadaljni vezniki imajo v linearni logiki enaka pravila vpeljave kot v navadnem sekventnem računu in se ne delijo na dva dela, še vseeno pa so to \emph{linearni} vezniki, že samo zaradi pogojev pod katerimi so vpeljani. Če na primer $A$ linearno implicira $B$ to pomeni, da natanko en $A$ implicira natanko en $B$, linearna negacija formule $A$ pa negira natanko en $A$.

Za vpeljavo implikacije zopet potrebujemo drugačen simbol kot smo ga vajeni, saj se $\Rightarrow$ že uporablja v strukturi sekventa samega. Običajni sekventni račun v ta namen uporablja $\rightarrow$, linearna implikacija pa, da se loči od nelinearne, spet uporabi svoj simbol.

\begin{definicija}
	\emph{Implikacija}, označena s simbolom $\multimap$, je vpeljana z naslednjimi pravili.
    \begin{center}
        \begin{bprooftree}
            \AxiomC{$\Gamma \Rightarrow A,\Delta$}
            \AxiomC{$\Gamma',B \Rightarrow \Delta'$}
            \pravilo{L$\multimap$}
            \BinaryInfC{$\Gamma,\Gamma',A \multimap B \Rightarrow \Delta,\Delta'$}
        \end{bprooftree}
        \begin{bprooftree}
            \AxiomC{$\Gamma,A \Rightarrow B,\Delta$}
            \pravilo{R$\multimap$}
            \UnaryInfC{$\Gamma \Rightarrow A \multimap B,\Delta$}
        \end{bprooftree}
    \end{center}
    Kot lahko vidimo je desno pravilo vpeljave dokaj jasno za interpretacijo. Če dokazujemo $A \multimap B$, je dovolj da pod predpostavko $A$ dokažemo $B$. Levo pravilo pa je morda lažje brati od zgoraj navzdol. Če torej z $\Gamma$ dokažemo $A$ ter neke druge sklepe $\Delta$, iz $\Gamma'$ in $B$ pa dokažemo $\Delta'$, lahko iz združenih predpostavk $\Gamma,\Gamma'$ ter dejstva, da iz $A$ sledi $B$ dokažemo združene sklepe $\Delta,\Delta'$.
\end{definicija}

\begin{definicija}
    \emph{Negacija}, označena s simbolom $\negacija$, ima naslednji pravili vpeljave.
    \begin{center}
        \begin{bprooftree}
            \AxiomC{$\Gamma \Rightarrow A,\Delta$}
            \pravilo{L$\negacija$}
            \UnaryInfC{$\Gamma,\negacija A \Rightarrow \Delta$}
        \end{bprooftree}
        \begin{bprooftree}
            \AxiomC{$\Gamma,A \Rightarrow \Delta$}
            \pravilo{R$\negacija$}
            \UnaryInfC{$\Gamma \Rightarrow \negacija A,\Delta$}
        \end{bprooftree}
    \end{center}
    ?? Kaj je tuki sploh intuicija lmao??
\end{definicija}


\subsection{Propozicijske konstante}
V običajnem sekventnem računu imamo dvoje konstant; resnico in neresnico, ki pa se v linearni logiki spet razdelita. Resnici se delita na enoto za $\star$ ter enoto za $\sqcap$, neresnici pa na enoto za + ter enoto za $\sqcup$.

\begin{definicija}
    \emph{Enota} $\enota$, ima levo in desno pravilo:
    \begin{center}
        \begin{bprooftree}
            \AxiomC{$\Gamma \Rightarrow \Delta$}
            \pravilo{L$\enota$}
            \UnaryInfC{$\Gamma,\enota \Rightarrow \Delta$}
        \end{bprooftree}
        \begin{bprooftree}
            \AxiomC{}
            \pravilo{R$\enota$}
            \UnaryInfC{$\Rightarrow \enota$}
        \end{bprooftree}
    \end{center}
    Lahko jo brez predpostavk dokažemo, kar nam pove desno pravilo, levo pravilo pa pove, da če pa vemo da enota velja je to trivialna informacija.
\end{definicija}

\begin{definicija}
    \emph{Resnica} $\top$, ima le desno pravilo:
    \begin{prooftree}
        \AxiomC{}
        \pravilo{R$\top$}
        \UnaryInfC{$\Gamma \Rightarrow \top,\Delta$}
    \end{prooftree}
    Ker levega pravila nima, je med predpostavkami ne moremo uporabiti. Kar pa nam desno pravilo pove, je da resnica vedno velja.
\end{definicija}

\begin{lema} \label{enoti}
	Konstanta $\enota$ je enota za za veznik $\star$, konstanta $\top$ pa je enota za veznik $\sqcap$.
\end{lema}
\begin{dokaz}
    Za dokaz leme nam je potrebno izpeljati sekvente $A \star \enota \Rightarrow A$, $A \Rightarrow A \star \enota$, $A \sqcap \top \Rightarrow A$ ter $A \Rightarrow A \sqcap \top$:
    \begin{center}
        \vskip 10pt
        \begin{bprooftree}
            \AxiomC{}
            \pravilo{Ax}
            \UnaryInfC{$A \Rightarrow A$}
            \pravilo{L$\enota$}
            \UnaryInfC{$A,\enota \Rightarrow A$}
            \pravilo{L$\star$}
            \UnaryInfC{$A \star \enota \Rightarrow A$}
        \end{bprooftree}
        \begin{bprooftree}
            \AxiomC{}
            \levopravilo{Ax}
            \UnaryInfC{$A \Rightarrow A$}

            \AxiomC{}
            \pravilo{R$\enota$}
            \UnaryInfC{$\Rightarrow \enota$}

            \pravilo{R$\star$}
            \BinaryInfC{$A \Rightarrow A \star \enota$}
        \end{bprooftree}
    \end{center}
    \vskip 10pt
    \begin{center}
        \begin{bprooftree}
            \AxiomC{}
            \pravilo{Ax}
            \UnaryInfC{$A \Rightarrow A$}
            \pravilo{L$\sqcap$}
            \UnaryInfC{$A \sqcap \top \Rightarrow A$}
        \end{bprooftree}
        \begin{bprooftree}
            \AxiomC{}
            \levopravilo{Ax}
            \UnaryInfC{$A \Rightarrow A$}

            \AxiomC{}
            \pravilo{R$\top$}
            \UnaryInfC{$A \Rightarrow \top$}

            \pravilo{R$\sqcap$}
            \BinaryInfC{$A \Rightarrow A \sqcap \top$}
        \end{bprooftree}
    \end{center}
    \qedhere
\end{dokaz}

\begin{definicija}
	\emph{Ničla} $\nicla$ ima levo in desno pravilo:
	 \begin{center}
        \begin{bprooftree}
            \AxiomC{}
            \pravilo{L$\nicla$}
            \UnaryInfC{$\nicla \Rightarrow$}
        \end{bprooftree}
        \begin{bprooftree}
            \AxiomC{$\Gamma \Rightarrow \Delta$}
            \pravilo{R$\nicla$}
            \UnaryInfC{$\Gamma \Rightarrow \nicla,\Delta$}
        \end{bprooftree}
    \end{center}
\end{definicija}
\begin{definicija}
    \emph{Neresnica} $\bot$ ima le levo pravilo, kar pomeni, da ne more biti uporabljena kot sklep:
    \begin{prooftree}
        \AxiomC{}
        \pravilo{L$\bot$}
        \UnaryInfC{$\Gamma, \bot \Rightarrow \Delta$}
    \end{prooftree}
\end{definicija}

Kot lahko vidimo, so zgornja pravila za ničlo in neresnico simetrična pravilom za enoto in resnico, prav tako pa je simetričen dokaz naslednje leme dokazu leme \ref{enoti}, zato ga bomo opustili.

\begin{lema}
	Konstanta $\nicla$ je enota za za veznik +, konstanta $\bot$ pa je enota za veznik $\sqcup$.
\end{lema}


\subsection{Kvantifikatorja}
\begin{definicija}
	\emph{Univerzalni kvantifikator}, označen kot navadno s simbolom $\forall$, je definiran z naslednjima praviloma vpeljave. Tu $y$ ne sme biti prost v $\Gamma$ in $\Delta$.
	%notacija tuki bo t/x means substitute vsak x s tjem
	%also citiras simpsona as well
	\begin{center}
        \begin{bprooftree}
            \AxiomC{$\Gamma, A[t/x] \Rightarrow \Delta$}
            \pravilo{L$\forall$}
            \UnaryInfC{$\Gamma,\forall x A \Rightarrow \Delta$}
        \end{bprooftree}
        \begin{bprooftree}
            \AxiomC{$\Gamma \Rightarrow A[y/x],\Delta$}
            \pravilo{R$\forall$}
            \UnaryInfC{$\Gamma \Rightarrow \forall x A,\Delta$}
        \end{bprooftree}
	\end{center}
	Notacija $A[y/x]$ pomeni, da vsako instanco spremenljivke $x$ zamenjamo s spremenljivko $y$. Spremenljivka $t$ v definiciji označuje nek specifičen term $t$, ki si ga izberemo. Levo pravilo vpeljave torej pomeni, da če želimo iz dejstva, da za vsak $x$ velja formula $A$ dokazati $\Delta$, je dovolj, da spremenljivko $x$ v $A$ zamenjamo z nekim specifičnim termom in z njim dokažemo $\Delta$. Spremenljivka $y$ v definiciji pa označuje prosto spremenljivko. Desno pravilo vpeljave je ekvivalentno temu, da pri dokazovanju, da za vsak $x$ velja $A$, fiksiramo poljuben $y$ in dokazujemo $A$.
\end{definicija}

\begin{definicija}
    \emph{Eksistenčni kvantifikator} je spet brez sprememb označen s simbolom $\exists$. Spremenljivka $y$ spet ne sme  biti prosta v $\Gamma$ ter $\Delta$.
    \begin{center}
        \begin{bprooftree}
            \AxiomC{$\Gamma,A[y/x] \Rightarrow \Delta$}
            \pravilo{L$\exists$}
            \UnaryInfC{$\Gamma,\exists x A \Rightarrow \Delta$}
        \end{bprooftree}
        \begin{bprooftree}
            \AxiomC{$\Gamma \Rightarrow A[t/x],\Delta$}
            \pravilo{R$\exists$}
            \UnaryInfC{$\Gamma \Rightarrow \exists x A,\Delta$}
        \end{bprooftree}
	\end{center}
	Tokrat levo pravilo vsebuje prosto spremenjlivko $y$, desno pa specifičen term $t$. Če torej želimo uporabiti dejstvo, da obstaja $x$, da velja $A$, fiksiramo poljuben $y$ in dokazujemo $A$, če pa želimo dokazati, da obstaja $x$, da velja $A$, le poščemo nek specifičen term $t$, da $A$ velja.
\end{definicija}


\subsection{Eksponenta}
Včasih si želimo v linearni logiki emulirati tudi običajen sekventni račun, vključno torej z ošibitvijo ter skrčitvijo, a želimo to nelinearnost omejiti na specifične formule. Namen teh ,,nelinearnih'' formul je, da nam dovolijo v linearni logiki dokazati vse, kar je možno dokazati tudi v običajnem sekventnem računu, a da je iz dokaza takoj razvidno, kateri deli sekventa so bili dokazani linearno in kateri ne. Ker želimo označiti formulo kot nelinearno, jo modificiramo v novo formulo z veznikom. Imamo dva takšna veznika, ki ju imenujemo \emph{eksponenta}.

\begin{definicija}
    Veznik \emph{seveda} je označen s simbolom !, poleg levega in desnega pravila vpeljave pa zanj veljata še ošibitev in skrčitev na levi strani sekventa. Veznik \emph{zakaj ne} pa je označen s simbolom ?, zanj pa prav tako veljajo štiri pravila; levo in desno pravilo vpeljave ter skrčitev in ošibitev na desni strani sekventa.
    \begin{center}
        \begin{bprooftree}
            \AxiomC{$\Gamma,A \Rightarrow \Delta$}
            \pravilo{L!}
            \UnaryInfC{$\Gamma,!A \Rightarrow \Delta$}
        \end{bprooftree}
        \begin{bprooftree}
            \AxiomC{$!\Gamma \Rightarrow A,?\Delta$}
            \pravilo{R!}
            \UnaryInfC{$!\Gamma \Rightarrow \ !A,?\Delta$}
        \end{bprooftree}
        \begin{bprooftree}
            \AxiomC{$\Gamma \Rightarrow \Delta$}
            \pravilo{W!}
            \UnaryInfC{$\Gamma,!A \Rightarrow \Delta$}
        \end{bprooftree}
        \begin{bprooftree}
            \AxiomC{$\Gamma,!A,!A \Rightarrow \Delta$}
            \pravilo{C!}
            \UnaryInfC{$\Gamma,!A \Rightarrow \Delta$}
        \end{bprooftree}
    \end{center}
    \begin{center}
        \begin{bprooftree}
            \AxiomC{$!\Gamma,A \Rightarrow ?\Delta$}
            \pravilo{L?}
            \UnaryInfC{$!\Gamma,?A \Rightarrow ?\Delta$}
        \end{bprooftree}
        \begin{bprooftree}
            \AxiomC{$\Gamma \Rightarrow A,\Delta$}
            \pravilo{R?}
            \UnaryInfC{$\Gamma \Rightarrow \ !A,\Delta$}
        \end{bprooftree}
        \begin{bprooftree}
            \AxiomC{$\Gamma \Rightarrow A,\Delta$}
            \pravilo{W?}
            \UnaryInfC{$\Gamma \Rightarrow \ ?A,\Delta$}
        \end{bprooftree}
        \begin{bprooftree}
            \AxiomC{$\Gamma \Rightarrow \ ?A,?A,\Delta$}
            \pravilo{C?}
            \UnaryInfC{$\Gamma \Rightarrow \ ?A,\Delta$}
        \end{bprooftree}
    \end{center}
    Notacija $!\Gamma$ (ali $?\Delta$) označuje, da je vsaka formula v $\Gamma$ (ali $\Delta$) predznačena z veznikom ! (ali ?).
\end{definicija}

Kar želimo z eksponenti označiti je, da imamo ,,poljubno mnogo'' označene formule na voljo. Ker vejice na levi beremo kot ,,in'', vejice na desni pa kot ,,ali'', je treba podati dva različna veznika. Formula $!A$ torej označuje ,,$A$ in $A$ in $\ldots$ $A$'', kolikor kopij pač potrebujemo, formula $?A$ pa označuje ,,$A$ ali $A$ ali $\ldots$ $A$''.

Interpretacija levega pravila vpeljave za ! je dokaj enostavna. Pove le, da se kadarkoli v procesu dokazovanja lahko odločimo formulo označiti kot nelinearno. Desno pravilo pa je malce bolj komplicirano, saj veznika ! ni tako lahko razumeti na desni strani sekventa. Rabimo, da je že celoten sekvent nelinearen, da lahko ! vpeljemo na desni. Interpretacija levega in desnega pravila za ? je simetrična.


\subsection{Notacija in druge formalnosti}
Na koncu tega poglavja je potrebnih še nekaj opomb glede zapisa veznikov ter strukture sekventov.

\subsubsection{Poimenovanje veznikov}

Zapis, uporabljen v tem diplomskem delu, je črpan iz vira \cite{troelstra}, ni pa najbolj standarden. Jean-Yves Girard, ki se je prvi ukvarjal z linearno logiko je veznike in konstante označil drugače \cite{girard}, ta zapis pa se je tudi ohranil. Razlog za spremembo zapisa v mojem delu je bolj slogoven kot zgodovinski. Načini kako je zapis spremenjen ter standardna imena veznikov v angleščini so prikazana v spodnji tabeli:
\begin{center}
\begin{tabular}{||c|c|c||}
    \hline
    Simbol veznika & Simbol v standardnem zapisu & Ime \\
    \hline\hline
    $\sqcap$ & $\&$ & with \\
    \hline
    $\star$ & $\otimes$ & tensor \\
    \hline
    $\sqcup$ & $\oplus$ & plus \\
    \hline
    + & $\parr$ & par \\
    \hline
    $\top$ & $\top$ & top \\
    \hline
    $\enota$ & $\enota$ & one \\
    \hline
    $\bot$ & $\nicla$ & zero \\
    \hline
    $\nicla$ & $\bot$ & bottom \\
    \hline
    \end{tabular}
\end{center}
Kot lahko vidimo so v standardnem zapisu parni drugačni vezniki kot v našem. Razlog za to je, da veznik + distribuira čez veznik $\sqcap$, veznik $\star$ pa distribuira čez $\sqcup$ \cite[trditev 1.13]{girard}. Velja torej:
\begin{align*}
    A + (B \sqcap C) &\equiv (A + B) \sqcap (A + C)\\
    A \star (B \sqcup C) &\equiv (A \star B) \sqcup (A \star C)
\end{align*}
Toda če veznik + negiramo, ne dobimo veznika $\sqcap$, ampak $\star$, če negiramo veznik $\sqcap$ pa dobimo $\sqcup$ in seveda obratno. Dualna para sta torej $(\star,+)$ ter $(\sqcap,\sqcup)$, kar je veliko bolj razvidno pri naših oznakah. Poleg tega sta si že sami pravili za veznika $\star$ ter +, iz definicij \ref{tenzor} in \ref{plus}, simetrični, kot sta si pravili za $\sqcap$ ter $\sqcup$, iz definicij \ref{in} in \ref{ali}. Zdi se mi, da sta to dovolj pomembna razloga za razumevanje veznikov, da sem se odločila za nestandarden zapis.

\subsubsection{Intuicionistična linearna logika} \label{ill}

Članki o linearni logiki pogosto omenajo tudi intuicionistično linearno logiko, zato se mi zdi pomembno slednje vsaj predstaviti tudi v mojem delu.

Intuicionistična logika je logika brez principa izključene tretje možnosti. To je aksiom, ki pravi, da za poljubno trditev $P$ velja $P\lor\neg P$. V sekventnem računu kot smo ga predstavili do sedaj, se da to pravilo izpeljati iz pravila aksioma. V linearni logiki ta princip velja za veznik +:
\begin{prooftree}
	\AxiomC{}
	\pravilo{Ax}
	\UnaryInfC{$A \Rightarrow A$}
	\pravilo{R$\negacija$}
	\UnaryInfC{$\Rightarrow A,\negacija A$}
	\pravilo{R+}
	\UnaryInfC{$\Rightarrow A + (\negacija A)$}
\end{prooftree}
V običajnem sekventnem računu je izpeljava sekventa $\Rightarrow A\lor\neg A$ zelo podobna. Če torej želimo delati z intuicionistično linearno logiko, moramo strukturo sekventnega računa nekoliko spremeniti. Dovolj je, da na desni strani sekventov ne dopustimo več kot ene formule \cite[stran 20]{troelstra}. Kot lahko hitro vidimo, zgornja izpeljava sedaj ne deluje, saj sekvent $\Rightarrow \negacija A,A$ ni več veljaven.

Vendar pa intuicionistična logika poruši simetrijo, ki jo drugače nosi klasični sekventni račun. Zaradi omejitve števila sklepov desnega pravila za veznik + na primer ne moremo imeti. Ker sama izključena tretja možnost in njen (ne)obstoj v mojem diplomskem delu ne igrata nobene vloge, sem se zavoljo ohranjanja simetrije odločila uporabljati klasično -- torej neintuicionistično -- logiko.



\section{Izrek o eliminaciji rezov} \label{ier}
Poslednje pravilo, ki ga je potrebno vpeljati, je \emph{pravilo reza}. To pravilo obstaja tudi v običajnem sekventnem računu in formalizira koncept dokazovanja s pomočjo leme.

\begin{definicija}
	\emph{Pravilo reza}, krajše $Rez$:
	\begin{prooftree}
        \AxiomC{$\Gamma \Rightarrow A,\Delta$}
        \AxiomC{$\Gamma',A \Rightarrow \Delta'$}
        \pravilo{Rez}
        \BinaryInfC{$\Gamma,\Gamma' \Rightarrow \Delta,\Delta'$}
	\end{prooftree}
	To pravilo pravi, da če znamo pod določenimi predpostavkami dokazati formulo $A$, potem pa iz te formule dokažemo nekaj drugega, lahko $A$ enostavno režemo iz procesa.
\end{definicija}

\begin{opomba}
	Vsa dosedanja pravila v linearni logiki so bila na nek način deterministična. Če smo imeli v sekventu določen veznik, smo lahko uporabili pravilo vpeljave tega veznika, če pa v sekvetnu ni nastopal, tega nismo mogli narediti. Pravilo reza pa je v tem smislu drugačno, saj lahko drevo izpeljave poljubno razvejamo z novimi ,,vrinjenimi'' sekventi.
\end{opomba}
Preden smo v naš sekventni račun vpeljali pravilo reza, je bilo torej dokaj enostavno videti, če sekvent ne velja. Oglejmo si na primer iz podpoglavja \ref{ill}, kjer trdimo, da pricip izključene tretje možnosti ne velja za veznik $\sqcup$, tudi pri klasični linearni logiki.
\begin{center}
    \begin{bprooftree}
        \AxiomC{$\Rightarrow A$}
        \pravilo{R$\sqcup$}
        \UnaryInfC{$\Rightarrow A \sqcup (\negacija A)$}
    \end{bprooftree}
    \begin{bprooftree}
        \AxiomC{$\Rightarrow \negacija A$}
        \pravilo{R$\sqcup$}
        \UnaryInfC{$\Rightarrow A \sqcup (\negacija A)$}
    \end{bprooftree}
\end{center}
Ko dokazujemo sekvent $\Rightarrow A \sqcup (\negacija A)$ brez uporabe pravila reza, imamo na voljo le desno pravilo vpeljave veznika $\sqcup$ in nič drugega. Edina dva možna koraka sta torej prikazana zgoraj, sekvent $\Rightarrow A$ ali $\Rightarrow \negacija A$ pa ne bo veljal za poljubno formulo $A$, torej lahko trdimo, da sekvent, ki ga želimo dokazati, ne velja.

Če pa dopustimo uporabo pravila reza, lahko drevo neskončno razvejamo z vrivanjem poljubnih formul, na primer:
\begin{prooftree}
	\AxiomC{.}
	\noLine
	\UnaryInfC{.}
	\noLine
	\UnaryInfC{.}
	\noLine
	\UnaryInfC{$\Rightarrow B$}

	\AxiomC{.}
	\noLine
	\UnaryInfC{.}
	\noLine
	\UnaryInfC{.}
	\noLine
	\UnaryInfC{$B \Rightarrow A \sqcup (\negacija A)$}

	\pravilo{Rez}
	\BinaryInfC{$\Rightarrow A \sqcup (\negacija A)$}
\end{prooftree}
Drevo izpeljave se lahko razvejuje v neskončnost, zato ne moremo nikoli z gotovostjo trditi, da sekventa ne moremo izpeljati. A če pravilo reza res interpretiramo kot dokaz z uporabo leme, bi bilo smiselno, da če sekventa ne moremo dokazati brez uporabe rezov, ga tudi z rezi ne moremo dokazati. Tu nastopi naslednji izrek.

\begin{izrek}[Izrek o eliminaciji rezov] \label{izrek}
    Vsak sekvent, dokazan z uporabo reza, lahko dokažemo tudi brez uporabe reza.
\end{izrek}

\begin{posledica} %lah pises se o gentzenu etc
    Problem, opisan zgoraj, se torej ne pojavi. Ker sekventa brez reza ne moremo dokazati, ga tudi z rezom ne moremo, ne glede na to, kako razvejamo drevo izpeljave. Izrek nam torej zagotavlja konsistentnost sistema dokazovanja.
\end{posledica}

\subsection{Potrebne definicije} \label{defs}
Dokaz zgornjega izreka poteka s pomočjo indukcije na drevesih izpeljave nad rezom ter na kompleksnosti formul, ki jih režemo. Zavoljo tega moramo tako na drevesih kot formulah vpeljati nekakšno mero, glede na katero bomo potem izvalali proces indukcije. Formula je induktivno definirana v definiciji \ref{formula}, zato je tudi kompleksnost formule definirana induktivno.

\begin{definicija}
    Naj bosta $A$ in $B$ poljubni formuli, $P$ pa naj bo neka osnovna formula. \emph{Rang formule}, označen s simbolom $\R$, je definiran sledeče:
    \begin{itemize}
        \item $\R(P) = \R(\top) = \R(\bot) = \R(\enota) = \R(\nicla) = 1$
        \item $\R(\negacija A) = \R(\forall x A) = \R(\exists x A) = \R(A) + 1$
        \item $\R(A*B) = \max\{\R(A),\R(B)\} + 1$ ; kjer je $*$ poljuben veznik, ki sprejme dve formuli.
    \end{itemize}
\end{definicija}

Podobno je definirana višina drevesa. Imejmo naslednji dve drevesi, levo označeno z $\D$, desno pa z $\D'$:
\begin{center}
    \begin{bprooftree}
        \AxiomC{$\D_0$}
        \AxiomC{$\D_1$}
        \AxiomC{$\ldots$}
        \AxiomC{$\D_n$}
        \pravilo{Pravilo}
        \QuaternaryInfC{$\mathcal{S}$}
    \end{bprooftree}
    \begin{bprooftree}
        \AxiomC{}
        \pravilo{Pravilo'}
        \UnaryInfC{$\mathcal{S}$'}
    \end{bprooftree}
\end{center}
Tu je $n\in\mathbb{N}$, Pravilo je poljubno pravilo, ki sprejme $n$ sekventov, Pravilo' pa je poljubno pravilo, ki jih ne sprejme nič. Slednje je na primer pravilo aksioma ali nekatera izmed pravil za konstante:

\begin{definicija}
    \emph{Višina drevesa izpeljave} je označena s simbolom $\h$ in je enaka:
    \begin{itemize}
        \item$\h(\D) = \max\{\h(\D_0),\h(\D_1),\ldots,\h(\D_n)\} + 1$
        \item$\h(\D') = 1$
    \end{itemize}
\end{definicija}

Sedaj ko imamo mero tako na formulah kot na drevesih izpeljave, lahko defniramo skupno mero vsakega reza. Oglejmo si poljubno pravilo reza:
\begin{prooftree} \label{rez general}
    \AxiomC{$\D_0$}
    \noLine
    \UnaryInfC{$\Gamma,A \Rightarrow \Delta$}

    \AxiomC{$\D_1$}
    \noLine
    \UnaryInfC{$\Gamma' \Rightarrow A,\Delta'$}

    \pravilo{Rez}
    \BinaryInfC{$\Gamma,\Gamma' \Rightarrow \Delta,\Delta'$}
\end{prooftree}
Tu $\D_0$ in $\D_1$ označujeta drevesi izpeljav, ki sta vodili do sekventov pod njima.

\begin{definicija} \label{stopnja}
    \emph{Stopnja reza} je par števil $(\R,\h)$, kjer je $\R$ rang rezane formule, $\h$ pa višina samega drevesa. Z zgornjimi oznakami je stopnja reza torej enaka:
    $$
    (\R,\h) = (\R(A), \max\{\h(\D_0),\h(\D_1)\} + 2)
    $$
    Stopnja reza je urejena \emph{leksikografsko}, kar pomeni, da relacijo < na paru $(\R,\h)$ definiramo sledeče:
    $$
    (\R,\h) < (\R',\h') \Leftrightarrow (\R < \R') \text{ ali } ((\R = \R') \text{ in } (\h < \h'))
    $$
\end{definicija}
\begin{definicija} \label{najnizji}
    Rezu bomo rekli \emph{najnižji rez}, če se nad njim ne pojavi nobeno drugo pravilo reza. To seveda ne pomeni nujno, da ima najnižjo stopnjo ali da je v drevesu izpeljave le en takšen rez.
\end{definicija}

Preden začnemo z dokazom je nazadnje potrebno definirati še eno vrsto reza, in sicer glavni rez.

\begin{definicija} \label{gl rez}
    Če je rezana formula vpeljana v obeh poddrevesih nad pravilom reza, to imenujemo \emph{glavni rez}.
\end{definicija}

\begin{primer*} \label{gl rez in}
    Glavni rez za formulo $A \sqcap B$.
    \begin{prooftree}
        \AxiomC{$\Gamma,A \Rightarrow \Delta$}
        \levopravilo{L$\sqcap$}
        \UnaryInfC{$\Gamma,A \sqcap B \Rightarrow \Delta$}

        \AxiomC{$\Gamma' \Rightarrow A,\Delta'$}
        \AxiomC{$\Gamma' \Rightarrow B,\Delta'$}
        \pravilo{R$\sqcap$}
        \BinaryInfC{$\Gamma' \Rightarrow A \sqcap B,\Delta'$}

        \pravilo{Rez}
        \BinaryInfC{$\Gamma,\Gamma' \Rightarrow \Delta,\Delta'$}
    \end{prooftree}
\end{primer*}


\subsection{Dokaz izreka o eliminaciji rezov} \label{dokaz}

Razlog, da izrek \ref{izrek} imenujemo izrek o \emph{eliminaciji rezov}, leži v tem, da dokaz poteka s postopno eliminacijo rezov iz poljubnega drevesa izpeljave, dokler na koncu ne dobimo drevesa izpeljave brez kakeršnegakoli pravila reza. Kot smo omenili na začetku podpoglavja \ref{defs}, dokaz poteka z dvojno indukcijo. Zunanja indukcija, je strukturna indukcija na drevesoma izpeljave nad rezom, notranja pa je strukturna indukcija na rezani formuli. Označimo z $\mathcal{D}_0$ ter $\mathcal{D}_1$ drevesi izpeljave nad rezom, torej:
\begin{prooftree}
    \AxiomC{$\mathcal{D}_0$}
    \AxiomC{$\mathcal{D}_1$}
    \pravilo{Rez}
    \BinaryInfC{$\mathcal{S}$}
\end{prooftree}

Začnimo z indukcijskim korakom. Predpostavljamo torej, da iz poddreves $\mathcal{D}_0$ in $\mathcal{D}_1$ znamo eliminirati reze. Da se izognemo problemu, kjer je bilo na primer zadnje pravilo v $\mathcal{D}_0$ ali $\mathcal{D}_1$ rez, predpostavimo, da smo reze tudi že eliminirali iz poddreves. Sedaj dokazujemo, da lahko zgornje drevo izpeljave preobrazimo v drevo izpeljave, kjer se rez pojavi višje v drevesu, torej da je vsaj eno izmed dreves nad novim rezom poddrevo $\mathcal{D}_0$ ali poddrevo $\mathcal{D}_1$. To ustreza indukcijski predpostavki, zato je korak indukcije opravljen.

Znotraj tega koraka, pa seveda delamo tudi indukcijo na strukturi formule. Če torej zgornjega drevesa izpeljave ne uspemo prevesti na drevo, kjer se pravilo reza sedaj pojavi višje, ga lahko prevedemo na rez, ki reže podformulo prvotne formule. Baza notranje indukcije je, da pravilo reza, ki reže osnovno formulo, res prestavimo višje.

\begin{opomba}
    Zgoraj opisano indukcijo si lahko predstavljamo kot algoritem, ki sproti eliminira reze iz drevesa izpeljave. Začnemo z poddrevesom, ki kot zadnje pravilo uporabi rez, v nobenem izmed poddreves pa reza ne uporablja več, torej začnemo z ,,najvišjim'' rezom. Če je takih poddreves več, si arbitrarno izberemo enega izmed njih. V tem poddrevesu postopoma potiskamo rez višje ali pa vsaj znižujemo njegovo kompleksnost, dokler reza ne eliminiramo iz poddrevesa. To ponovimo za vsak rez v drevesu izpeljave. Ker je slednje končno, se proces ustavi in drevo izpeljave ne vključuje več rezov.
\end{opomba}

Vrnimo se h koraku indukcije. Pri obravnavi eliminacije reza je potrebno ločiti, ali je rez, ki ga eliminiramo \emph{glaven}, kot ga opiše definicija \ref{gl rez}, ali pa ni.

\subsubsection{Eliminacija glavnega reza} \label{gl rez vezniki}
Od tu naprej bomo zavoljo preglednosti nad sekventi označevali še drevesa izpeljave, ki so do sekventov vodila.

Oblika glavnega reza je odvisna od vsake rezane formule posebej, zato je njegovo eliminacijo potrebno ločiti glede na veznik, ki rezano formulo sestavlja. Začnimo kar z glavnim rezom veznika $\sqcap$, kot v primeru \ref{gl rez in}:
\begin{prooftree}
    \derivation{0}{$\Gamma,A \Rightarrow \Delta$}
    \levopravilo{L$\sqcap$}
    \UnaryInfC{$\Gamma,A \sqcap B \Rightarrow \Delta$}

    \derivation{1}{$\Gamma' \Rightarrow A,\Delta'$}
    \derivation{2}{$\Gamma' \Rightarrow B,\Delta'$}
    \pravilo{R$\sqcap$}
    \BinaryInfC{$\Gamma' \Rightarrow A \sqcap B,\Delta'$}

    \pravilo{Rez}
    \BinaryInfC{$\Gamma,\Gamma' \Rightarrow \Delta,\Delta'$}
\end{prooftree}
\dol
\begin{prooftree}
    \derivation{0}{$\Gamma,A \Rightarrow \Delta$}
    \derivation{1}{$\Gamma' \Rightarrow A,\Delta'$}
    \pravilo{Rez}
    \BinaryInfC{$\Gamma,\Gamma' \Rightarrow \Delta,\Delta'$}
\end{prooftree}
Sklep pred puščico in po njej je enak, saj iz istih poddreves dokažemo isti sekvent. Tako smo rez stopnje $(\R(A) + \R(B) + 1,\h)$ zamenjali z rezom stopnje $(\R(A),\h')$, ki je očitno manjša. Podobno lahko naredimo za glavni rez $A \star B$:
\begin{prooftree}
    \derivation{0}{$\Gamma,A,B \Rightarrow \Delta$}
    \levopravilo{L$\star$}
    \UnaryInfC{$\Gamma,A \star B \Rightarrow \Delta$}

    \derivation{1}{$\Gamma' \Rightarrow A,\Delta'$}
    \derivation{2}{$\Gamma'' \Rightarrow B,\Delta''$}
    \pravilo{R$\star$}
    \BinaryInfC{$\Gamma',\Gamma'' \Rightarrow A \star B,\Delta',\Delta''$}

    \pravilo{Rez}
    \BinaryInfC{$\Gamma,\Gamma',\Gamma'' \Rightarrow \Delta,\Delta',\Delta''$}
\end{prooftree}
\dol
\begin{prooftree}
    \derivation{0}{$\Gamma,A,B \Rightarrow \Delta$}
    \derivation{1}{$\Gamma' \Rightarrow A,\Delta'$}
    \levopravilo{Rez}
    \BinaryInfC{$\Gamma,\Gamma',B \Rightarrow \Delta,\Delta'$}

    \derivation{2}{$\Gamma'' \Rightarrow B,\Delta''$}
    \pravilo{Rez}
    \BinaryInfC{$\Gamma,\Gamma',\Gamma'' \Rightarrow \Delta,\Delta',\Delta''$}
\end{prooftree}
Tokrat smo prvotno drevo izpeljave prevedli na drevo z dvema rezoma, a imata oba nižjo stopnjo, saj je rang rezane formule očitno nižji. Zelo podobno kot zgornje dva primera izvedemo korak indukcije za $A \sqcup B$ ter $A+B$:
\begin{prooftree}
    \derivation{0}{$\Gamma,A \Rightarrow \Delta$}
    \derivation{1}{$\Gamma,B \Rightarrow \Delta$}
    \levopravilo{L$\sqcup$}
    \BinaryInfC{$\Gamma,A \sqcup B \Rightarrow \Delta$}

    \derivation{2}{$\Gamma' \Rightarrow A,\Delta'$}
    \pravilo{R$\sqcup$}
    \UnaryInfC{$\Gamma' \Rightarrow A \sqcup B,\Delta'$}

    \pravilo{Rez}
    \BinaryInfC{$\Gamma,\Gamma' \Rightarrow \Delta,\Delta'$}
\end{prooftree}
\dol
\begin{prooftree}
    \derivation{0}{$\Gamma,A \Rightarrow \Delta$}
    \derivation{2}{$\Gamma' \Rightarrow A,\Delta'$}
    \pravilo{Rez}
    \BinaryInfC{$\Gamma,\Gamma' \Rightarrow \Delta,\Delta'$}
\end{prooftree}
Kot vidimo je zgornji korak indukcije simetričen koraku indukcije za $A \sqcap B$, saj sta tudi veznika sama simetrična. Enako je korak indukcije za $A+B$ simetričen koraku indukcije za $A \star B$:
\begin{prooftree}
    \derivation{0}{$\Gamma,A \Rightarrow \Delta$}
    \derivation{1}{$\Gamma',B \Rightarrow \Delta'$}
    \levopravilo{L+}
    \BinaryInfC{$\Gamma,\Gamma',A + B \Rightarrow \Delta,\Delta'$}

    \derivation{2}{$\Gamma'' \Rightarrow A,B,\Delta''$}
    \pravilo{R+}
    \UnaryInfC{$\Gamma'' \Rightarrow A + B,\Delta''$}

    \pravilo{Rez}
    \BinaryInfC{$\Gamma,\Gamma',\Gamma'' \Rightarrow \Delta,\Delta',\Delta''$}
\end{prooftree}
\dol
\begin{prooftree}
    \derivation{0}{$\Gamma,A \Rightarrow \Delta$}
    \derivation{2}{$\Gamma'' \Rightarrow A,B,\Delta''$}
    \levopravilo{Rez}
    \BinaryInfC{$\Gamma,\Gamma'' \Rightarrow B,\Delta,\Delta''$}

    \derivation{1}{$\Gamma',B \Rightarrow \Delta'$}
    \pravilo{Rez}
    \BinaryInfC{$\Gamma,\Gamma',\Gamma'' \Rightarrow \Delta,\Delta',\Delta''$}
\end{prooftree}
Naslednji primer, ki ga obravnavamo, je veznik $\multimap$. Zopet se rez prevede na dva reza nižje stopnje, na enak način kot pri veznikih $A\star B$ in $A+B$:
\begin{prooftree}
	\derivation{0}{$\Gamma,A \Rightarrow \Delta$}
    \derivation{1}{$\Gamma',B \Rightarrow \Delta'$}
    \levopravilo{L$\multimap$}
    \BinaryInfC{$\Gamma,\Gamma',A \multimap B \Rightarrow \Delta,\Delta'$}

    \derivation{2}{$\Gamma'',A \Rightarrow B,\Delta''$}
    \pravilo{R$\multimap$}
    \UnaryInfC{$\Gamma'' \Rightarrow A \multimap B,\Delta''$}

    \pravilo{Rez}
    \BinaryInfC{$\Gamma,\Gamma',\Gamma'' \Rightarrow \Delta,\Delta',\Delta''$}
\end{prooftree}
\dol
\begin{prooftree}
	\derivation{0}{$\Gamma,A \Rightarrow \Delta$}
	\derivation{2}{$\Gamma'',A \Rightarrow B,\Delta''$}
    \levopravilo{Rez}
    \BinaryInfC{$\Gamma,\Gamma'' \Rightarrow B,\Delta,\Delta''$}

    \derivation{1}{$\Gamma',B \Rightarrow \Delta'$}
    \pravilo{Rez}
    \BinaryInfC{$\Gamma,\Gamma',\Gamma'' \Rightarrow \Delta,\Delta',\Delta''$}
\end{prooftree}
Poslednji izmed izjavnih veznikov, ki nam ga je potrebno obravnavati je negacija:
\begin{prooftree}
    \derivation{0}{$\Gamma \Rightarrow A,\Delta$}
	\levopravilo{L$\negacija$}
	\UnaryInfC{$\Gamma,\negacija A \Rightarrow \Delta$}

	\derivation{1}{$\Gamma',A \Rightarrow \Delta'$}
	\pravilo{R$\negacija$}
	\UnaryInfC{$\Gamma' \Rightarrow \negacija A,\Delta'$}

	\pravilo{Rez}
	\BinaryInfC{$\Gamma,\Gamma' \Rightarrow \Delta,\Delta'$}
\end{prooftree}
\dol
\begin{prooftree}
    \derivation{0}{$\Gamma \Rightarrow A,\Delta$}
	\derivation{1}{$\Gamma',A \Rightarrow \Delta'$}
	\pravilo{Rez}
	\BinaryInfC{$\Gamma,\Gamma' \Rightarrow \Delta,\Delta'$}
\end{prooftree}
Pri eliminaciji glavnega reza, kjer režemo izjavno konstanto, imamo moč obravnavati le konstanti $\enota$ in $\nicla$, saj za $\top$ levo pravilo ne obstaja, za $\bot$ pa ni desnega. Glavni rez, kjer režemo $\top$ ali $\bot$ se torej ne more zgoditi. Za konstanti $\enota$ in $\nicla$ eliminacija glavnega reza izgleda sledeče:
\begin{center}
    \begin{bprooftree}
        \derivation{}{$\Gamma \Rightarrow \Delta$}
        \levopravilo{L$\enota$}
        \UnaryInfC{$\Gamma,\enota \Rightarrow \Delta$}

        \AxiomC{}
        \pravilo{R$\enota$}
        \UnaryInfC{$ \Rightarrow \enota$}

        \pravilo{Rez}
        \BinaryInfC{$\Gamma \Rightarrow \Delta$}
    \end{bprooftree}
    \begin{bprooftree}
        \AxiomC{}
        \levopravilo{L$\nicla$}
        \UnaryInfC{$\nicla \Rightarrow$}

        \derivation{}{$\Gamma \Rightarrow \Delta$}
        \pravilo{R$\nicla$}
        \UnaryInfC{$\Gamma \Rightarrow \nicla,\Delta$}

        \pravilo{Rez}
        \BinaryInfC{$\Gamma \Rightarrow \Delta$}
    \end{bprooftree}
\end{center}
\begin{align*}
    &\downarrow & &\downarrow
\end{align*}
\begin{center}
    \begin{bprooftree}
        \derivation{}{$\Gamma \Rightarrow \Delta$}
    \end{bprooftree} \qquad \qquad \qquad \quad
    \begin{bprooftree}
        \derivation{}{$\Gamma \Rightarrow \Delta$}
    \end{bprooftree}
\end{center}
Kot lahko vidimo sta primera dokaj trivialna, saj je že sam rez take vrste trivialen. Ker smo rez popolnoma eliminirali je indukcijski predpostavki zadoščeno na prazno. Sedaj si oglejmo eliminacijo glavnega reza obeh kvantifikatorjev. Zopet s $t$ označimo specifičen term¸ z $y$ pa neko (svežo) prosto spremenljivko:
\begin{prooftree}
    \derivation{0}{$\Gamma,A[t/x] \Rightarrow \Delta$}
    \levopravilo{L$\forall$}
    \UnaryInfC{$\Gamma,\forall x A \Rightarrow \Delta$}

    \derivation{1}{$\Gamma' \Rightarrow A[y/x],\Delta'$}
    \pravilo{R$\forall$}
    \UnaryInfC{$\Gamma' \Rightarrow \forall x A,\Delta'$}

    \pravilo{Rez}
    \BinaryInfC{$\Gamma,\Gamma' \Rightarrow \Delta,\Delta'$}
\end{prooftree}
\dol
\begin{prooftree}
    \derivation{0}{$\Gamma,A[t/x] \Rightarrow \Delta$}

    \derivation{1}{$\Gamma' \Rightarrow A[y/x],\Delta'$}
    \pravilo{$y:=t$}
    \UnaryInfC{$\Gamma' \Rightarrow A[t/x],\Delta'$}

    \pravilo{Rez}
    \BinaryInfC{$\Gamma,\Gamma' \Rightarrow \Delta,\Delta'$}
\end{prooftree}
V desnem poddrevesu novonastalega drevesa izpeljave spremenljivko $y$ nadomestimo s specifičnim termom $t$. Ker je bil $y$ \emph{prosta} spremenljivka, to lahko naredimo. Tako dobimo formulo, ki je enaka formuli v levem poddrevesu, in jo zato lahko režemo. V definiciji ranga formule so pomembni le vezniki, ki formulo sestavljajo, ne pa tudi termi, ki se v njej pojavljajo, zato je $\R(A) = \R(A[t/x])$. Dalje je:
$$
\R(\forall x A) = \R(A) + 1 = \R(A[t/x]) + 1
$$
To pomeni, da je stopnja novega reza nižja od stopnje prvega, torej je indukcijski predpostavki zadoščeno. Postopek pri eliminaciji reza eksistenčnega kvantifikatorja je simetričen:
\begin{prooftree}
    \derivation{0}{$\Gamma,A[y/x] \Rightarrow \Delta$}
    \levopravilo{L$\exists$}
    \UnaryInfC{$\Gamma,\exists x A \Rightarrow \Delta$}

    \derivation{1}{$\Gamma' \Rightarrow A[t/x],\Delta'$}
    \pravilo{R$\exists$}
    \UnaryInfC{$\Gamma' \Rightarrow \exists x A,\Delta'$}

    \pravilo{Rez}
    \BinaryInfC{$\Gamma,\Gamma' \Rightarrow \Delta,\Delta'$}
\end{prooftree}
\dol
\begin{prooftree}
    \derivation{0}{$\Gamma,A[y/x] \Rightarrow \Delta$}
    \levopravilo{$y:=t$}
    \UnaryInfC{$\Gamma,A[t/x] \Rightarrow \Delta$}

    \derivation{1}{$\Gamma' \Rightarrow A[t/x],\Delta'$}
    \pravilo{Rez}
    \BinaryInfC{$\Gamma,\Gamma' \Rightarrow \Delta,\Delta'$}
\end{prooftree}


\subsubsection{Glavni rez eksponentov ter posplošeno pravilo reza}
Eliminacija glavnega reza eksponentov zahteva posebno obravnavo, saj se dokazovanje tu nekoliko zaplete. Veznika ! in ? sta simetrična, zato bomo podrobno obravnavali le veznik !, bralec pa si lahko sam izpelje dokaze še za veznik ?.

Veznik ! ima štiri logična pravila, ki ga definirajo. Tri pravila veznik vpeljejo na levi strani sekventa, desno pravilo pa ga vpelje na desni. Zato moramo glavni rez formule $!A$ ločiti na tri primere, glede na to kako je bil vpeljan na levi strani. Ogljemo si naprej glavni rez, kjer je $!A$ na levi vpeljan s skrčitvijo:
\begin{prooftree}
    \derivation{0}{$\Gamma,!A,!A \Rightarrow \Delta$}
    \levopravilo{C!}
    \UnaryInfC{$\Gamma,!A \Rightarrow \Delta$}

    \derivation{1}{$!\Gamma' \Rightarrow A,?\Delta'$}
    \pravilo{R!}
    \UnaryInfC{$!\Gamma' \Rightarrow \ !A,?\Delta'$}

    \pravilo{Rez}
    \BinaryInfC{$\Gamma,!\Gamma' \Rightarrow \Delta,?\Delta'$}
\end{prooftree}
Mikalo bi nas zgornje drevo izpeljave zamenjati s sledečim:
\begin{prooftree}
    \derivation{0}{$\Gamma,!A,!A \Rightarrow \Delta$}

    \derivation{1}{$!\Gamma' \Rightarrow A,?\Delta'$}
    \pravilo{R!}
    \UnaryInfC{$!\Gamma' \Rightarrow \ !A,?\Delta'$}

    \levopravilo{Rez}
    \BinaryInfC{$\Gamma,!\Gamma',!A \Rightarrow \Delta,?\Delta'$}

    \derivation{1}{$!\Gamma' \Rightarrow A,?\Delta'$}
    \pravilo{R!}
    \UnaryInfC{$!\Gamma' \Rightarrow \ !A,?\Delta'$}

    \pravilo{Rez}
    \BinaryInfC{$\Gamma,!\Gamma',!\Gamma' \Rightarrow \Delta,?\Delta',?\Delta'$}
    \pravilo{C!$\times|\Gamma'|$}
    \UnaryInfC{$\Gamma,!\Gamma' \Rightarrow \Delta,?\Delta',?\Delta'$}
    \pravilo{C?$\times|\Delta'|$}
    \UnaryInfC{$\Gamma,!\Gamma' \Rightarrow \Delta,?\Delta'$}
\end{prooftree}
Če označimo z $(\R,\h)$ stopnjo prvotnega reza, z $(\R',\h')$ stopjo zgornjega izmed novih rezov, z $(\R'',\h'')$ pa stopnjo spodnjega, velja:
\begin{align*}
    (\R,\h) &= (\R(A) + 1,\h(\D_0) + \h(\D_1) + 5)\\
    (\R',\h') &= (\R(A) + 1,\h(\D_0) + \h(\D_1) + 4)\\
    (\R'',\h'') &= (\R(A) + 1,\h(\D_0) + 2*\h(\D_1) + 7)
\end{align*}
Takoj lahko vidimo, da je rang rezane formule v vseh primerih enak, zato bi bilo potrebno zmanjšati višino. Zgornji izmed novih rezov ima sicer nižjo višino kot prvotni rez, pri spodnjem pa višina znatno naraste. To pomeni, da koraku indukcije ne zadostimo. Da bi lahko ta korak indukcije vseeno opravili, potrebujemo pomožno (razširjeno) pravilo reza.

\begin{definicija}
    \emph{Posplošeni pravili reza}, označeni z Rez!$_n$ in Rez?$_{n}$, sta definirani za vsak $n\in\mathbb{N}_{>0}$;
    \begin{prooftree}
        \AxiomC{$\Gamma,(!A)^n \Rightarrow \Delta$}
        \AxiomC{$\Gamma' \Rightarrow \ !A,\Delta'$}
        \pravilo{Rez!$_n$}
        \BinaryInfC{$\Gamma,\Gamma' \Rightarrow \Delta,\Delta'$}
    \end{prooftree}
    \begin{prooftree}
        \AxiomC{$\Gamma,?A \Rightarrow \Delta$}
        \AxiomC{$\Gamma' \Rightarrow (?A)^n,\Delta'$}
        \pravilo{Rez?$_{n}$}
        \BinaryInfC{$\Gamma,\Gamma' \Rightarrow \Delta,\Delta'$}
    \end{prooftree}
\end{definicija}

\begin{opomba}
    Formula $(!A)^n$ v definiciji predstavlja $n$-kratno pojavitev formule $!A$. Pravili Rez!$_{1}$ ter Rez?$_{1}$ sta torej le pravilo Rez, kjer režemo ali formulo $!A$, ali pa formulo $?A$.
\end{opomba}

\begin{lema}
    Pravili Rez!$_n$ ter Rez?$_{n}$ sta dopustni, kar pomeni, da ju lahko izpeljemo iz že definiranih pravil linearne logike.
\end{lema}

\begin{dokaz}
    Lemo dokažemo z indukcijo na številu $n$. Primer pri $n=1$ je seveda le običajno pravilo reza, kot omenjeno že v zgornji opombi. Če predpostavimo, da pravilo Rez!$_n$ že znamo izpeljati, lahko izpeljemo Rez!$_{n+1}$ na naslednji način.
    \begin{prooftree}
        \AxiomC{$\Gamma,(!A)^{n+1} \Rightarrow \Delta$}
        \UnaryInfC{$\Gamma,(!A)^{n-1},!A,!A \Rightarrow \Delta$}
        \levopravilo{C!}
        \UnaryInfC{$\Gamma,(!A)^{n-1},!A \Rightarrow \Delta$}
        \UnaryInfC{$\Gamma,(!A)^n \Rightarrow \Delta$}

        \AxiomC{$\Gamma' \Rightarrow \ !A,\Delta'$}
        \pravilo{Rez!$_n$}
        \BinaryInfC{$\Gamma,\Gamma' \Rightarrow \Delta,\Delta'$}
    \end{prooftree}
    Pri indukcijskem koraku iz $n=1$ na $n=2$ moramo paziti, saj se v drugi vrstici dokaza pojavi izraz $(!A)^0$. To enostavno interpretiramo kot prazno multimnožico formul. Dokaz dopustnosti pravila Rez?$_{n}$ je simetričen.
\end{dokaz}

Če želimo uporabiti zgoraj definirano posplošeno pravilo reza, moramo izrek \ref{izrek}, ki ga dokazujemo, preoblikovati tako, da ga bo vseboval.
\begin{izrek}
    Vsak sekvent, izpeljan z uporabo pravila reza ali posplošenega pravila reza, lahko dokažemo tudi brez uporabe kateregakoli izmed njiju.
\end{izrek}

Zaradi dopustnosti posplošenega reza je ta izrek le posledica izreka \ref{izrek}. A ker novi izrek eliminira tako navadni kot posplošeni rez, je izrek \ref{izrek} prav tako le posledica tega, zato sta si izreka na nek način ekvivalentna. Kar je pomembno za nas je slednje; če dokažemo zgornji izrek, dokažemo tudi izrek \ref{izrek}.

Sedaj si lahko v dokazu pomagamo s posplošenim rezom. To pomeni, da lahko drevo izpeljave preobrazimo tako, da bo namesto prvotnega reza vsebovalo enega ali več rezov \emph{ali posplošenih rezov} nižje stopnje. A to pomeni, da moramo znati poleg navadnega reza sedaj eliminirati še posplošenega. V podpoglavju \ref{gl rez vezniki}, torej pri obravnavi glavnega reza vseh veznikov razen eksponentov, glavnega posplošenega reza niti ne moremo obravnavati. Ta namreč lahko nastopi le, če iz formule režemo eksponente. V nadaljevanju dokaza pa bomo morali biti pazljivi in obdelati še vse primere eliminacije posplošenega reza.

Ostane nam le še definicija stopnje posplošenega reza, saj ta trenutno velja le za navadni rez. V ta namen si oglejmo pravilo Rez!$_n$:
\begin{prooftree}
    \derivation{0}{$\Gamma,(!A)^n \Rightarrow \Delta$}
    \derivation{1}{$\Gamma' \Rightarrow \ !A,\Delta'$}
    \pravilo{Rez!$_n$}
    \BinaryInfC{$\Gamma,\Gamma' \Rightarrow \Delta,\Delta'$}
\end{prooftree}

\begin{definicija}
    \emph{Stopnja posplošenega reza} je par števil $(\R,\h)$, kjer je $\R$ rang formule $!A$ (ali $?A$, glede na vrsto posplošenega reza), $\h$ pa višina drevesa. Stopnja zgornjega posplošenega reza je torej enaka:
    $$
        (\R,\h) = (\R(A) + 1, \h(\D_0) + \h(\D_1)\} + 3)
    $$
\end{definicija}
\begin{opomba}
    Iz definicije je razvidno, da število rezanih formul ne vpliva na stopnjo reza. Če v indukcijskem koraku torej Rez!$_n$ zamenjamo z Rez!$_{n+1}$ na isti višini, režemo pa še zmeraj formulo $!A$, smo stopnjo reza ohranili.
\end{opomba}

Lotimo se še enkrat drevesa izpeljave, kjer na levi $!A$ vpeljemo s skrčitvijo, tokrat s posplošenim pravilom reza v žepu. Obravnavamo lahko kar posplošeni rez za poljuben $n\in\mathbb{N}_{>0}$, vključno z $n = 1$, torej navadnim pravilom reza:
\begin{prooftree}
    \derivation{0}{$\Gamma,(!A)^{n+1} \Rightarrow \Delta$}
    \levopravilo{C!}
    \UnaryInfC{$\Gamma,(!A)^n \Rightarrow \Delta$}

    \derivation{1}{$!\Gamma' \Rightarrow A,?\Delta'$}
    \pravilo{R!}
    \UnaryInfC{$!\Gamma' \Rightarrow \ !A,?\Delta'$}

    \pravilo{Rez!$_n$}
    \BinaryInfC{$\Gamma,!\Gamma' \Rightarrow \Delta,?\Delta'$}
\end{prooftree}
\dol
\begin{prooftree}
    \derivation{0}{$\Gamma,(!A)^{n+1} \Rightarrow \Delta$}

    \derivation{1}{$!\Gamma' \Rightarrow A,?\Delta'$}
    \pravilo{R!}
    \UnaryInfC{$!\Gamma' \Rightarrow \ !A,?\Delta'$}

    \pravilo{Rez!$_{n+1}$}
    \BinaryInfC{$\Gamma,!\Gamma' \Rightarrow \Delta,?\Delta'$}
\end{prooftree}
Zopet se rang formule ohrani, a tokrat je višina novega reza za $1$ nižja od višine prvotnega, torej smo indukcijski predpostavki zadostili. Oglejmo si sedaj glavni rez¸ kjer je formula $!A$ na levi vpeljana z ošibitvijo. Tu korak indukcije za posplošeni rez, kjer $n\neq1$, ter navadni rez ni združljiv, zato primera ločimo, začenši z navadnim pravilom reza:
\begin{prooftree}
    \derivation{0}{$\Gamma \Rightarrow \Delta$}
    \levopravilo{W!}
    \UnaryInfC{$\Gamma,!A \Rightarrow \Delta$}

    \derivation{1}{$!\Gamma' \Rightarrow A,?\Delta'$}
    \pravilo{R!}
    \UnaryInfC{$!\Gamma' \Rightarrow \ !A,?\Delta'$}

    \pravilo{Rez}
    \BinaryInfC{$\Gamma,!\Gamma' \Rightarrow \Delta,?\Delta'$}
\end{prooftree}
\dol
\begin{prooftree}
	\derivation{0}{$\Gamma \Rightarrow \Delta$}
    \levopravilo{W!$\times|\Gamma'|$}
    \UnaryInfC{$\Gamma,!\Gamma' \Rightarrow \Delta$}
    \pravilo{W?$\times|\Delta'|$}
    \UnaryInfC{$\Gamma,!\Gamma' \Rightarrow \Delta,?\Delta'$}
\end{prooftree}
Spet smo na prazno zadostili indukcijski predpostavki in se reza v celoti znebili. Za Rez!$_n$, kjer je $n\geq2$, pa je postopek sledeč:
\begin{prooftree}
    \derivation{0}{$\Gamma,(!A)^n \Rightarrow \Delta$}
    \levopravilo{W!}
    \UnaryInfC{$\Gamma,(!A)^{n+1} \Rightarrow \Delta$}

    \derivation{1}{$!\Gamma' \Rightarrow A,?\Delta'$}
    \pravilo{R!}
    \UnaryInfC{$!\Gamma' \Rightarrow \ !A,?\Delta'$}

    \pravilo{Rez!$_{n+1}$}
    \BinaryInfC{$\Gamma,!\Gamma' \Rightarrow \Delta,?\Delta'$}
\end{prooftree}
\dol
\begin{prooftree}
    \derivation{0}{$\Gamma,(!A)^n \Rightarrow \Delta$}

    \derivation{1}{$!\Gamma' \Rightarrow A,?\Delta'$}
    \pravilo{R!}
    \UnaryInfC{$!\Gamma' \Rightarrow \ !A,?\Delta'$}

    \pravilo{Rez!$_n$}
    \BinaryInfC{$\Gamma,!\Gamma' \Rightarrow \Delta,?\Delta'$}
\end{prooftree}
Stopnja reza je tu znižana na enak način, kot v primeru, ko na levi $!A$ vpeljemo s skrčitvijo. Pri obravnavi glavnega reza formule $!A$ z levim pravilom je zopet potrebno ločiti Rez!$_n$, kjer $n\geq2$, od navadnega reza:
\begin{prooftree}
    \derivation{0}{$\Gamma,A \Rightarrow \Delta$}
    \levopravilo{L!}
    \UnaryInfC{$\Gamma,!A \Rightarrow \Delta$}

    \derivation{1}{$!\Gamma' \Rightarrow A,?\Delta'$}
    \pravilo{R!}
    \UnaryInfC{$!\Gamma' \Rightarrow \ !A,?\Delta'$}

    \pravilo{Rez}
    \BinaryInfC{$\Gamma,!\Gamma' \Rightarrow \Delta,?\Delta'$}
\end{prooftree}
\dol
\begin{prooftree}
    \derivation{0}{$\Gamma,A \Rightarrow \Delta$}
    \derivation{1}{$!\Gamma' \Rightarrow A,?\Delta'$}
    \pravilo{Rez}
    \BinaryInfC{$\Gamma,!\Gamma' \Rightarrow \Delta,?\Delta'$}
\end{prooftree}
Uspelo nam je znižati rang rezane formule, zato je stopnja novega reza nižja. Pri eliminaciji pravila Rez!$_n$, ko je $n\geq2$ se je treba malce bolj potruditi:
\begin{prooftree}
    \derivation{0}{$\Gamma,(!A)^{n-1},A \Rightarrow \Delta$}
    \levopravilo{L!}
    \UnaryInfC{$\Gamma,(!A)^n \Rightarrow \Delta$}

    \derivation{1}{$!\Gamma' \Rightarrow A,?\Delta'$}
    \pravilo{R!}
    \UnaryInfC{$!\Gamma' \Rightarrow \ !A,?\Delta'$}

    \pravilo{Rez!$_n$}
    \BinaryInfC{$\Gamma,!\Gamma' \Rightarrow \Delta,?\Delta'$}
\end{prooftree}
\dol
\begin{prooftree}
    \derivation{0}{$\Gamma,(!A)^{n-1},A \Rightarrow \Delta$}

    \derivation{1}{$!\Gamma' \Rightarrow A,?\Delta'$}
    \pravilo{R!}
    \UnaryInfC{$!\Gamma' \Rightarrow \ !A,?\Delta'$}

    \levopravilo{Rez!$_{n-1}$}
    \BinaryInfC{$\Gamma,!\Gamma',A \Rightarrow \Delta,?\Delta'$}

    \derivation{1}{$!\Gamma' \Rightarrow A,?\Delta'$}
    \pravilo{Rez}
    \BinaryInfC{$\Gamma,!\Gamma',!\Gamma' \Rightarrow \Delta,?\Delta',?\Delta'$}
    \pravilo{C!$\times|\Gamma'|$}
    \UnaryInfC{$\Gamma,!\Gamma' \Rightarrow \Delta,?\Delta',?\Delta'$}
    \pravilo{C?$\times|\Delta'|$}
    \UnaryInfC{$\Gamma,!\Gamma' \Rightarrow \Delta,?\Delta'$}
\end{prooftree}
Novonastalo drevo izpeljave bi nas lahko spominjalo na problem iz začetka tega podpoglavja. Oglejmo si stopnje rezov, kjer prvotni rez zopet označimo z $(\R,\h)$, nova reza pa (po vrsti) z $(\R',\h')$ in $(\R'',\h'')$:
\begin{align*}
    (\R,\h) &= (\R(A) + 1,\h(\D_0) + \h(\D_1) + 5)\\
    (\R',\h') &= (\R(A) + 1,\h(\D_0) + \h(\D_1) + 4)\\
    (\R'',\h'') &= (\R(A),\h(\D_0) + 2*\h(\D_1) + 6)
\end{align*}
Pri prvem izmed novih dreves rang rezane formule ostane isti, vendar se višina zmanjša za $1$, torej je stopnja tega reza res nižja od stopnje prvotnega. Pri drugem rezu spet nastopi težava veliko večje višine, a se je, za razliko problema iz začetka podpoglavja, rang formule znižal. Ker je ureditev leksikografska, se lahko višina poljubno veča; čim je rang rezane formule nižji, bo stopnja reza nižja. Indukcijski predpostavki je torej zadoščeno in ta korak indukcije je opravljen.


\subsubsection{Eliminacija reza, ki ni glaven} \label{non principal}
Če rez ni bil glaven, se je moralo na levi ali desni strani tik pred rezom zgoditi pravilo, ki ni vpeljalo ravnokar rezane formule. Denimo, da je bil na levi ravnokar vpeljan veznik $\sqcap$, nato pa smo rezali z veznikom nepovezano formulo $C$.
\begin{prooftree}
    \AxiomC{$\Gamma,A,C \Rightarrow \Delta$}
    \levopravilo{L$\sqcap$}
    \UnaryInfC{$\Gamma,A \sqcap B,C \Rightarrow \Delta$}

    \AxiomC{$\Gamma' \Rightarrow C,\Delta'$}
    \pravilo{Rez}
    \BinaryInfC{$\Gamma,\Gamma',A \sqcap B \Rightarrow \Delta,\Delta'$}
\end{prooftree}
To lahko, ne glede na to kakšna je formula $C$ in ne glede na to ali je bila na desni vpeljana kakšna druga formula ali ne, preobrazimo tako, da bo zadoščalo indukcijski predpostavki.
\begin{prooftree}
    \AxiomC{$\Gamma,A,C \Rightarrow \Delta$}
    \AxiomC{$\Gamma' \Rightarrow C,\Delta'$}
    \pravilo{Rez}
    \BinaryInfC{$\Gamma,\Gamma',A \Rightarrow \Delta,\Delta'$}
    \pravilo{L$\sqcap$}
    \UnaryInfC{$\Gamma,\Gamma',A \sqcap B \Rightarrow \Delta,\Delta'$}
\end{prooftree}
Rez smo res premaknili višje v drevesu izpeljave in kot lahko vidimo sama formula $C$ za indukcijski korak ni bila pomembna. Zato namesto glede na rezano formulo, ločimo primere glede na ravnokar vpeljano formulo, ki ni $C$. Seveda tudi ni pomembno, ali se ta vpeljava zgodi v levem ali desnem poddrevesu nad rezom, saj je to asimetrično le če obravnavamo rezano formulo. Pomembno pa je, ali je bil veznik vpeljan z levim pravilom vpeljave ali je bil vpeljan z desnim, zato si oglejmo še drugi primer, ko je vpeljana formula $A \sqcap B$.
\begin{prooftree}
    \AxiomC{$\Gamma,C \Rightarrow A,\Delta$}
    \AxiomC{$\Gamma,C \Rightarrow B,\Delta$}
    \levopravilo{R$\sqcap$}
    \BinaryInfC{$\Gamma,C \Rightarrow A \sqcap B,\Delta$}

    \AxiomC{$\Gamma' \Rightarrow C,\Delta'$}
    \pravilo{Rez}
    \BinaryInfC{$\Gamma,\Gamma' \Rightarrow A \sqcap B,\Delta,\Delta'$}
\end{prooftree}
\dol
\begin{prooftree}
    \AxiomC{$\Gamma,C \Rightarrow A,\Delta$}
    \AxiomC{$\Gamma' \Rightarrow C,\Delta'$}
    \levopravilo{Rez}
    \BinaryInfC{$\Gamma,\Gamma' \Rightarrow A,\Delta,\Delta'$}

    \AxiomC{$\Gamma,C \Rightarrow B,\Delta$}
    \AxiomC{$\Gamma' \Rightarrow C,\Delta'$}
    \pravilo{Rez}
    \BinaryInfC{$\Gamma,\Gamma' \Rightarrow B,\Delta,\Delta'$}

    \pravilo{R$\sqcap$}
    \BinaryInfC{$\Gamma,\Gamma' \Rightarrow A \sqcap B,\Delta,\Delta'$}
\end{prooftree}
Tu smo sicer pridelali dva reza, a sta oba višje v drevesu izpeljave, zato zadostimo indukcijski predpostavki. Poglejmo si še, kaj se zgodi če na levi vpeljemo veznik $\star$.
\begin{prooftree}
    \AxiomC{$\Gamma,A,B,C \Rightarrow \Delta$}
    \levopravilo{L$\star$}
    \UnaryInfC{$\Gamma,A \star B,C \Rightarrow \Delta$}

    \AxiomC{$\Gamma' \Rightarrow C,\Delta'$}
    \pravilo{Rez}
    \BinaryInfC{$\Gamma,\Gamma',A \star B \Rightarrow \Delta,\Delta'$}
\end{prooftree}
\dol
\begin{prooftree}
    \AxiomC{$\Gamma,A,B,C \Rightarrow \Delta$}
    \AxiomC{$\Gamma' \Rightarrow C,\Delta'$}
    \pravilo{Rez}
    \BinaryInfC{$\Gamma,\Gamma',A,B \Rightarrow \Delta,\Delta'$}

    \pravilo{L$\star$}
    \UnaryInfC{$\Gamma,\Gamma',A \star B \Rightarrow \Delta,\Delta'$}
\end{prooftree}
Kot pri primeru levega pravila vpeljave za veznik $\sqcap$, pravilo reza ter pravilo vpeljave le zamenjamo, zato enostavno zadostimo indukcijski predpostavki.
\begin{prooftree}
    \AxiomC{$\Gamma,C \Rightarrow A,\Delta$}
    \AxiomC{$\Gamma' \Rightarrow B,\Delta'$}
    \levopravilo{R$\star$}
    \BinaryInfC{$\Gamma,\Gamma',C \Rightarrow A \star B,\Delta,\Delta'$}

    \AxiomC{$\Gamma'' \Rightarrow C,\Delta''$}
    \pravilo{Rez}
    \BinaryInfC{$\Gamma,\Gamma',\Gamma'' \Rightarrow A \star B,\Delta,\Delta',\Delta''$}
\end{prooftree}
\dol
\begin{prooftree}
    \AxiomC{$\Gamma,C \Rightarrow A,\Delta$}
    \AxiomC{$\Gamma'' \Rightarrow C,\Delta''$}
    \levopravilo{Rez}
    \BinaryInfC{$\Gamma,\Gamma'' \Rightarrow A,\Delta,\Delta''$}

    \AxiomC{$\Gamma' \Rightarrow B,\Delta'$}
    \pravilo{R$\star$}
    \BinaryInfC{$\Gamma,\Gamma',\Gamma'' \Rightarrow A \star B,\Delta,\Delta',\Delta''$}
\end{prooftree}
Pri zgornjem koraku indukcije je pomembno, da se formula $C$, ki jo režemo, kot predpostavka dodatno pojavi le v enem izmed skventov $\Gamma \Rightarrow A,\Delta$ ter $\Gamma' \Rightarrow B,\Delta'$, ne v obeh, saj desno pravilo vpeljave veznika $\star$ združi konteksta in bi morali drugače $C$ iz sekventa rezati dvakrat. To pa nam tudi omogoči zelo enostaven korak indukcije, kot je izveden zgoraj. Primeri za veznika $\sqcup$ ter + so popolnoma simetrični prejšnjim štirim primerom, zato jih prepustimo bralcu. Oglejmo si sedaj primer, ko na levi tik pred rezom vpeljemo implikacijo.
\begin{prooftree}
    \AxiomC{$\Gamma,C \Rightarrow A,\Delta$}
    \AxiomC{$\Gamma',B \Rightarrow \Delta'$}
    \levopravilo{L$\multimap$}
    \BinaryInfC{$\Gamma,\Gamma',C,A \multimap B \Rightarrow \Delta,\Delta'$}

    \AxiomC{$\Gamma'' \Rightarrow C,\Delta''$}
    \pravilo{Rez}
    \BinaryInfC{$\Gamma,\Gamma',\Gamma'',A \multimap B \Rightarrow \Delta,\Delta',\Delta''$}
\end{prooftree}
\dol
\begin{prooftree}
    \AxiomC{$\Gamma,C \Rightarrow A,\Delta$}
    \AxiomC{$\Gamma'' \Rightarrow C,\Delta''$}
    \levopravilo{Rez}
    \BinaryInfC{$\Gamma,\Gamma'' \Rightarrow A,\Delta,\Delta''$}

    \AxiomC{$\Gamma',B \Rightarrow \Delta'$}
    \pravilo{L$\multimap$}
    \BinaryInfC{$\Gamma,\Gamma',\Gamma'',A \multimap B \Rightarrow \Delta,\Delta',\Delta''$}
\end{prooftree}
Kot lahko vidimo, je ta primer zelo podoben primeru, ko tik nad rezom vpeljemo $\star$ z desnim pravilom vpeljave. Prav tako je primer, ko je $\multimap$ vpeljan na desni, zelo podoben primeru, ko je $\star$ vpeljan na levi.
\begin{prooftree}
    \AxiomC{$\Gamma,C,A \Rightarrow B,\Delta$}
    \levopravilo{R$\multimap$}
    \UnaryInfC{$\Gamma,C \Rightarrow A \multimap B,\Delta$}

    \AxiomC{$\Gamma' \Rightarrow C,\Delta'$}
    \pravilo{Rez}
    \BinaryInfC{$\Gamma,\Gamma' \Rightarrow A \multimap B,\Delta,\Delta'$}
\end{prooftree}
\dol
\begin{prooftree}
    \AxiomC{$\Gamma,C,A \Rightarrow B,\Delta$}
    \AxiomC{$\Gamma' \Rightarrow C,\Delta'$}
    \pravilo{Rez}
    \BinaryInfC{$\Gamma,\Gamma',A \Rightarrow B,\Delta,\Delta'$}

    \pravilo{R$\multimap$}
    \UnaryInfC{$\Gamma,\Gamma' \Rightarrow A \multimap B,\Delta,\Delta'$}
\end{prooftree}
Zadnji izmed propozicijskih veznikov je zopet negacija. Obravnavali bomo le levo pravilo vpeljave, saj je desno popolnoma simetrično in je zato tudi postopek eliminacije take vrste reza simetričen.
\begin{prooftree}
    \AxiomC{$\Gamma,C \Rightarrow A,\Delta$}
    \levopravilo{L$\negacija$}
    \UnaryInfC{$\Gamma,C,\negacija A \Rightarrow \Delta$}

    \AxiomC{$\Gamma' \Rightarrow C,\Delta'$}
    \pravilo{Rez}
    \BinaryInfC{$\Gamma,\Gamma',\negacija A \Rightarrow \Delta,\Delta'$}
\end{prooftree}
\dol
\begin{prooftree}
    \AxiomC{$\Gamma,C \Rightarrow A,\Delta$}
    \AxiomC{$\Gamma' \Rightarrow C,\Delta'$}
    \pravilo{Rez}
    \BinaryInfC{$\Gamma,\Gamma' \Rightarrow A,\Delta,\Delta'$}
    \pravilo{L$\negacija$}
    \UnaryInfC{$\Gamma,\Gamma',\negacija A \Rightarrow \Delta,\Delta'$}
\end{prooftree}
Oglejmo si še konstante. Tokrat seveda obravnavamo tudi konstanti $\top$ ter $\bot$, saj rez ni glaven in nimamo več težave z dejstvom, da imata vsaka le po eno pravilo vpeljave.
\begin{prooftree}
    \AxiomC{}
    \levopravilo{R$\top$}
    \UnaryInfC{$\Gamma,C \Rightarrow \top,\Delta$}
    \AxiomC{$\Gamma' \Rightarrow C,\Delta'$}
    \pravilo{Rez}
    \BinaryInfC{$\Gamma,\Gamma' \Rightarrow \top,\Delta,\Delta'$}
\end{prooftree}
\dol
\begin{prooftree}
    \AxiomC{}
    \levopravilo{R$\top$}
    \UnaryInfC{$\Gamma,\Gamma' \Rightarrow \top,\Delta,\Delta'$}
\end{prooftree}
Ker lahko $\top$ med sklepi vedno dokažemo, lahko končni sekvent dobimo tudi ne da bi rezali formulo $C$, saj je ta že od začetka med predpostavkami nastala ,,umetno''. Korak indukcije za konstanto $\bot$ izvedemo popolnoma simetrično, saj zanjo velja ista lastnost, le da velja med predpostavkami. Pri obravnavi konstante $\enota$ desnega pravila vpeljave ne moremo obravnavati, saj ob formuli $\enota$ v sekventu ni nobene druge formule, ki bi jo lahko rezali. Zato obravnavamo le levo pravilo vpeljave.
\begin{prooftree}
    \AxiomC{$\Gamma,C \Rightarrow \Delta$}
    \levopravilo{L$\enota$}
    \UnaryInfC{$\Gamma,C,\enota \Rightarrow \Delta$}

    \AxiomC{$\Gamma' \Rightarrow C,\Delta'$}
    \pravilo{Rez}
    \BinaryInfC{$\Gamma,\Gamma',\enota \Rightarrow \Delta,\Delta'$}
\end{prooftree}
\dol
\begin{prooftree}
    \AxiomC{$\Gamma,C \Rightarrow \Delta$}
    \AxiomC{$\Gamma' \Rightarrow C,\Delta'$}
    \pravilo{Rez}
    \BinaryInfC{$\Gamma,\Gamma' \Rightarrow \Delta,\Delta'$}

    \pravilo{L$\enota$}
    \UnaryInfC{$\Gamma,\Gamma',\enota \Rightarrow \Delta,\Delta'$}
\end{prooftree}
Korak indukcije pri konstanti $\nicla$ je simetričen, obravnavamo pa le njeno desno pravilo vpeljave, iz enakih razlogov. Pri kvantifikatorjih je ta korak spet precej enostaven in simetričen za vse štiri primere, zato bomo prikazali le enega, namreč primer, ko na levi tik pred rezom vpeljemo formulo $\forall x A$.
\begin{prooftree}
    \AxiomC{$\Gamma,C,A[t/x] \Rightarrow \Delta$}
    \levopravilo{L$\forall$}
    \UnaryInfC{$\Gamma,C,\forall x A \Rightarrow \Delta$}

    \AxiomC{$\Gamma' \Rightarrow C,\Delta'$}
    \pravilo{Rez}
    \BinaryInfC{$\Gamma,\Gamma',\forall x A \Rightarrow \Delta,\Delta'$}
\end{prooftree}
\dol
\begin{prooftree}
    \AxiomC{$\Gamma,C,A[t/x] \Rightarrow \Delta$}
    \AxiomC{$\Gamma' \Rightarrow C,\Delta'$}
    \pravilo{Rez}
    \BinaryInfC{$\Gamma,\Gamma',A[t/x] \Rightarrow \Delta,\Delta'$}

    \pravilo{L$\forall$}
    \UnaryInfC{$\Gamma,\Gamma',\forall x A \Rightarrow \Delta,\Delta'$}
\end{prooftree}
Lotimo se sedaj eksponentov. Začnimo s primerom, kjer je bila nad rezom ravnokar vpeljana formula $!A$, s skrčitvijo, ošibitvijo ali levim pravilom vpeljave. Vsi ti primeri so si enaki, zato obravnavajmo le skrčitev.
\begin{prooftree}
    \AxiomC{$\Gamma,C,!A,!A \Rightarrow \Delta$}
    \levopravilo{C!}
    \UnaryInfC{$\Gamma,C,!A \Rightarrow \Delta$}

    \AxiomC{$\Gamma' \Rightarrow C,\Delta'$}
    \pravilo{Rez}
    \BinaryInfC{$\Gamma,\Gamma',!A \Rightarrow \Delta,\Delta'$}
\end{prooftree}
\dol
\begin{prooftree}
    \AxiomC{$\Gamma,C,!A,!A \Rightarrow \Delta$}
    \AxiomC{$\Gamma' \Rightarrow C,\Delta'$}
    \pravilo{Rez}
    \BinaryInfC{$\Gamma,\Gamma',!A,!A \Rightarrow \Delta,\Delta'$}
    \pravilo{C!}
    \UnaryInfC{$\Gamma,\Gamma',!A \Rightarrow \Delta,\Delta'$}
\end{prooftree}
Primeri, ko je bila vpeljana formula $?A$ s skrčitvijo, ošibitvijo ali desnim pravilom vpeljave, so prav tako enaki, zato jih ne bomo obravnavali.
Pri desnem pravilu vpeljave fomrule $!A$ ter pri levem pravilu vpeljave formule $?A$ pa se je treba malce bolj potruditi. Primera sta zopet simetrična, zato si oglejmo le desno pravilo vpeljave formule $!A$. Da smo lahko kot sklep sploh vpeljali formulo $!A$, je morala biti formula $C$, ki jo režemo, oblike $!C$.
\begin{prooftree}
    \AxiomC{$!\Gamma,!C \Rightarrow A,?\Delta$}
    \levopravilo{R!}
    \UnaryInfC{$!\Gamma,!C \Rightarrow !A,?\Delta$}

    \AxiomC{$\Gamma' \Rightarrow !C,\Delta'$}
    \pravilo{Rez}
    \BinaryInfC{$!\Gamma,\Gamma' \Rightarrow !A,?\Delta,\Delta'$}
\end{prooftree}
Tu pa nastopi težava, saj drevesa izpeljave ne moremo preoblikovati, ne da bi vedeli kaj več o desnem poddrevesu nad rezom.
\begin{prooftree}
    \AxiomC{$!\Gamma,!C \Rightarrow A,?\Delta$}
    \AxiomC{$\Gamma' \Rightarrow !C,\Delta'$}
    \pravilo{Rez}
    \BinaryInfC{$!\Gamma,\Gamma' \Rightarrow A,?\Delta,\Delta'$}
    \UnaryInfC{?}
\end{prooftree}
Če namreč $\Gamma'$ ni oblike $!\Gamma'$ in $\Delta'$ ni oblike $?\Delta'$, desnega pravila vpeljave za veznik ! ne moremo več uporabiti. Ločiti je treba primer, ko je bila formula $!C$ na desni ravnokar vpeljana z desnim pravilom vpeljave veznika ! in ko ni bila. Če formula $!C$ ni bila ravnokar vpeljana, je moral biti vpeljan nek drug veznik, vse take primere pa smo že obravnavali. Edini primer, ki bi nam lahko povzročal težave, bi bil prav tako na desni vpeljan veznik !, a je to povsem enako primeru, ko je bila formula $!C$ ravnokar vpeljana, zato si oglejmo le slednje.
\begin{prooftree}
    \AxiomC{$!\Gamma,!C \Rightarrow A,?\Delta$}
    \levopravilo{R!}
    \UnaryInfC{$!\Gamma,!C \Rightarrow !A,?\Delta$}

    \AxiomC{$!\Gamma' \Rightarrow C,?\Delta'$}
    \pravilo{R!}
    \UnaryInfC{$!\Gamma' \Rightarrow !C,?\Delta'$}
    \pravilo{Rez}
    \BinaryInfC{$!\Gamma,!\Gamma' \Rightarrow !A,?\Delta,?\Delta'$}
\end{prooftree}
\dol
\begin{prooftree}
    \AxiomC{$!\Gamma,!C \Rightarrow A,?\Delta$}

    \AxiomC{$!\Gamma' \Rightarrow C,?\Delta'$}
    \pravilo{R!}
    \UnaryInfC{$!\Gamma' \Rightarrow !C,?\Delta'$}
    \pravilo{Rez}
    \BinaryInfC{$!\Gamma,!\Gamma' \Rightarrow A,?\Delta,?\Delta'$}
    \pravilo{R!}
    \UnaryInfC{$!\Gamma,!\Gamma' \Rightarrow !A,?\Delta,?\Delta'$}
\end{prooftree}
Enak problem nastane, če je formula $!A$ vpeljana z desnim pravilom vpeljave v desnem poddrevesu nad rezom. V tem primeru je morala formula $C$ biti oblike $?C$ in je bila v levem poddrevesu vpeljana kot predpostavka. To ima seveda enako rešitev kot primer zgoraj.


\subsubsection{Eliminacija posplošenega reza, ki ni glaven}
Kot smo že omenili, ko smo vpeljali posplošeni pravili reza, je potrebno vse korake indukcije opraviti tudi zanju. Pravili Rez!$_n$ ter Rez?$_n$ sta si simetrični, zato bomo zopet obravnavali le pravilo Rez!$_n$. Drevo izpeljave, ki ga torej obravnavamo v tem poglavju je sledeče.
\begin{prooftree}
    \AxiomC{$\Gamma,(!A)^n \Rightarrow \Delta$}
    \AxiomC{$\Gamma' \Rightarrow \ !A,\Delta'$}
    \pravilo{Rez!$_n$}
    \BinaryInfC{$\Gamma,\Gamma' \Rightarrow \Delta,\Delta'$}
\end{prooftree}
Posplošeni glavni rez smo že obravnavali, zato nam zopet preostane primer, ko smo v enem izmed poddreves tik nad rezom ravnokar vpeljali neko drugo formulo. Za razliko od prejšnjega poglavja, je tokrat pomembno, ali se je to zgodilo na levem ali desnem poddrevesu, saj ti nista več simetrični. Če je bila nova formula vpeljana na desni strani nad rezom, je to ne glede na veznik, ki je bil vpeljan, popolnoma enako primerom iz prejšnjega poglavja. Tam smo namreč rezali le eno formulo $C$, za katero nam ni bilo pomembno, ali je oblike $!C$ ali ne, prav tako pa ni bilo pomembno, kakšne oblike ali kako je bil vpeljan $C$ v drugem poddrevesu nad rezom. Edino pravilo vpeljave, ki je bilo na to pozorno, je bilo pravilo R! (ter seveda L?), ki pa se zopet ne moreta pojaviti kot zadnji pravili v desnem poddrevesu, saj nabor sklepov v sekventu $\Gamma' \Rightarrow \ !A,?\Delta'$ ni oblike $?\Delta''$.

Preostane nam torej preveriti, kaj se zgodi, ko je formula $!A$ na v desnem poddrevesu ravnokar vpeljana. Obravnavamo torej naslednje drevo izpeljave.
\begin{prooftree}
    \AxiomC{$\Gamma,(!A)^n \Rightarrow \Delta$}

    \AxiomC{$!\Gamma' \Rightarrow A,?\Delta'$}
    \pravilo{R!}
    \UnaryInfC{$!\Gamma' \Rightarrow \ !A,?\Delta'$}
    \pravilo{Rez!$_n$}
    \BinaryInfC{$\Gamma,!\Gamma' \Rightarrow ?\Delta,\Delta'$}
\end{prooftree}
Sedaj moramo ločiti primere glede na to, kaj je bilo zadnje pravilo uporabljeno v levem poddrevesu nad rezom. Obravnavali bomo le nekaj nazornih primerov, začenši z desnim pravilom za veznik $\sqcap$. Desnega pravila vpeljave formule $!A$ ne bomo pisali vsakič posebej, saj ni relevantno za korak indukcije. Kar je pomembno pri tem, da je bilo to pravilo ravnokar uporabljeno, sta le konteksta $!\Gamma$ ter $?\Delta$, kot bomo videli pri nekaterih izmed obravnavanih primerov.
\begin{prooftree}
    \AxiomC{$\Gamma,(!A)^n \Rightarrow B,\Delta$}
    \AxiomC{$\Gamma,(!A)^n \Rightarrow C,\Delta$}
    \levopravilo{R$\sqcap$}
    \BinaryInfC{$\Gamma,(!A)^n \Rightarrow B \sqcap C,\Delta$}

    \AxiomC{$!\Gamma' \Rightarrow \ !A,?\Delta'$}
    \pravilo{Rez!$_n$}
    \BinaryInfC{$\Gamma,!\Gamma' \Rightarrow B \sqcap C,\Delta,?\Delta'$}
\end{prooftree}
\dol
\begin{prooftree}
    \AxiomC{$\Gamma,(!A)^n \Rightarrow B,\Delta$}
    \AxiomC{$!\Gamma' \Rightarrow !A,?\Delta'$}
    \levopravilo{Rez!$_n$}
    \BinaryInfC{$\Gamma,!\Gamma' \Rightarrow B,\Delta,?\Delta'$}


    \AxiomC{$\Gamma,(!A)^n \Rightarrow C,\Delta$}
    \AxiomC{$!\Gamma' \Rightarrow !A,?\Delta'$}
    \pravilo{Rez!$_n$}
    \BinaryInfC{$\Gamma,!\Gamma' \Rightarrow C,\Delta,?\Delta'$}

    \levopravilo{R$\sqcap$}
    \BinaryInfC{$\Gamma,!\Gamma' \Rightarrow B \sqcap C,\Delta,?\Delta'$}
\end{prooftree}
Kot lahko vidimo je korak indukcije, kjer formule $(!A)^n$ ostanejo nespremenjene, čisto enak primerom, kjer obravnavamo navadni rez, saj dejstvo, da je bilo rezanih več formul naenkrat ne pride v poštev. Zato bomo obravnavali le še primer, kjer se predpostavke v pravilu vpeljave razdelijo na dva dela. Dober primer takega pravila je R$\star$.
\begin{prooftree}
    \AxiomC{$\Gamma,(!A)^p \Rightarrow B,\Delta$}
    \AxiomC{$\Gamma',(!A)^q \Rightarrow C,\Delta'$}
    \levopravilo{R$\star$}
    \BinaryInfC{$\Gamma,\Gamma',(!A)^{p+q} \Rightarrow B \star C,\Delta,\Delta'$}

    \AxiomC{$!\Gamma'' \Rightarrow \ !A,?\Delta''$}
    \pravilo{Rez!$_{p+q}$}
    \BinaryInfC{$\Gamma,\Gamma',!\Gamma'' \Rightarrow B \star C,\Delta,\Delta',?\Delta''$}
\end{prooftree}
\dol
\begin{prooftree} \footnotesize
    \AxiomC{$\Gamma,(!A)^p \Rightarrow B,\Delta$}
    \AxiomC{$!\Gamma'' \Rightarrow !A,?\Delta''$}
    \levopravilo{Rez!$_p$}
    \BinaryInfC{$\Gamma,!\Gamma'' \Rightarrow B,\Delta,?\Delta''$}

    \AxiomC{$\Gamma',(!A)^q \Rightarrow C,\Delta'$}
    \AxiomC{$!\Gamma'' \Rightarrow !A,?\Delta''$}
    \pravilo{Rez!$_q$}
    \BinaryInfC{$\Gamma',!\Gamma'' \Rightarrow C,\Delta',?\Delta''$}

    \pravilo{R$\star$}
    \BinaryInfC{$\Gamma,\Gamma',!\Gamma'',!\Gamma'' \Rightarrow B \star C,\Delta,\Delta',?\Delta'',?\Delta''$}
    \pravilo{C!$\times|\Gamma''|$}
    \UnaryInfC{$\Gamma,\Gamma',!\Gamma'' \Rightarrow B \star C,\Delta,\Delta',?\Delta'',?\Delta''$}
    \pravilo{C?$\times|\Delta''|$}
    \UnaryInfC{$\Gamma,\Gamma',!\Gamma'' \Rightarrow B \star C,\Delta,\Delta',?\Delta''$}
\end{prooftree}
Tu je bilo res pomembno, da so bile predpostavke v desnem poddrevesu oblike $!\Gamma''$, sklepi pa oblike $?\Delta''$, saj smo jih tako lahko dvakrat uporabili.

Z zgornjim poddrevesom smo prikazali, kako bi korak indukcije potekal za vsa pravila, kjer je število formul, ki jih režemo naenkrat, pomembno, torej pravila, kjer se predpostavke delijo na dva. S tem pa smo tudi obdelali vse možne korake indukcije tega dokaza.


\subsubsection{Baza indukcije}
Začnimo z navadnim rezom. Kot omenjeno na začetku tega dokaza, je pri bazi indukcije potrebno preveriti, kaj se zgodi, ko je stopnja reza enaka $(1,\h)$, za nek $h\in\mathbb{N}_{\geq3}$. Denimo, da $\h\neq3$:
\begin{prooftree}
    \derivation{0}{$\Gamma,A \Rightarrow \Delta$}
    \derivation{1}{$\Gamma' \Rightarrow A,\Delta'$}
    \pravilo{Rez}
    \BinaryInfC{$\Gamma,\Gamma' \Rightarrow \Delta,\Delta'$}
\end{prooftree}
To pomeni, da $\h(\D_0) \neq 0$ ali $\h(\D_1) \neq 0$. Brez škode za splošnost lahko predpostavimo, da to velja prvo. Ker je $\R(A) = 1$, je $A$ po definiciji lahko le osnovna formula ali pa ena izmed konstant. Osnovna formula je lahko vpeljana le s strani pravila aksiom, zato je moralo zadnje pravilo vpeljati neko drugo formulo v $\Gamma$ ali $\Delta$. To pa pomeni, da se lahko skličemo na podpoglavje \ref{non principal}. Enako se zgodi, če je bila $A$ ena izmed konstant, ki na levi ni bila vpeljana.

Če pe je bila formula $A$ neka konstanta, ki jo je vpeljalo zadnje pravilo na levi, imamo tri možnosti. Če je bila vpeljana tudi na drugi strani, je to glavni rez konstante in lahko se skličemo na podpoglavje \ref{gl rez vezniki}. Če je bila na desni vpeljana neka druga formula, smo to zopet že obravnavali, tako da nam preostane le primer, ko je na desni pravilo aksioma:
\begin{prooftree}
    \derivation{0}{$\Gamma,A \Rightarrow \Delta$}
    \AxiomC{}
    \pravilo{Ax}
    \UnaryInfC{$A \Rightarrow A$}
    \pravilo{Rez}
    \BinaryInfC{$\Gamma,A \Rightarrow \Delta$}
\end{prooftree}
\dol
\begin{prooftree}
    \derivation{0}{$\Gamma,A \Rightarrow \Delta$}
\end{prooftree}
Kot lahko vidimo je to precej trivialen korak, saj rezanje formule, ki je bila ravnokar vpeljana z aksiomom, nima učinka.
Ostane nam le, da preverimo ali lahko rez eliminiramo, če $\h=3$. Drevo je moralo biti oblike:
\begin{prooftree}
    \AxiomC{}
    \levopravilo{Ax}
    \UnaryInfC{$A \Rightarrow A$}

    \AxiomC{}
    \pravilo{Ax}
    \UnaryInfC{$A \Rightarrow A$}

    \pravilo{Rez}
    \BinaryInfC{$A \Rightarrow A$}
\end{prooftree}
\dol
\begin{prooftree}
    \AxiomC{}
    \pravilo{Ax}
    \UnaryInfC{$A \Rightarrow A$}
\end{prooftree}

Baza indukcije pri posplošenem rezu se obravnava za $\R = 2$, saj je to najnižji možni rang rezane formule pri take vrste rezu. Višina reza ne more biti $3$, saj na levi ni moglo nastopati pravilo aksioma. Vsi argumenti, kako to prevesti na probleme prejšnjih podpoglavij, so enaki kot pri navadnem rezu, kjer $\h\neq3$. Oglejmo si le, kaj se zgodi, če je na desni strani tik nad rezom aksiom:

\begin{prooftree}
    \derivation{0}{$\Gamma,(!A)^n \Rightarrow \Delta$}
    \AxiomC{}
    \pravilo{Ax}
    \UnaryInfC{$!A \Rightarrow !A$}
    \pravilo{Rez!$_n$}
    \BinaryInfC{$\Gamma,!A \Rightarrow \Delta$}
\end{prooftree}
\dol
\begin{prooftree}
    \derivation{0}{$\Gamma,(!A)^n \Rightarrow \Delta$}
    \pravilo{C!$\times (n-1)$}
    \UnaryInfC{$\Gamma,!A \Rightarrow \Delta$}
\end{prooftree}

S tem smo dokaz izreka o eliminaciji rezov zaključili, saj smo obdelali vse primere koraka indukcije ter bazni primer za obe vrsti rezov.



% \section{Povezava med linearno in nelinearno logiko} \label{cl v cll}
% %Tuki citiraj troelstro pa maybe grishina
Dokazali smo, da je linearna logika konsistenten sistem dokazovanja. Kot smo lahko videli, smo imeli največ težav z dokazom pri eliminaciji eksponentov. To nam da misliti, da je dokaz eliminacije rezov v nelinearnem sekventnem računu težji, saj lahko na vsakem koraku uporabimo ošibitev ter skrčitev. Koristno bi torej bilo, če bi lahko že z dokazom eliminacije rezov v linearni logiki poskrbeli za eliminacijo rezov v nelinearni logiki. Zato si oglejmo vložitev običajnega sekventnega računa v linearni sekventni račun.

Vsak sekvent oblike $\Gamma \Rightarrow \Delta$, dokazan v običajnem sekventnem računu bomo torej želeli preobraziti v nek sekvent $\Gamma' \Rightarrow \Delta'$, dokazan linearno. Možnih načinov, kako dobiti ta sekvent je več. Lahko bi enostavno vse predznačili z eksponenti in nato dokazali, da je sekvent $\Gamma' \Rightarrow \Delta'$ res izpeljiv v linearni logiki, a mi se bomo lotili bolj varčne vložitve, ki poudari kater deli dokaza so bili linearni in kateri ne. Za začetek bomo induktivno definirali operaciji $^+$ ter $^-$, ki nam bosta pri vložitvi pomagali.
\begin{definicija}
    Operacija $^+$ je funkcija, ki slika iz običajnega sekventnega računa v linearnega. Njen predpis je definiran induktivno, glede na strukturo formule.
    \begin{enumerate}
        \item Če je $P$ neka osnovna formula, je $P^+ := \nicla\sqcup P$.
    \end{enumerate}
\end{definicija}

- formuliraj izrek gamma -> delta sledi gamma+ -> delta-

- dokazi weakening + contraction na levi, pa use veznike na levi, to je simetricno

Vse kar torej lahko dokažemo nelinearno, lahko dokažemo tudi linearno. To nam po eni strani olajša dokaz eliminacije reza, kot že omenjeno na začetku tega poglavja, po drugi strani pa nam tudi omeji nelinearnost le na del dokaza. To je lahko zelo uporabno pri razumevanju strukture dokaza samega, še posebej če želimo biti pozorni na to, koliko predpostavk uporabimo in kolikokrat dokažemo sklepe.


\end{document}

%TODO general stvar, maybe introduce the fact da so nad vsemi temi pravili se neka druga drevesa izpeljave
%TODO pravilo vpeljave -> logično pravilo
%TODO . -> :
%TODO levo pravilo -> L* etc


%Sources: simpson, troelstra, girard
