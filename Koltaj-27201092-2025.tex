\documentclass[mat1, tisk]{fmfdelo}
% \documentclass[fin1, tisk]{fmfdelo}
% Če pobrišete možnost tisk, bodo povezave obarvane,
% na začetku pa ne bo praznih strani po naslovu, …

\usepackage{bussproofs}
\usepackage{amsmath}
\usepackage{amssymb}
\usepackage{cmll}

\newenvironment{bprooftree}
  {\leavevmode\hbox\bgroup}
  {\DisplayProof\egroup}

\def\fCenter{\mbox{\ $\Rightarrow$\ }}
\newcommand{\pravilo}[1]{\RightLabel{\footnotesize{#1}}}
\newcommand{\levopravilo}[1]{\LeftLabel{\footnotesize{#1}}}
\newcommand{\negacija}{\mathord{\sim}}
\newcommand{\enota}{\mathbf{1}}
\newcommand{\nicla}{\mathbf{0}}
%%%%%%%%%%%%%%%%%%%%%%%%%%%%%%%%%%%%%%%%%%%%%%%%%%%%%%%%%%%%%%%%%%%%%%%%%%%%%%%
% METAPODATKI
%%%%%%%%%%%%%%%%%%%%%%%%%%%%%%%%%%%%%%%%%%%%%%%%%%%%%%%%%%%%%%%%%%%%%%%%%%%%%%%

% - vaše ime
\avtor{Jona Koltaj}

% - naslov dela v slovenščini
\naslov{Eliminacija rezov v linearni logiki}

% - naslov dela v angleščini
\title{Cut elimination in linear logic}

% - ime mentorja/mentorice s polnim nazivom:
%   - doc.~dr.~Ime Priimek
%   - izr.~prof.~dr.~Ime Priimek
%   - prof.~dr.~Ime Priimek
%   za druge variante uporabite ustrezne ukaze
\mentor{prof.~dr.~Andrej Bauer}
% \somentor{...}
% \mentorica{...}
% \somentorica{...}
% \mentorja{...}{...}
% \somentorja{...}{...}
% \mentorici{...}{...}
% \somentorici{...}{...}

% - leto diplome
\letnica{2025}

% - povzetek v slovenščini
%   V povzetku na kratko opišite vsebinske rezultate dela. Sem ne sodi razlaga
%   organizacije dela, torej v katerem razdelku je kaj, pač pa le opis vsebine.
\povzetek{...}

% - povzetek v angleščini
\abstract{...}

% - klasifikacijske oznake, ločene z vejicami
%   Oznake, ki opisujejo področje dela, so dostopne na strani https://www.ams.org/msc/
\klasifikacija{..., ...}

% - ključne besede, ki nastopajo v delu, ločene s \sep
\kljucnebesede{...\sep ...}

% - angleški prevod ključnih besed
\keywords{...\sep ...} % angleški prevod ključnih besed

% - angleško-slovenski slovar strokovnih izrazov
% \slovar{
% \geslo{angleški izraz}{slovenski izraz}
% ...
% }

% - ime datoteke z viri (vključno s končnico .bib), če uporabljate BibTeX
% \literatura{....bib}

%%%%%%%%%%%%%%%%%%%%%%%%%%%%%%%%%%%%%%%%%%%%%%%%%%%%%%%%%%%%%%%%%%%%%%%%%%%%%%%
% DODATNE DEFINICIJE
%%%%%%%%%%%%%%%%%%%%%%%%%%%%%%%%%%%%%%%%%%%%%%%%%%%%%%%%%%%%%%%%%%%%%%%%%%%%%%%

% naložite dodatne pakete, ki jih potrebujete
% \usepackage{...}

% deklarirajte vse matematične operatorje, da jih bo LaTeX pravilno stavil
% \DeclareMathOperator{\...}{...}

% vstavite svoje definicije ...
% \newcommand{\...}{...}

%%%%%%%%%%%%%%%%%%%%%%%%%%%%%%%%%%%%%%%%%%%%%%%%%%%%%%%%%%%%%%%%%%%%%%%%%%%%%%%
% ZAČETEK VSEBINE
%%%%%%%%%%%%%%%%%%%%%%%%%%%%%%%%%%%%%%%%%%%%%%%%%%%%%%%%%%%%%%%%%%%%%%%%%%%%%%%

\begin{document}

% \section{Uvod}
% ...

\section{Sekventni račun} \label{seq_calc}
Sekventni račun je formalni sistem dokazovanja, ki sestoji iz t.~i.\ sekventov in vnaprej določenih pravil, kako jih smemo preoblikovati. Vsak korak dokaza torej uporabi enega izmed teh pravil, dokler začetnega sekventa ali sekventov ne preoblikujemo v tistega, ki smo ga želeli dokazati.

Korake ločimo s horizontalno črto, nad katero so vsi sekventi, ki jih pravilo uporabljeno na tem koraku sprejme, velikokrat pomimenovani \emph{hipoteze}, pod njo pa je novo dobljeni sekvent, navadno imenovan \emph{sklep}. Označimo hipoteze s $\mathcal{H}_0, \mathcal{H}_1, \ldots, \mathcal{H}_n$, sklep pa s $\mathcal{C}$. Korak izpeljave bo torej izgledal takole:

\begin{prooftree}
    \AxiomC{$\mathcal{H}_0$}
    \AxiomC{$\mathcal{H}_1$}
    \AxiomC{$\dots$}
    \AxiomC{$\mathcal{H}_n$}
    \pravilo{Pravilo}
    \QuaternaryInfC{$\mathcal{C}$}
\end{prooftree}

Na desni ponavadi označimo, katero pravilo smo uporabili na tem koraku, zavoljo preglednosti.

\subsection{Sekvent in formula}

\begin{definicija}
    Naj bosta $A_0,A_1,\ldots,A_n$ ter $B_0,B_1,\ldots,B_n$ končni zaporedji logičnih formul. \emph{Sekvent} je izraz oblike $A_0,\dots,A_n \Rightarrow B_0,\dots,B_m$.
\end{definicija}

Formulam na levi strani sekventa navadno pravimo \emph{predpostavke}, formulam na desni pa \emph{sklepi}. Celotno zaporedje predpostavk bomo označevali z $\Gamma$, zaporedje sklepov pa z $\Delta$.

Pomembno je omeniti še, da simbol $\Rightarrow$ v sekventu ne predstavlja običajne implikacije in ga raje beremo kot ,,dokaže''. Vejice na levi strani sekventa se bere kot ,,in'', vejice na desni strani pa kot ,,ali''. Sekvent $A,B \Rightarrow C,D$ bi se torej razumel kot ,,formuli $A$ in $B$ dokažeta formulo $C$ ali formulo $D$''.

V zgornji definiciji smo uporabili besedo \emph{formula}, ki jo je potrebno bolj formalno definirati. Definicija je induktivna, kar pomeni, da se formule gradijo iz podformul, te iz svojih podformul in tako dalje, na dnu pa so t.~i.\ osnovne formule. Te pa so zgrajene iz termov, ki so prav tako induktivno definirani.

%pr obeh tuki lahko citiram simpsonova predavanja
\begin{definicija}
    \emph{Term} je izraz, ki je lahko v treh različnih oblikah.
    \begin{enumerate}
        \item To je lahko neka spremenljivka.
        \item Lahko je konstanta.
        \item Lahko pa je rezultat funkcije, ki sprejme določeno število termov.
    \end{enumerate}
\end{definicija}
\begin{primer*}
    Na voljo imamo denimo naravna števila, na katerih je definirana funkcija $+$, ki sprejme dva terma. Možni termi, ki jih lahko tvorimo, so torej lahko npr.\ spremenljivka $x$, konstanta $3$ ali pa izraz $x+3$.
\end{primer*}

\begin{definicija} \label{formula}
	\emph{Formula} je izraz v dveh oblikah.
	\begin{enumerate}
        \item \emph{Osnovna formula} je predikatni simbol ali relacija, ki sprejme določeno število termov.
        \item \emph{Sestavljena formula} je -- kot pove ime -- sestavljena iz ene ali več podformul, med seboj povezanih z veznikom.
	\end{enumerate}
\end{definicija}
\begin{primer*}
    Imamo denimo terma $t_1$ in $t_2$ in relacijo $=$, ki sprejme dva terma. Tvorimo torej lahko osnovno formulo $t_1=t_2$.
\end{primer*}
\begin{primer*}
    Če imamo dve formuli $A$ in $B$, so npr.\ $A \land B$, $\neg A$, $A \lor B$ tudi formule.
\end{primer*}
Katere sestavljene formule lahko tvorimo je odvisno od tega, s kakšnimi vezniki želimo delati. Če naša logika na primer ne uporablja veznika $\land$, formula $A \land B$ ne pomeni ničesar. Specifične veznike, ki jih bomo uporabljali pri linearni logiki, bomo natančneje definirali v ?? odseku.

\subsection{Pravila pri sekventnem računu}

Pravila pri sekventnem računu delimo na \emph{strukturna pravila}, ki nam povedo kako ravnati s poljubnimi zaporedji formul, \emph{logična pravila} ali \emph{pravila vpeljave}, ki nam povedo kako z različnimi vezniki tvorimo nove formule, in pa pravilo aksioma.

\begin{definicija}
    \emph{Aksiom} je vsak sekvent oblike $A \Rightarrow A$, kar lahko interpretiramo kot ,,formula dokaže sama sebe''. To je seveda vedno res, zato pravilo aksioma, skrajšano $Ax$, pravi, da aksiome lahko vedno tvorimo, t.~j.\ zanje ne potrebujemo predhodnih sekventov. Zapisano v sekventnem računu torej:
    \begin{prooftree}
        \AxiomC{}
        \pravilo{Ax}
        \UnaryInfC{$A \Rightarrow A$}
    \end{prooftree}
\end{definicija}

\subsubsection{Pravila vpeljave}

Pravila vpeljave pri sekventnem računu načeloma sestojijo iz \emph{levega pravila vpeljave} ter \emph{desnega pravila vpeljave}. Prvo nam pove kako veznik uporabiti med predpostavkami, drugo pa kako dani veznik dokazati.

Oglejmo si kot primer pravilo vpeljave za veznik $\land$. Več veznikov bomo vpeljali in si podrobneje pogledali v poglavju ??.

\begin{definicija} \label{inl}
	\emph{Levo pravilo vpeljave veznika $\land$}, krajše $L\land$, pravi, da če znamo nekaj dokazati iz formule $A$, znamo isto dokazati iz $A \land B$ za poljubno formulo $B$. Ker je veznik $\land$ simetričen, je tudi to pravilo simetrično.
	\begin{center}
        \begin{bprooftree}
            \AxiomC{$\Gamma, A \Rightarrow \Delta$}
            \pravilo{L$\land$}
            \UnaryInfC{$\Gamma,A \land B \Rightarrow \Delta$}
        \end{bprooftree}\qquad
        in\qquad
        \begin{bprooftree}
            \AxiomC{$\Gamma, B \Rightarrow \Delta$}
            \pravilo{L$\land$}
            \UnaryInfC{$\Gamma,A \land B \Rightarrow \Delta$}
        \end{bprooftree}
    \end{center}
\end{definicija}

\begin{definicija} \label{inr}
	\emph{Desno pravilo vpeljave veznika $\land$}, krajše $R\land$, pa pravi, da če znamo iz nekih predpostavk dobiti formulo $A$ ter iz istih predpostavk dobiti formulo $B$, znamo iz teh predpostavk dobiti tudi formulo $A \land B$.
	\begin{prooftree}
        \AxiomC{$\Gamma \Rightarrow A,\Delta$}
        \AxiomC{$\Gamma \Rightarrow B,\Delta$}
        \pravilo{R$\land$}
        \BinaryInfC{$\Gamma \Rightarrow A \land B,\Delta$}
    \end{prooftree}
\end{definicija}

\subsubsection{Strukturna pravila}

Običajno je sekventni račun opremljen s tremi strukturnimi pravili.

\begin{definicija}
	\emph{Pravilo menjave}, krajše $Ex$, nam pove, da lahko vrstni red predpostavk in sklepov med seboj poljubno menjamo.
	\begin{center}
        \begin{bprooftree}
            \AxiomC{$\Gamma,A,B,\Gamma' \Rightarrow \Delta$}
            \pravilo{Ex}
            \UnaryInfC{$\Gamma,B,A,\Gamma' \Rightarrow \Delta$}
        \end{bprooftree}\qquad
        in \qquad
        \begin{bprooftree}
            \AxiomC{$\Gamma \Rightarrow \Delta,A,B,\Delta'$}
            \pravilo{Ex}
            \UnaryInfC{$\Gamma \Rightarrow \Delta,B,A,\Delta'$}
        \end{bprooftree}
	\end{center}
\end{definicija}

\begin{opomba}
	Do sedaj smo na $\Gamma$ in $\Delta$ gledali kot zaporedji formul. Če ju namesto tega definiramo kot \emph{multimnožici}, torej množici, kjer je vsakemu elementu prirejeno število pojavitev, lahko pravilo menjave zavržemo, saj sledi že iz same strukture predpostavk in sklepov. Zavoljo enostavnosti bomo torej na $\Gamma$ in $\Delta$ v nadaljevanju gledali kot multimnožici.

	Tu je pomembno, da to ni le množica, saj nas še vedno zanima koliko formul, četudi istih, nastopa v sekventu.
\end{opomba}

\begin{definicija} \label{weakening}
	\emph{Ošibitev}, krajše $W$, nam pove, da lahko tako predpostavke kot sklepe ,,ošibimo'' z dodatno formulo.
    \begin{center}
        \begin{bprooftree}
            \AxiomC{$\Gamma \Rightarrow \Delta$}
            \pravilo{W}
            \UnaryInfC{$\Gamma,A \Rightarrow \Delta$}
        \end{bprooftree} \qquad
        in \qquad
        \begin{bprooftree}
            \AxiomC{$\Gamma \Rightarrow \Delta$}
            \pravilo{W}
            \UnaryInfC{$\Gamma \Rightarrow \Delta, A$}
        \end{bprooftree}
    \end{center}
\end{definicija}

Kar to pomeni na levi je, da če znamo že iz $\Gamma$ dokazati $\Delta$, potem lahko med predpostavke dodamo kakršnokoli odvečno formulo in bomo $\Delta$ še vedno znali dokazati. Odvečne predpostavke nam torej ne škodujejo.

Na desni pa, ker tam vejico beremo kot ,,ali'' velja podobno. Če znamo iz $\Gamma$ dokazati $\Delta$, potem znamo iz $\Gamma$ dokazati tudi $\Delta$ ali $A$.

\begin{definicija} \label{contraction}
    \emph{Skrčitev}, krajše $C$, nam pove, da število ponovitev formule tako med predpostavkami kot sklepi ni pomembno.
    \begin{center}
        \begin{bprooftree}
            \AxiomC{$\Gamma,A,A \Rightarrow \Delta$}
            \pravilo{C}
            \UnaryInfC{$\Gamma,A \Rightarrow \Delta$}
        \end{bprooftree} \qquad
        in \qquad
        \begin{bprooftree}
            \AxiomC{$\Gamma \Rightarrow \Delta,A,A$}
            \pravilo{C}
            \UnaryInfC{$\Gamma \Rightarrow \Delta,A$}
        \end{bprooftree}
    \end{center}
\end{definicija}

Če torej znamo dokazati $\Delta$ iz dveh ponovitev formule $A$, znamo isto dokazati iz le ene kopije. Prav tako, če znamo dvakrat dokazati $A$, znamo to seveda narediti tudi enkrat.

Preden preidemo specifično na linearno logiko je potrebno še omeniti, da pri sekventnem računu dokazujemo ,,od spodaj navzgor''. Začnemo torej s sekventom, ki bi ga želeli dokazati in poiščemo katera pravila, strukturna ali logična, so nam na voljo. Analogija pri dokazovanju v vsakdanji matematiki je, da začnemo s problemom, ki ga želimo dokazati, in ga razčlenimo na manjše podprobleme, dokler ne dobimo nečesa, za kar gotovo vemo da je res. Prav tako poskušamo pri sekventnem računu sekvente postopoma prevesti na aksiom, ki pa bo vedno veljal.

Tudi pravila si zato lahko interpretiramo drugače. Levo pravilo za veznik $\land$ iz definicije \ref{inl} lahko sedaj razumemo kot; če želimo iz $A \land B$ dokazati neke sklepe $\Delta$ je dovolj da $\Delta$ dokažemo iz formule $A$ ali iz formule $B$. Desno pravilo za $\land$ iz definicije \ref{inr} pa razumemo kot; če želimo iz predpostavk $\Gamma$ dokazati $A \land B$, je dovolj da iz $\Gamma$ dokažemo $A$ ter da iz $\Gamma$ dokažemo $B$.

Za primer dokaza v sekventnem računu si oglejmo skoraj trivialen dokaz komutativnosti veznika $\land$.

\begin{prooftree}
    \AxiomC{}
    \pravilo{Ax}
    \UnaryInfC{$B \Rightarrow B$}
    \pravilo{L$\land$}
    \UnaryInfC{$A \land B \Rightarrow B$}

    \AxiomC{}
    \pravilo{Ax}
    \UnaryInfC{$A \Rightarrow A$}
    \pravilo{L$\land$}
    \UnaryInfC{$A \land B \Rightarrow A$}

    \pravilo{R$\land$}
    \BinaryInfC{$A \land B \Rightarrow B \land A$}
\end{prooftree}

V besedah lahko ta dokaz razumemo sledeče. Želimo izpeljati, da $A \land B$ dokaže $B \land A$. Dovolj je, da dokažemo, da $A \land B$ dokaže $A$ ter da dokaže $B$. Na levi je potem dovolj pokazati, da že $B$ dokaže $B$, kar pa je vedno res. Na desni se zgodi podobno.

Od sedaj naprej bomo vsa pravila in dokaze interpretirali od spodaj navzgor.


\section{Linearna logika}
Linearna logika je podzvrst logike sekventnega računa, kjer zavržemo pravili ošibitve in skrčitve iz definicij \ref{weakening} in \ref{contraction}. To pomeni, da moramo vsako predpostavko uporabiti natanko enkrat ter da ne smemo imeti odvečnih predpostavk. Prav tako moramo vsak sklep dokazati natanko enkrat, brez odvečnih sklepov.

Za primer si oglejmo še eno možno definicijo veznika $\land$, poleg \ref{inl} ter \ref{inr} , označimo ga v tem primeru z $\land'$.
\begin{definicija} \label{in'l}
    \emph{Levo pravilo vpeljave veznika $\land'$} pravi, če želimo iz $A \land' B$ dokazati $\Delta$, lahko veznik na levi enostavno prevedemo nazaj v vejico in iz $A$ ter $B$ dokazujemo $\Delta$.
    \begin{prooftree}
        \AxiomC{$\Gamma,A,B \Rightarrow \Delta$}
        \pravilo{L$\land'$}
        \UnaryInfC{$\Gamma,A \land' B \Rightarrow \Delta$}
    \end{prooftree}
\end{definicija}

\begin{definicija} \label{in'r}
	\emph{Desno pravilo vpeljave veznika $\land'$} pa pravi, da če želimo $A \land' B$ dokazati, lahko predpostavke (in preostale sklepe, ki niso povezani z $A$ in $B$) ločimo na dva dela ter z enim delom dokažemo $A$, z drugim pa $B$. Obratno gledano, če znamo iz $\Gamma$ dokazati $A$ ter iz $\Gamma'$ dokazati $B$, lahko predpostavke združimo in dokažemo $A \land' B$.
    \begin{prooftree}
        \AxiomC{$\Gamma \Rightarrow A,\Delta$}
        \AxiomC{$\Gamma' \Rightarrow B,\Delta'$}
        \pravilo{R$\land'$}
        \BinaryInf$\Gamma,\Gamma' \fCenter A \land' B,\Delta,\Delta'$
    \end{prooftree}
\end{definicija}

\begin{lema}
    Če dopustimo uporabo ošibitve in skrčitve, sta si levi in desni pravili vpeljave za $\land$ ter $\land'$ ekvivalentni.
\end{lema}
\begin{dokaz}
    Začnimo z dokazom ekvivalence levih pravil za veznika $\land$ ter $\land'$. Dokaz ekvivalence v tem kontekstu pomeni, da sta pravili medsebojno izpeljivi.
    Izpeljava levega pravila za $\land'$ iz definicije \ref{in'l} na podlagi levega pravila za $\land$ iz definicije \ref{inl} poteka s pomočjo skrčitve.
    \begin{prooftree}
        \AxiomC{$\Gamma,A,B \Rightarrow \Delta$}
        \pravilo{L$\land$}
        \UnaryInfC{$\Gamma,A \land B,B \Rightarrow \Delta$}
        \pravilo{L$\land$}
        \UnaryInfC{$\Gamma,A \land B, A \land B \Rightarrow \Delta$}
        \pravilo{C}
        \UnaryInfC{$\Gamma,A \land B \Rightarrow \Delta$}
    \end{prooftree}
    Najprej torej predpostavko $A \land B$ ,,podvojimo'', nato pa dvakrat uporabimo levo pravilo za $\land$, vsakič ne eni izmed podvojenih predpostavk.

    Obratna izpeljava pa poteka s pomočjo ošibitve. Tu najprej uporabimo levo pravilo za $\land'$, nato pa odvečno izmed predpostavk odstranimo s pomočjo ošibitve.
    \begin{prooftree}
        \AxiomC{$\Gamma,A \Rightarrow \Delta$}
        \pravilo{W}
        \UnaryInfC{$\Gamma,A,B \Rightarrow \Delta$}
        \pravilo{L$\land'$}
        \UnaryInfC{$\Gamma,A \land' B \Rightarrow \Delta$}
    \end{prooftree}

    Podobno dokažemo ekvivalenco desnega pravila za $\land$ iz definicije \ref{inr} ter desnega pravila za $\land'$ iz definicije \ref{in'r}.
    \begin{prooftree}
        \AxiomC{$\Gamma \Rightarrow A,\Delta$}
        \levopravilo{W$\times |\Gamma' \cup \Delta'|$}
        \UnaryInfC{$\Gamma,\Gamma' \Rightarrow A,\Delta,\Delta'$}

        \AxiomC{$\Gamma' \Rightarrow B,\Delta'$}
        \pravilo{W$\times |\Gamma \cup \Delta|$}
        \UnaryInfC{$\Gamma,\Gamma' \Rightarrow B,\Delta,\Delta'$}

        \pravilo{R$\land$}
        \BinaryInf$\Gamma,\Gamma' \fCenter A \land B,\Delta,\Delta'$
    \end{prooftree}
    Ko zgoraj iz desnega pravila za $\land$ izpeljujemo desno pravilo za $\land'$, najprej uporabimo desno pravilo za $\land$, torej predpostavk (in sklepov) ne razdelimo na dva dela, zato se s pomočjo ošibitve na levi ,,znebimo'' (saj pravila beremo od spodaj navzgor) predpostavk $\Gamma'$ in sklepov $\Delta'$. To naredimo tako, da ošibitev iteriramo tolikokrat, kolikor je velikost multimnožice $\Gamma'\cup\Delta'$. Podobno naredimo na desni strani.

    Ko pa iz desnega pravila za $\land'$ izpeljujemo desno pravilo za $\land$, najprej ,,podvojimo'' vse predpostavke v $\Gamma$ in vse sklepe v $\Delta$, nato pa uporabimo desno pravilo za $\land'$ in podvojene predpostavke spet razpolovimo.
    \begin{prooftree}
        \AxiomC{$\Gamma \Rightarrow A,\Delta$}
        \AxiomC{$\Gamma \Rightarrow B,\Delta$}
        \pravilo{R$\land'$}
        \BinaryInf$\Gamma, \Gamma \fCenter A \land' B, \Delta, \Delta$
        \pravilo{C$\times |\Gamma \cup \Delta|$}
        \UnaryInfC{$\Gamma \Rightarrow A \land' B, \Delta$}
    \end{prooftree}
\end{dokaz}
\begin{opomba}
	V zgornjem dokazu smo se malce podrobneje spustili v intuicijo posameznega dela dokaza, saj je to prvi formalen dokaz v sekventnem računu v tem delu. V nadaljnem je intuicija za posamezne vrstice dokaza načeloma prepuščena bralcu.
\end{opomba}

Kot smo v zgornjem dokazu lahko videli, se za dokaz ekvivalence $\land$ ter $\land'$ na bistven način uporabi tako skrčitev kot ošibitev. Slutimo lahko, da brez teh dveh pravil veznika pravzaprav nista ekvivalentna, kar se tudi izkaže za resnično ??cite??. Zato sta sta ta dva veznika v linearni logiki dva različna veznika

\subsection{Propozicijski vezniki}

\begin{definicija}
    Veznik $\land$, s pravili iz definicij \ref{inl} in \ref{inr}, se še vedno glasi \emph{in}, zapišemo pa ga s simbolom $\sqcap$. Zapišimo še enkrat njegovo levo in desno pravilo, tokrat s pravilnim pojmovanjem.
    \begin{center}
        \begin{bprooftree}
            \AxiomC{$\Gamma, A \Rightarrow \Delta$}
            \pravilo{L$\sqcap$}
            \UnaryInfC{$\Gamma,A \sqcap B \Rightarrow \Delta$}
        \end{bprooftree}
        \begin{bprooftree}
            \AxiomC{$\Gamma, B \Rightarrow \Delta$}
            \pravilo{L$\sqcap$}
            \UnaryInfC{$\Gamma,A \sqcap B \Rightarrow \Delta$}
        \end{bprooftree}
        \begin{bprooftree}
            \AxiomC{$\Gamma \Rightarrow A,\Delta$}
            \AxiomC{$\Gamma \Rightarrow B,\Delta$}
            \pravilo{R$\sqcap$}
            \BinaryInf$\Gamma \fCenter A \sqcap B,\Delta$
        \end{bprooftree}
    \end{center}
\end{definicija}

\begin{definicija}
    Veznik $\land'$, s pravili iz definicij \ref{in'l} in \ref{in'r} pa preimenujemo v \emph{tenzor} ter ga zapišemo s simbolom $\star$.
    \begin{center}
        \begin{bprooftree}
            \AxiomC{$\Gamma,A,B \Rightarrow \Delta$}
            \pravilo{L$\star$}
            \UnaryInfC{$\Gamma,A \star B \Rightarrow \Delta$}
        \end{bprooftree}
        \begin{bprooftree}
            \AxiomC{$\Gamma \Rightarrow A,\Delta$}
            \AxiomC{$\Gamma' \Rightarrow B,\Delta'$}
            \pravilo{R$\star$}
            \BinaryInf$\Gamma,\Gamma' \fCenter A \star B,\Delta,\Delta'$
        \end{bprooftree}
    \end{center}
\end{definicija}

Zakaj te dva veznika v kontekstu linearne logike nista enaka je razvidno že če primerjamo njuni levi in desni pravili. Kot smo omenili na začetku tega poglavja je pomembno, da vsako predpostavko uporabimo natanko enkrat. Veznik $\sqcap$ med predpostavkami na nek način vsebuje le eno izmed predpostavk, ki ju združuje, medtem ko veznik $\star$ vsebuje obe. Ko torej uporabimo $A \sqcap B$¸ da dokažemo neki $\Delta$, uporabimo le $A$ ali $B$, medtem ko pri $A \star B$ uporabimo tako $A$ kot $B$. Če pa želimo dokazati, da velja $A \sqcap B$, pa je dovolj, da iz istih predpostavk dokažemo $A$ ter $B$, prav tako ostale sklepe na desni strani sekventa pustimo pri miru. To spet implicira, da vsebuje $A\sqcap B$ enako število informacij kot le $A$ ali $B$. Če pa dokazujemo $A\star B$, pa moramo posebej dokazati $A$, nato pa iz ločenega sklopa predpostavk dokazati $B$. Ostale sklepe poleg $A\star B$ je tudi potrebno posebej dokazati. Vse to spet implicira, da vsebuje tenzor informacij tako za $A$ kot $B$.

Oglejmo si sedaj še preostale veznike, začenši z veznikom $\lor$. V linearni logiki se ta spet razdeli na dvoje, intuicija za to pa je simetična intuiciji za veznik $\land$, zato jo prepustimo bralcu.

\begin{definicija}
	Veznik \emph{ali}, označen z $\sqcup$, ima sledeče levo in desno pravilo vpeljave.
	\begin{center}
        \begin{bprooftree}
            \AxiomC{$\Gamma,A \Rightarrow \Delta$}
            \AxiomC{$\Gamma,B \Rightarrow \Delta$}
            \pravilo{L$\sqcup$}
            \BinaryInf$\Gamma,A \sqcup B \fCenter \Delta$
        \end{bprooftree}
        \begin{bprooftree}
            \AxiomC{$\Gamma \Rightarrow A,\Delta$}
            \pravilo{R$\sqcup$}
            \UnaryInfC{$\Gamma \Rightarrow A \sqcup B,\Delta$}
        \end{bprooftree}
        \begin{bprooftree}
            \AxiomC{$\Gamma \Rightarrow B, \Delta$}
            \pravilo{R$\sqcup$}
            \UnaryInfC{$\Gamma \Rightarrow A \sqcup B,\Delta$}
        \end{bprooftree}
    \end{center}
    Kot vidimo sta obe pravili popolnoma simetrični praviloma za veznik $\sqcap$.
\end{definicija}

\begin{definicija}
	Veznik \emph{plus}, označen z + pa je analogno simetričen vezniku $\star$.
	\begin{center}
        \begin{bprooftree}
            \AxiomC{$\Gamma,A \Rightarrow \Delta$}
            \AxiomC{$\Gamma',B \Rightarrow \Delta'$}
            \pravilo{L+}
            \BinaryInf$\Gamma,\Gamma',A + B \fCenter \Delta,\Delta'$
        \end{bprooftree}
        \begin{bprooftree}
            \AxiomC{$\Gamma \Rightarrow A,B,\Delta$}
            \pravilo{R+}
            \UnaryInfC{$\Gamma \Rightarrow A + B,\Delta$}
        \end{bprooftree}
    \end{center}
\end{definicija}

Vsi nadaljni vezniki imajo v linearni logiki enaka pravila vpeljave kot v navadnem sekventnem računu in se ne delijo na dva dela, še vseeno pa so to \emph{linearni} vezniki, že samo zaradi pogojev pod katerimi so vpeljani. Če na primer $A$ linearno implicira $B$ to pomeni, da natanko en $A$ implicira natanko en $B$, linearna negacija formule $A$ pa negira natanko en $A$.

Za vpeljavo implikacije zopet potrebujemo drugačen simbol kot smo ga vajeni, saj se $\Rightarrow$ že uporablja v strukturi sekventa samega. Običajni sekventni račun v ta namen uporablja $\rightarrow$, linearna implikacija pa, da se loči od nelinearne, spet uporabi svoj simbol.

\begin{definicija}
	\emph{Implikacija}, označena s simbolom $\multimap$, je vpeljana z naslednjimi pravili.
    \begin{center}
        \begin{bprooftree}
            \AxiomC{$\Gamma \Rightarrow A,\Delta$}
            \AxiomC{$\Gamma',B \Rightarrow \Delta'$}
            \pravilo{L$\multimap$}
            \BinaryInf$\Gamma,\Gamma',A \multimap B \fCenter \Delta,\Delta'$
        \end{bprooftree}
        \begin{bprooftree}
            \AxiomC{$\Gamma,A \Rightarrow B,\Delta$}
            \pravilo{R$\multimap$}
            \UnaryInfC{$\Gamma \Rightarrow A \multimap B,\Delta$}
        \end{bprooftree}
    \end{center}
    Kot lahko vidimo je desno pravilo vpeljave dokaj jasno za interpretacijo. Če dokazujemo $A \multimap B$, je dovolj da pod predpostavko $A$ dokažemo $B$. Levo pravilo pa je morda lažje brati od zgoraj navzdol. Če torej z $\Gamma$ dokažemo $A$ ter neke druge sklepe $\Delta$, iz $\Gamma'$ in $B$ pa dokažemo $\Delta'$, lahko iz združenih predpostavk $\Gamma,\Gamma'$ ter dejstva, da iz $A$ sledi $B$ dokažemo združene sklepe $\Delta,\Delta'$.
\end{definicija}

\begin{definicija}
    \emph{Negacija}, označena s simbolom $\negacija$, ima naslednji pravili vpeljave.
    \begin{center}
        \begin{bprooftree}
            \AxiomC{$\Gamma \Rightarrow A,\Delta$}
            \pravilo{L$\negacija$}
            \UnaryInfC{$\Gamma,\negacija A \Rightarrow \Delta$}
        \end{bprooftree}
        \begin{bprooftree}
            \AxiomC{$\Gamma,A \Rightarrow \Delta$}
            \pravilo{R$\negacija$}
            \UnaryInfC{$\Gamma \Rightarrow \negacija A,\Delta$}
        \end{bprooftree}
    \end{center}
    ?? Kaj je tuki sploh intuicija lmao??
\end{definicija}

\subsection{Propozicijske konstante}

V običajnem sekventnem računu imamo dvoje konstant; resnico in neresnico, ki pa se v linearni logiki spet vsaka razdelita na dvoje. Resnici se delita na enoto za $\star$ ter enoto za $\sqcap$, neresnici pa na enoto za + ter enoto za $\sqcup$.

\begin{definicija}
    \emph{Enota}, označena z $\enota$, ima levo in desno pravilo vpeljave:
    \begin{center}
        \begin{bprooftree}
            \AxiomC{$\Gamma \Rightarrow \Delta$}
            \pravilo{L$\enota$}
            \UnaryInfC{$\Gamma,\enota \Rightarrow \Delta$}
        \end{bprooftree}
        \begin{bprooftree}
            \AxiomC{}
            \pravilo{R$\enota$}
            \UnaryInfC{$\Rightarrow \enota$}
        \end{bprooftree}
    \end{center}
    Enoto torej lahko brez predpostavk vedno dokažemo, kar nam pove desno pravilo, če pa vemo da enota velja je to trivialna informacija, kar nam pove levo pravilo.
\end{definicija}

\begin{definicija}
    \emph{Resnica}, označena z $\top$, ima le desno pravilo vpeljave. Ne moremo je torej uporabiti kot sklep.
    \begin{prooftree}
        \AxiomC{}
        \pravilo{R$\top$}
        \UnaryInfC{$\Gamma \Rightarrow \top,\Delta$}
    \end{prooftree}
    Kar nam to pove je, da resnica vedno velja.
\end{definicija}

\begin{lema} \label{enoti}
	Enota $\enota$ je enota za za $\star$, resnica $\top$ pa je enota za $\sqcap$.
\end{lema}
\begin{dokaz}
    Za dokaz leme potrebujemo izpeljati sekvente $A \star \enota \Rightarrow A$, $A \Rightarrow A \star \enota$, $A \sqcap \top \Rightarrow A$ ter $A \Rightarrow A \sqcap \top$.
    \begin{center}
        \vskip 10pt
        \begin{bprooftree}
            \AxiomC{}
            \pravilo{Ax}
            \UnaryInfC{$A \Rightarrow A$}
            \pravilo{L$\enota$}
            \UnaryInfC{$A,\enota \Rightarrow A$}
            \pravilo{L$\star$}
            \UnaryInfC{$A \star \enota \Rightarrow A$}
        \end{bprooftree}
        \begin{bprooftree}
            \AxiomC{}
            \levopravilo{Ax}
            \UnaryInfC{$A \Rightarrow A$}

            \AxiomC{}
            \pravilo{R$\enota$}
            \UnaryInfC{$\Rightarrow \enota$}

            \pravilo{R$\star$}
            \BinaryInf$A \fCenter A \star \enota$
        \end{bprooftree}
    \end{center}
    \vskip 10pt
    \begin{center}
        \begin{bprooftree}
            \AxiomC{}
            \pravilo{Ax}
            \UnaryInfC{$A \Rightarrow A$}
            \pravilo{L$\sqcap$}
            \UnaryInfC{$A \sqcap \top \Rightarrow A$}
        \end{bprooftree}
        \begin{bprooftree}
            \AxiomC{}
            \levopravilo{Ax}
            \UnaryInfC{$A \Rightarrow A$}

            \AxiomC{}
            \pravilo{R$\top$}
            \UnaryInfC{$A \Rightarrow \top$}

            \pravilo{R$\sqcap$}
            \BinaryInf$A \fCenter A \sqcap \top$
        \end{bprooftree}
    \end{center}
\end{dokaz}

Pri neresnici je razlog za razdvojitev enak, pravila vpeljave pa so simetrična, zato interpretacijo prepustimo bralcu.

\begin{definicija}
	\emph{Ničla}, označena z $\nicla$ ima levo in desno pravilo vpeljave:
	 \begin{center}
        \begin{bprooftree}
            \AxiomC{}
            \pravilo{L$\nicla$}
            \UnaryInfC{$\nicla \Rightarrow$}
        \end{bprooftree}
        \begin{bprooftree}
            \AxiomC{$\Gamma \Rightarrow \Delta$}
            \pravilo{R$\nicla$}
            \UnaryInfC{$\Gamma \Rightarrow \nicla,\Delta$}
        \end{bprooftree}
    \end{center}
\end{definicija}
\begin{definicija}
    \emph{Neresnica}, označena z $\bot$, ima le levo pravilo vpeljave. Ne moremo je torej uporabiti kot predpostavko.
    \begin{prooftree}
        \AxiomC{}
        \pravilo{L$\bot$}
        \UnaryInfC{$\Gamma, \bot \Rightarrow \Delta$}
    \end{prooftree}
\end{definicija}

Dokaz naslednje leme bomo opustili, saj je simetričen dokazu leme \ref{enoti}.
\begin{lema}
	Ničla $\nicla$ je enota za +, neresnica $\bot$ pa je enota za $\sqcup$.
\end{lema}

\subsection{Kvantifikatorja}

\begin{definicija}
	\emph{Univerzalni kvantifikator}, označen kot navadno s simbolom $\forall$, je definiran z naslednjima praviloma vpeljave. Tu $y$ ne sme biti prost v $\Gamma$ in $\Delta$.
	%notacija tuki bo t/x means substitute vsak x s tjem
	%also citiras simpsona as well
	\begin{center}
        \begin{bprooftree}
            \AxiomC{$\Gamma, A[t/x] \Rightarrow \Delta$}
            \pravilo{L$\forall$}
            \UnaryInfC{$\Gamma,\forall x A \Rightarrow \Delta$}
        \end{bprooftree}
        \begin{bprooftree}
            \AxiomC{$\Gamma \Rightarrow A[y/x],\Delta$}
            \pravilo{R$\forall$}
            \UnaryInfC{$\Gamma \Rightarrow \forall x A,\Delta$}
        \end{bprooftree}
	\end{center}
	Notacija $A[y/x]$ pomeni, da vsako instanco spremenljivke $x$ zamenjamo s spremenljivko $y$. Spremenljivka $t$ v definiciji označuje nek specifičen term $t$, ki si ga izberemo. Levo pravilo vpeljave torej pomeni, da če želimo iz dejstva, da za vsak $x$ velja formula $A$ dokazati $\Delta$, je dovolj, da spremenljivko $x$ v $A$ zamenjamo z nekim specifičnim termom in z njim dokažemo $\Delta$. Spremenljivka $y$ v definiciji pa označuje prosto spremenljivko. Desno pravilo vpeljave je ekvivalentno temu, da pri dokazovanju, da za vsak $x$ velja $A$, fiksiramo poljuben $y$ in dokazujemo $A$.
\end{definicija}

\begin{definicija}
    \emph{Eksistenčni kvantifikator} je spet brez sprememb označen s simbolom $\exists$. Spremenljivka $y$ spet ne sme  biti prosta v $\Gamma$ ter $\Delta$.
    \begin{center}
        \begin{bprooftree}
            \AxiomC{$\Gamma,A[y/x] \Rightarrow \Delta$}
            \pravilo{L$\exists$}
            \UnaryInfC{$\Gamma,\exists x A \Rightarrow \Delta$}
        \end{bprooftree}
        \begin{bprooftree}
            \AxiomC{$\Gamma \Rightarrow A[t/x],\Delta$}
            \pravilo{R$\exists$}
            \UnaryInfC{$\Gamma \Rightarrow \exists x A,\Delta$}
        \end{bprooftree}
	\end{center}
	Tokrat levo pravilo vsebuje prosto spremenjlivko $y$, desno pa specifičen term $t$. Če torej želimo dejstvo, da obstaja $x$, da velja $A$.
\end{definicija}




%TODO napisi na koncu Osnovni vezniki se da mas drugacno notacijo kot girard
%TODO napisi se kako se intuicionisticno logiko dela in kaj to je, pa zakaj ne bomo obravnaval
%TODO zapisi se, da ni nujno da se linearno logiko dela v sekventnem racunu ampak da je lahko tut v natural deductionu.


% \section{Zaključek}
% ...

\end{document}
