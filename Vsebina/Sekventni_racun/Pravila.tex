Pravila sekventnega računa delimo na \emph{pravilo aksioma}, \emph{logična pravila}, ki nam povedo kako z različnimi vezniki tvorimo nove formule in pa \emph{strukturna pravila}, ki nam povedo kako ravnati s poljubnimi multimnožicami formul.

\subsubsection{Pravilo aksioma}

\begin{definicija}
    \emph{Aksiom} je vsak sekvent oblike $A \Rightarrow A$, \emph{pravilo aksioma}, krajše $Ax$, pa pravi:
    \begin{prooftree}
        \AxiomC{}
        \pravilo{Ax}
        \UnaryInfC{$A \Rightarrow A$}
    \end{prooftree}
    Aksiom lahko interpretiramo kot ,,formula dokaže samo sebe'', kar pa seveda vedno velja. Zato pravilo aksioma pove, da lahko take vrste sekvent vedno tvorimo t.~j.\ da zanj ne potrebujemo predhodnih sekventov.
\end{definicija}

\subsubsection{Logična pravila}

Logična pravila praviloma sestojijo iz \emph{levega pravila} in \emph{desnega pravila}. Prvo nam pove kako veznik uporabiti med predpostavkami, drugo pa kako dani veznik dokazati. Oglejmo si kot primer logični pravili za veznik $\land$.

\begin{definicija} \label{inl}
	\emph{Levo pravilo za veznik $\land$}, krajše $L\land$:
	\begin{center}
        \begin{bprooftree}
            \AxiomC{$\Gamma, A \Rightarrow \Delta$}
            \pravilo{L$\land$}
            \UnaryInfC{$\Gamma,A \land B \Rightarrow \Delta$}
        \end{bprooftree}\qquad
        in\qquad
        \begin{bprooftree}
            \AxiomC{$\Gamma, B \Rightarrow \Delta$}
            \pravilo{L$\land$}
            \UnaryInfC{$\Gamma,A \land B \Rightarrow \Delta$}
        \end{bprooftree}
    \end{center}
    To pomeni da če znamo nekaj dokazati iz formule $A$, znamo isto dokazati iz $A \land B$ za poljubno formulo $B$. Ker je veznik $\land$ simetričen, je tudi to pravilo simetrično.
\end{definicija}

\begin{definicija} \label{inr}
	\emph{Desno pravilo za veznik $\land$}, krajše $R\land$:
	\begin{prooftree}
        \AxiomC{$\Gamma \Rightarrow A,\Delta$}
        \AxiomC{$\Gamma \Rightarrow B,\Delta$}
        \pravilo{R$\land$}
        \BinaryInfC{$\Gamma \Rightarrow A \land B,\Delta$}
    \end{prooftree}
    To pa pomeni da če znamo iz nekih predpostavk dobiti formulo $A$ in iz istih predpostavk dobiti formulo $B$, znamo iz teh predpostavk dobiti tudi formulo $A \land B$.
\end{definicija}

\subsubsection{Strukturna pravila}

Običajno je sekventni račun opremljen z dvema strukturnima praviloma.

\begin{definicija} \label{weakening}
	\emph{Ošibitev}, krajše $W$, nam pove, da lahko tako predpostavke kot sklepe ,,ošibimo'' z dodatno formulo:
    \begin{center}
        \begin{bprooftree}
            \AxiomC{$\Gamma \Rightarrow \Delta$}
            \pravilo{W}
            \UnaryInfC{$\Gamma,A \Rightarrow \Delta$}
        \end{bprooftree} \qquad
        in \qquad
        \begin{bprooftree}
            \AxiomC{$\Gamma \Rightarrow \Delta$}
            \pravilo{W}
            \UnaryInfC{$\Gamma \Rightarrow \Delta, A$}
        \end{bprooftree}
    \end{center}
    Levo pravilo pomeni, da smemo med predpostavke dodati odvečne formule, ki jih nikjer ne uporabimo. Ker na desni vejice beremo kot ,,ali'', je dovolj da dokažemo vsaj enega izmed sklepov, ne pa nujno vse. To pa je ravno bistvo desnega pravila, saj če smo namreč iz $\Gamma$ že dokazali $\Delta$, znamo seveda dokazati tudi $\Delta$ ali $A$.
\end{definicija}

\begin{definicija} \label{contraction}
    \emph{Skrčitev}, krajše $C$, nam pove, da število ponovitev formule tako med predpostavkami kot sklepi ni pomembno:
    \begin{center}
        \begin{bprooftree}
            \AxiomC{$\Gamma,A,A \Rightarrow \Delta$}
            \pravilo{C}
            \UnaryInfC{$\Gamma,A \Rightarrow \Delta$}
        \end{bprooftree} \qquad
        in \qquad
        \begin{bprooftree}
            \AxiomC{$\Gamma \Rightarrow \Delta,A,A$}
            \pravilo{C}
            \UnaryInfC{$\Gamma \Rightarrow \Delta,A$}
        \end{bprooftree}
    \end{center}
    Če torej znamo dokazati $\Delta$ iz dveh ponovitev formule $A$, znamo isto dokazati iz le ene kopije. Prav tako nam je vseeno koliko kopij formule imamo na desni, saj je potrebno dokazati le eno izmed njih.
\end{definicija}

\begin{opomba}
    Kot smo omenili, predstavljata $\Gamma$ in $\Delta$ multimnožici formul. Če bili to raje množici, pravila skrčitve ne bi potrebovali, saj bi sledilo že iz same strukture sekventa. Vendar nam bo število predpostavk in sklepov v nadaljnih poglavjih pomembno, zato $\Gamma$ in $\Delta$ ostaneta multimnožici.
\end{opomba}

Preden preidemo specifično na linearno logiko je potrebno še omeniti, da pri sekventnem računu dokazujemo ,,od spodaj navzgor''. Začnemo torej s sekventom, ki bi ga želeli dokazati in poiščemo katera pravila, strukturna ali logična, so nam na voljo. Analogija pri dokazovanju v vsakdanji matematiki je, da začnemo s problemom, ki ga želimo dokazati, in ga razčlenimo na manjše podprobleme, dokler ne dobimo nečesa, kar lahko zagotovo dokažemo. Podobno pri sekventnem računu sekvente postopoma prevajamo na aksiome.

Tudi pravila si zato lahko interpretiramo drugače. Pravilo L$\land$ iz definicije \ref{inl} lahko sedaj razumemo kot; če želimo iz $A \land B$ dokazati neke sklepe $\Delta$ je dovolj da $\Delta$ dokažemo že iz ene izmed formul $A$ ter $B$. Pravilo R$\land$ iz definicije \ref{inr} pa sedaj pravi; če želimo iz predpostavk $\Gamma$ dokazati $A \land B$, je dovolj da iz $\Gamma$ dokažemo $A$ ter da iz $\Gamma$ dokažemo $B$.

Za enostaven primer dokaza v sekventnem računu si oglejmo malodane trivialen dokaz komutativnosti veznika $\land$:
\begin{prooftree}
    \AxiomC{}
    \pravilo{Ax}
    \UnaryInfC{$B \Rightarrow B$}
    \pravilo{L$\land$}
    \UnaryInfC{$A \land B \Rightarrow B$}

    \AxiomC{}
    \pravilo{Ax}
    \UnaryInfC{$A \Rightarrow A$}
    \pravilo{L$\land$}
    \UnaryInfC{$A \land B \Rightarrow A$}

    \pravilo{R$\land$}
    \BinaryInfC{$A \land B \Rightarrow B \land A$}
\end{prooftree}

V besedah lahko ta dokaz razumemo sledeče. Želimo izpeljati, da $A \land B$ dokaže $B \land A$. Dovolj je, da pokažemo, da $A \land B$ dokaže $A$ ter da $A \land B$ dokaže $B$, po desnem pravilu za $\land$. Na levi je potem dovolj pokazati, da že $B$ dokaže $B$, po levem pravilu za $\land$, kar pa je ravno aksiom, torej velja. Na desni se zgodi podobno.

Od tu dalje bomo vsa pravila in dokaze brali od spodaj navzgor.
