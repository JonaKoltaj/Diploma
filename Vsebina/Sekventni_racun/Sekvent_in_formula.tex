Sekvent sestoji iz \emph{logičnih formul}, ki jih je torej potrebno definirati preden lahko definiramo sekvent. Definicija je induktivna, kar pomeni, da se formule gradijo iz podformul, te iz svojih podformul in tako dalje, na dnu pa so t.~i.\ osnovne formule. Te pa so zgrajene iz termov, ki so prav tako induktivno definirani.
\begin{definicija}
    \emph{Term} je izraz, ki je lahko oblike:
    \begin{itemize}
        \item spremenljivka,
        \item konstanta,
        \item $f(t_1,\ldots,t_n)$, kjer je $t_i$ term za vsak $i\in [n]$ in je $f$ nek funkcijski simbol, ki sprejme $n\in \mathbb{N}$ termov.
    \end{itemize}
\end{definicija}
\begin{primer*}
    Na voljo imamo denimo naravna števila, na katerih je definiran funkcijski simbol $+$, ki sprejme dva terma. Možni termi, ki jih lahko tvorimo, so torej lahko npr.\ spremenljivka $x$, konstanta $3$ ali pa izraz $x+3$.
\end{primer*}

\begin{definicija} \label{formula}
	\emph{Formula} je izraz oblike:
	\begin{itemize}
        \item \emph{osnovna formula}; $R(t_1, ..., t_n)$, kjer je $t_i$ term za vsak $i\in [n]$ in je $R$ nek \emph{relacijski simbol}, ki sprejme $n\in\mathbb{N}$ termov,
        \item \emph{sestavljena formula}, ki je -- kot pove ime -- sestavljena iz ene ali več podformul, med seboj povezanih z veznikom ali kvantifikatorjem.
	\end{itemize}
\end{definicija}
\begin{primer*}
    Imamo denimo terma $t_1$ in $t_2$ in relacijski simbol $\geq$, ki sprejme dva terma. Tvorimo torej lahko osnovno formulo $t_1\geq t_2$.
\end{primer*}
\begin{primer*}
    Če imamo formuli $A$ in $B$, so npr.\ $A \land B, \neg A, A \lor B$ ali $\forall x A$, kjer je $x$ neka prosta spremenljivka v $A$, tudi formule.
\end{primer*}
Katere sestavljene formule lahko tvorimo je odvisno od tega, s kakšnimi vezniki želimo delati. Če na primer nimamo veznika $\land$, $A \land B$ ne more biti formula. Specifične veznike, ki jih bomo uporabljali pri linearni logiki, bomo natančneje definirali v poglavju \ref{ll}.

\begin{definicija}
    Naj bodo $A_0,\ldots,A_n$ ter $B_0,\ldots,B_m$ neke logične formule. \emph{Sekvent} je izraz oblike $A_0,\ldots,A_n \Rightarrow B_0,\ldots,B_m$.
\end{definicija}

Formulam na levi strani sekventa navadno pravimo \emph{predpostavke}, formulam na desni pa \emph{sklepi}, obojemu pa lahko rečemo tudi \emph{kontekst}. Predpostavke navadno označujemo z $\Gamma$, sklepe pa z $\Delta$, kjer tako $\Gamma$ kot $\Delta$ predstavljata \emph{multimnožici} logičnih formul.

\begin{definicija}
    \emph{Multimnožica} je funkcija $f:A\to\mathbb{N}$, ki vsakemu elementu iz končne množice $A$ priredi naravno število. Drugače povedano je to množica, kjer se elementi lahko ponavljajo.
\end{definicija}
Pomembno je omeniti še, da simbol $\Rightarrow$ v sekventu ne predstavlja običajne implikacije in ga raje beremo kot ,,dokaže''. Vejice na levi strani sekventa se bere kot ,,in'', vejice na desni strani pa kot ,,ali''. Sekvent $A,B \Rightarrow C,D$ bi se torej razumel kot ,,predpostavki $A$ in $B$ dokažeta sklep $C$ ali sklep $D$''.
