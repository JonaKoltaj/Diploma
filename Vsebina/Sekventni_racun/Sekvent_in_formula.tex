\begin{definicija}
    Naj bosta $A_0,A_1,\ldots,A_n$ ter $B_0,B_1,\ldots,B_n$ končni zaporedji logičnih formul. \emph{Sekvent} je izraz oblike $A_0,\dots,A_n \Rightarrow B_0,\dots,B_m$.
\end{definicija}

Formulam na levi strani sekventa navadno pravimo \emph{predpostavke}, formulam na desni pa \emph{sklepi}. Celotno zaporedje predpostavk bomo označevali z $\Gamma$, zaporedje sklepov pa z $\Delta$.

Pomembno je omeniti še, da simbol $\Rightarrow$ v sekventu ne predstavlja običajne implikacije in ga raje beremo kot ,,dokaže''. Vejice na levi strani sekventa se bere kot ,,in'', vejice na desni strani pa kot ,,ali''. Sekvent $A,B \Rightarrow C,D$ bi se torej razumel kot ,,formuli $A$ in $B$ dokažeta formulo $C$ ali formulo $D$''.

V zgornji definiciji smo uporabili besedo \emph{formula}, ki jo je potrebno bolj formalno definirati. Definicija je induktivna, kar pomeni, da se formule gradijo iz podformul, te iz svojih podformul in tako dalje, na dnu pa so t.~i.\ osnovne formule. Te pa so zgrajene iz termov, ki so prav tako induktivno definirani.

%pr obeh tuki lahko citiram simpsonova predavanja
\begin{definicija}
    \emph{Term} je izraz, ki je lahko v treh različnih oblikah.
    \begin{enumerate}
        \item To je lahko neka spremenljivka.
        \item Lahko je konstanta.
        \item Lahko pa je rezultat funkcije, ki sprejme določeno število termov.
    \end{enumerate}
\end{definicija}
\begin{primer*}
    Na voljo imamo denimo naravna števila, na katerih je definirana funkcija $+$, ki sprejme dva terma. Možni termi, ki jih lahko tvorimo, so torej lahko npr.\ spremenljivka $x$, konstanta $3$ ali pa izraz $x+3$.
\end{primer*}

\begin{definicija} \label{formula}
	\emph{Formula} je izraz v dveh oblikah.
	\begin{enumerate}
        \item \emph{Osnovna formula} je predikatni simbol ali relacija, ki sprejme določeno število termov.
        \item \emph{Sestavljena formula} je -- kot pove ime -- sestavljena iz ene ali več podformul, med seboj povezanih z veznikom.
	\end{enumerate}
\end{definicija}
\begin{primer*}
    Imamo denimo terma $t_1$ in $t_2$ in relacijo $=$, ki sprejme dva terma. Tvorimo torej lahko osnovno formulo $t_1=t_2$.
\end{primer*}
\begin{primer*}
    Če imamo dve formuli $A$ in $B$, so npr.\ $A \land B$, $\neg A$, $A \lor B$ tudi formule.
\end{primer*}
Katere sestavljene formule lahko tvorimo je odvisno od tega, s kakšnimi vezniki želimo delati. Če naša logika na primer ne uporablja veznika $\land$, formula $A \land B$ ne pomeni ničesar. Specifične veznike, ki jih bomo uporabljali pri linearni logiki, bomo natančneje definirali v poglavju \ref{ll}.
