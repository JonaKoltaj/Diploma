Da bi lahko določene stvari dokazali, si želimo v linearni logiki modelirati tudi običajen sekventni račun, vključno torej z ošibitvijo ter skrčitvijo. A to nelinearnost želimo omejiti na specifične formule. Namen teh ,,nelinearnih'' formul je, da nam dovolijo v linearni logiki dokazati vse, kar je možno dokazati tudi v običajnem sekventnem računu, a da je iz dokaza takoj razvidno, kateri deli sekventa so bili dokazani linearno in kateri ne. Ker želimo označiti formulo kot nelinearno, jo modificiramo v novo formulo z veznikom. Imamo dva takšna veznika, ki ju imenujemo \emph{eksponenta}.

\begin{definicija}
    Veznik \emph{seveda} je označen s simbolom ! in označuje nelinearnost na levi strani sekventa, veznik \emph{zakaj ne} pa je označen s simbolom ? in označuje nelinearnost na desni strani:
    \begin{center}
        \begin{bprooftree}
            \AxiomC{$\Gamma,A \Rightarrow \Delta$}
            \pravilo{L!}
            \UnaryInfC{$\Gamma,!A \Rightarrow \Delta$}
        \end{bprooftree}
        \begin{bprooftree}
            \AxiomC{$!\Gamma \Rightarrow A,?\Delta$}
            \pravilo{R!}
            \UnaryInfC{$!\Gamma \Rightarrow \ !A,?\Delta$}
        \end{bprooftree}
        \begin{bprooftree}
            \AxiomC{$\Gamma \Rightarrow \Delta$}
            \pravilo{W!}
            \UnaryInfC{$\Gamma,!A \Rightarrow \Delta$}
        \end{bprooftree}
        \begin{bprooftree}
            \AxiomC{$\Gamma,!A,!A \Rightarrow \Delta$}
            \pravilo{C!}
            \UnaryInfC{$\Gamma,!A \Rightarrow \Delta$}
        \end{bprooftree}
    \end{center}
    \begin{center}
        \begin{bprooftree}
            \AxiomC{$!\Gamma,A \Rightarrow ?\Delta$}
            \pravilo{L?}
            \UnaryInfC{$!\Gamma,?A \Rightarrow ?\Delta$}
        \end{bprooftree}
        \begin{bprooftree}
            \AxiomC{$\Gamma \Rightarrow A,\Delta$}
            \pravilo{R?}
            \UnaryInfC{$\Gamma \Rightarrow \ !A,\Delta$}
        \end{bprooftree}
        \begin{bprooftree}
            \AxiomC{$\Gamma \Rightarrow A,\Delta$}
            \pravilo{W?}
            \UnaryInfC{$\Gamma \Rightarrow \ ?A,\Delta$}
        \end{bprooftree}
        \begin{bprooftree}
            \AxiomC{$\Gamma \Rightarrow \ ?A,?A,\Delta$}
            \pravilo{C?}
            \UnaryInfC{$\Gamma \Rightarrow \ ?A,\Delta$}
        \end{bprooftree}
    \end{center}
    Za oba veznika veljajo štiri pravila. Oba lahko vpeljemo na levi ali desni strani, lahko pa na ustrezni strani uporabimo tudi skrčitev in ošibitev. Zapis $!\Gamma$ (ali $?\Delta$) označuje, da je vsaka formula v $\Gamma$ (ali $\Delta$) predznačena z veznikom ! (ali ?).
\end{definicija}

Z eksponenti označimo, da imamo na voljo "poljubno mnogo" kopij predpostavke ali sklepa. Ker vejice na levi beremo kot ,,in'', vejice na desni pa kot ,,ali'', je za to treba podati dva različna veznika. Formula $!A$ torej označuje ,,$A$ in $A$ in $\ldots$ $A$'', kolikor kopij pač potrebujemo, formula $?A$ pa označuje ,,$A$ ali $A$ ali $\ldots$ $A$''.

Interpretacija L! je dokaj enostavna. Pove le, da se kadarkoli v procesu dokazovanja lahko odločimo formulo označiti kot nelinearno. Ker pa se med sklepi ! ne pojavi naravno, morajo za vpeljavo na desni veljati strožji pogoji, namreč, da je že celoten sekvent nelinearen. Interpretacija L? ter R? je simetrična.
