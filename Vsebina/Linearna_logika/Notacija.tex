Na koncu tega poglavja je potrebnih še nekaj opomb glede zapisa veznikov ter strukture sekventov.

\subsubsection{Poimenovanje veznikov}

Zapis, uporabljen v tem diplomskem delu, je črpan iz vira ??daj vir??, ni pa najbolj standarden. Jean-Yves Girard, ki se je prvi ukvarjal z linearno logiko je veznike in konstante označil drugače, ta zapis pa se je tudi ohranil. Razlog za spremembo zapisa v mojem delu je bolj slogoven kot zgodovinski. Načini kako je zapis spremenjen ter standardna imena veznikov v angleščini so prikazana v spodnji tabeli. ??Daj also vir?? %lahko citiras girarda
\begin{center}
\begin{tabular}{||c|c|c||}
    \hline
    Simbol veznika & Simbol v standardnem zapisu & Ime \\
    \hline\hline
    $\sqcap$ & $\&$ & with \\
    \hline
    $\star$ & $\otimes$ & tensor \\
    \hline
    $\sqcup$ & $\oplus$ & plus \\
    \hline
    + & $\parr$ & par \\
    \hline
    $\top$ & $\top$ & top \\
    \hline
    $\enota$ & $\enota$ & one \\
    \hline
    $\bot$ & $\nicla$ & zero \\
    \hline
    $\nicla$ & $\bot$ & bottom \\
    \hline
    \end{tabular}
\end{center}
Kot lahko vidimo so v standardnem zapisu parni drugačni vezniki kot v našem. Razlog za to je, da veznik + distribuira čez veznik $\sqcap$, veznik $\star$ pa distribuira čez $\sqcup$. Velja torej ??cite troelstra str 21 22??:
\begin{align*}
    A + (B \sqcap C) &\equiv (A + B) \sqcap (A + C)\\
    A \star (B \sqcup C) &\equiv (A \star B) \sqcup (A \star C)
\end{align*}
Toda če veznik + negiramo, ne dobimo veznika $\sqcap$, ampak $\star$, če negiramo veznik $\sqcap$ pa dobimo $\sqcup$ in seveda obratno. Dualna para sta torej $(\star,+)$ ter $(\sqcap,\sqcup)$, kar je veliko bolj razvidno pri naših oznakah. Poleg tega sta si že sami pravili za veznika $\star$ ter +, iz definicij \ref{tenzor} in \ref{plus}, simetrični, kot sta si pravili za $\sqcap$ ter $\sqcup$, iz definicij \ref{in} in \ref{ali}. Zdi se mi, da sta to dovolj pomembna razloga za razumevanje veznikov, da sem se odločila za nestandarden zapis.

\subsubsection{Intuicionistična linearna logika} \label{ill}

Članki o linearni logiki pogosto omenajo tudi intuicionistično linearno logiko, zato se mi zdi pomembno slednje vsaj predstaviti tudi v mojem delu.

Intuicionistična logika je logika brez principa izključene tretje možnosti. To je aksiom, ki pravi, da za poljubno trditev $P$ velja $P\lor\neg P$. V sekventnem računu kot smo ga predstavili do sedaj, se da to pravilo izpeljati iz pravila aksioma. V linearni logiki ta princip velja le za veznik +, ne pa za veznik $\sqcup$.??citiraj??
\begin{prooftree}
	\AxiomC{}
	\pravilo{Ax}
	\UnaryInfC{$A \Rightarrow A$}
	\pravilo{R$\negacija$}
	\UnaryInfC{$\Rightarrow A,\negacija A$}
	\pravilo{R+}
	\UnaryInfC{$\Rightarrow A + (\negacija A)$}
\end{prooftree}
V običajnem sekventnem računu je pravilo za negacijo enako, linearnost je ne spremeni, veznika + in $\sqcap$ pa sta si ekvivalnetna in sta oba reprezentaciji veznika $\lor$. To pomeni, da izpeljava principa izključene tretje možnosti v nelinearnem poteka enako kot zgoraj.

Če torej želimo delati z intuicionistično linearno logiko, moramo strukturo sekventnega računa nekoliko spremeniti. Dovolj je??cite troelstra??, da na desni strani sekventov ne dopustimo več kot ene formule. Kot lahko hitro vidimo, zgornja izpeljava ne deluje več, saj sekvent $\Rightarrow \negacija A,A$ ne more obstajati.

Intuicionistična logika daje poudarek na tem \emph{kako} dokazujemo izreke in trditve, ne le da jih dokažemo. A ker je linearna logika že zelo strukturirana, poleg tega pa bi dopuščanje le ene formule na desni strani sekventa uničilo simetrijo pravil, saj na primer ne bi smeli imeti desnega pravila za +, sem se odločila v tem delu uporabljati klasično -- torej neintuicionistično -- logiko.
