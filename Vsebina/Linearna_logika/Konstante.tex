V običajnem sekventnem računu imamo dvoje konstant; resnico in neresnico, ki pa se v linearni logiki spet razdelita. Resnici se delita na enoto za $\star$ ter enoto za $\sqcap$, neresnici pa na enoto za + ter enoto za $\sqcup$.

\begin{definicija}
    \emph{Enota} $\enota$, ima levo in desno pravilo:
    \begin{center}
        \begin{bprooftree}
            \AxiomC{$\Gamma \Rightarrow \Delta$}
            \pravilo{L$\enota$}
            \UnaryInfC{$\Gamma,\enota \Rightarrow \Delta$}
        \end{bprooftree}
        \begin{bprooftree}
            \AxiomC{}
            \pravilo{R$\enota$}
            \UnaryInfC{$\Rightarrow \enota$}
        \end{bprooftree}
    \end{center}
    Lahko jo brez predpostavk dokažemo, kar nam pove desno pravilo, levo pravilo pa pove, da če pa vemo da enota velja je to trivialna informacija.
\end{definicija}

\begin{definicija}
    \emph{Resnica} $\top$, ima le desno pravilo:
    \begin{prooftree}
        \AxiomC{}
        \pravilo{R$\top$}
        \UnaryInfC{$\Gamma \Rightarrow \top,\Delta$}
    \end{prooftree}
    Ker levega pravila nima, je med predpostavkami ne moremo uporabiti. Kar pa nam desno pravilo pove, je da resnica vedno velja.
\end{definicija}

\begin{lema} \label{enoti}
	Konstanta $\enota$ je enota za za veznik $\star$, konstanta $\top$ pa je enota za veznik $\sqcap$.
\end{lema}
\begin{dokaz}
    Za dokaz leme nam je potrebno izpeljati sekvente $A \star \enota \Rightarrow A$, $A \Rightarrow A \star \enota$, $A \sqcap \top \Rightarrow A$ ter $A \Rightarrow A \sqcap \top$:
    \begin{center}
        \vskip 10pt
        \begin{bprooftree}
            \AxiomC{}
            \pravilo{Ax}
            \UnaryInfC{$A \Rightarrow A$}
            \pravilo{L$\enota$}
            \UnaryInfC{$A,\enota \Rightarrow A$}
            \pravilo{L$\star$}
            \UnaryInfC{$A \star \enota \Rightarrow A$}
        \end{bprooftree}
        \begin{bprooftree}
            \AxiomC{}
            \levopravilo{Ax}
            \UnaryInfC{$A \Rightarrow A$}

            \AxiomC{}
            \pravilo{R$\enota$}
            \UnaryInfC{$\Rightarrow \enota$}

            \pravilo{R$\star$}
            \BinaryInfC{$A \Rightarrow A \star \enota$}
        \end{bprooftree}
    \end{center}
    \vskip 10pt
    \begin{center}
        \begin{bprooftree}
            \AxiomC{}
            \pravilo{Ax}
            \UnaryInfC{$A \Rightarrow A$}
            \pravilo{L$\sqcap$}
            \UnaryInfC{$A \sqcap \top \Rightarrow A$}
        \end{bprooftree}
        \begin{bprooftree}
            \AxiomC{}
            \levopravilo{Ax}
            \UnaryInfC{$A \Rightarrow A$}

            \AxiomC{}
            \pravilo{R$\top$}
            \UnaryInfC{$A \Rightarrow \top$}

            \pravilo{R$\sqcap$}
            \BinaryInfC{$A \Rightarrow A \sqcap \top$}
        \end{bprooftree}
    \end{center}
    \vskip -15pt \qedhere
\end{dokaz}

\begin{definicija}
	\emph{Ničla} $\nicla$ ima levo in desno pravilo:
	 \begin{center}
        \begin{bprooftree}
            \AxiomC{}
            \pravilo{L$\nicla$}
            \UnaryInfC{$\nicla \Rightarrow$}
        \end{bprooftree}
        \begin{bprooftree}
            \AxiomC{$\Gamma \Rightarrow \Delta$}
            \pravilo{R$\nicla$}
            \UnaryInfC{$\Gamma \Rightarrow \nicla,\Delta$}
        \end{bprooftree}
    \end{center}
\end{definicija}
\begin{definicija}
    \emph{Neresnica} $\bot$ ima le levo pravilo, kar pomeni, da ne more biti uporabljena kot sklep:
    \begin{prooftree}
        \AxiomC{}
        \pravilo{L$\bot$}
        \UnaryInfC{$\Gamma, \bot \Rightarrow \Delta$}
    \end{prooftree}
\end{definicija}

Kot lahko vidimo, so zgornja pravila za ničlo in neresnico simetrična pravilom za enoto in resnico, prav tako pa je simetričen dokaz naslednje leme dokazu leme \ref{enoti}, zato ga bomo opustili.

\begin{lema}
	Konstanta $\nicla$ je enota za za veznik +, konstanta $\bot$ pa je enota za veznik $\sqcup$.
\end{lema}
