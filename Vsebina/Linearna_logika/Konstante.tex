V običajnem sekventnem računu imamo dvoje konstant; resnico in neresnico, ki pa se v linearni logiki spet vsaka razdelita na dvoje. Resnici se delita na enoto za $\star$ ter enoto za $\sqcap$, neresnici pa na enoto za + ter enoto za $\sqcup$.

\begin{definicija}
    \emph{Enota}, označena z $\enota$, ima levo in desno pravilo vpeljave:
    \begin{center}
        \begin{bprooftree}
            \AxiomC{$\Gamma \Rightarrow \Delta$}
            \pravilo{L$\enota$}
            \UnaryInfC{$\Gamma,\enota \Rightarrow \Delta$}
        \end{bprooftree}
        \begin{bprooftree}
            \AxiomC{}
            \pravilo{R$\enota$}
            \UnaryInfC{$\Rightarrow \enota$}
        \end{bprooftree}
    \end{center}
    Enoto torej lahko brez predpostavk vedno dokažemo, kar nam pove desno pravilo, če pa vemo da enota velja je to trivialna informacija, kar nam pove levo pravilo.
\end{definicija}

\begin{definicija}
    \emph{Resnica}, označena z $\top$, ima le desno pravilo vpeljave. Ne moremo je torej uporabiti kot sklep.
    \begin{prooftree}
        \AxiomC{}
        \pravilo{R$\top$}
        \UnaryInfC{$\Gamma \Rightarrow \top,\Delta$}
    \end{prooftree}
    Kar nam to pove je, da resnica vedno velja.
\end{definicija}

\begin{lema} \label{enoti}
	Enota $\enota$ je enota za za $\star$, resnica $\top$ pa je enota za $\sqcap$.
\end{lema}
\begin{dokaz}
    Za dokaz leme potrebujemo izpeljati sekvente $A \star \enota \Rightarrow A$, $A \Rightarrow A \star \enota$, $A \sqcap \top \Rightarrow A$ ter $A \Rightarrow A \sqcap \top$.
    \begin{center}
        \vskip 10pt
        \begin{bprooftree}
            \AxiomC{}
            \pravilo{Ax}
            \UnaryInfC{$A \Rightarrow A$}
            \pravilo{L$\enota$}
            \UnaryInfC{$A,\enota \Rightarrow A$}
            \pravilo{L$\star$}
            \UnaryInfC{$A \star \enota \Rightarrow A$}
        \end{bprooftree}
        \begin{bprooftree}
            \AxiomC{}
            \levopravilo{Ax}
            \UnaryInfC{$A \Rightarrow A$}

            \AxiomC{}
            \pravilo{R$\enota$}
            \UnaryInfC{$\Rightarrow \enota$}

            \pravilo{R$\star$}
            \BinaryInfC{$A \Rightarrow A \star \enota$}
        \end{bprooftree}
    \end{center}
    \vskip 10pt
    \begin{center}
        \begin{bprooftree}
            \AxiomC{}
            \pravilo{Ax}
            \UnaryInfC{$A \Rightarrow A$}
            \pravilo{L$\sqcap$}
            \UnaryInfC{$A \sqcap \top \Rightarrow A$}
        \end{bprooftree}
        \begin{bprooftree}
            \AxiomC{}
            \levopravilo{Ax}
            \UnaryInfC{$A \Rightarrow A$}

            \AxiomC{}
            \pravilo{R$\top$}
            \UnaryInfC{$A \Rightarrow \top$}

            \pravilo{R$\sqcap$}
            \BinaryInfC{$A \Rightarrow A \sqcap \top$}
        \end{bprooftree}
    \end{center}
\end{dokaz}

Pri neresnici je razlog za razdvojitev enak, pravila vpeljave pa so simetrična, zato interpretacijo prepustimo bralcu.

\begin{definicija}
	\emph{Ničla}, označena z $\nicla$ ima levo in desno pravilo vpeljave:
	 \begin{center}
        \begin{bprooftree}
            \AxiomC{}
            \pravilo{L$\nicla$}
            \UnaryInfC{$\nicla \Rightarrow$}
        \end{bprooftree}
        \begin{bprooftree}
            \AxiomC{$\Gamma \Rightarrow \Delta$}
            \pravilo{R$\nicla$}
            \UnaryInfC{$\Gamma \Rightarrow \nicla,\Delta$}
        \end{bprooftree}
    \end{center}
\end{definicija}
\begin{definicija}
    \emph{Neresnica}, označena z $\bot$, ima le levo pravilo vpeljave. Ne moremo je torej uporabiti kot predpostavko.
    \begin{prooftree}
        \AxiomC{}
        \pravilo{L$\bot$}
        \UnaryInfC{$\Gamma, \bot \Rightarrow \Delta$}
    \end{prooftree}
\end{definicija}

Dokaz naslednje leme bomo opustili, saj je simetričen dokazu leme \ref{enoti}.
\begin{lema}
	Ničla $\nicla$ je enota za +, neresnica $\bot$ pa je enota za $\sqcup$.
\end{lema}
