\begin{definicija}
	\emph{Univerzalni kvantifikator} $\forall$ je definiran z naslednjima praviloma:
	%notacija tuki bo t/x means substitute vsak x s tjem
	\begin{center}
        \begin{bprooftree}
            \AxiomC{$\Gamma, A[t/x] \Rightarrow \Delta$}
            \pravilo{L$\forall$}
            \UnaryInfC{$\Gamma,\forall x A \Rightarrow \Delta$}
        \end{bprooftree}
        \begin{bprooftree}
            \AxiomC{$\Gamma \Rightarrow A[y/x],\Delta$}
            \pravilo{R$\forall$}
            \UnaryInfC{$\Gamma \Rightarrow \forall x A,\Delta$}
        \end{bprooftree} \qquad
        ; $y$ ni prost v $\Gamma$ in $\Delta$
	\end{center}
	Zapis $A[a/x]$ pomeni, da vsako prosto pojavitev spremenljivke $x$ zamenjamo s termom $a$. V L$\forall$ s $t$ označujemo specifičen term $t$, v R$\forall$ pa z $y$ označujemo poljubno spremenljivko. Levo pravilo nam torej pravi, da če želimo iz dejstva, da za vsak $x$ velja formula $A$, dokazati $\Delta$, je dovolj, da spremenljivko $x$ v $A$ nadomestimo z nekim specifičnim termom in s tem dokažemo $\Delta$. Desno pravilo pa je ekvivalentno temu, da pri dokazovanju tega, da za vsak $x$ velja $A$, fiksiramo poljuben $y$ in dokazujemo $A$.
\end{definicija}

\begin{definicija}
    \emph{Eksistenčni kvantifikator} $\exists$ je simetričen univerzalnemu kvantifikatorju:
    \begin{center}
        \begin{bprooftree}
            \AxiomC{$\Gamma,A[y/x] \Rightarrow \Delta$}
            \pravilo{L$\exists$}
            \UnaryInfC{$\Gamma,\exists x A \Rightarrow \Delta$}
        \end{bprooftree}
        \begin{bprooftree}
            \AxiomC{$\Gamma \Rightarrow A[t/x],\Delta$}
            \pravilo{R$\exists$}
            \UnaryInfC{$\Gamma \Rightarrow \exists x A,\Delta$}
        \end{bprooftree} \qquad
        ; $y$ ni prost v $\Gamma$ in $\Delta$
	\end{center}
	Tokrat levo pravilo vsebuje prosto spremenjlivko $y$, desno pa specifičen term $t$. Če torej želimo uporabiti dejstvo, da obstaja $x$, da velja $A$, je dovolj da $\Delta$ dokažemo pri poljubnem $y$ na mestu spremenljivke $x$, če pa želimo dokazati, da obstaja $x$, da velja $A$, le poščemo nek specifičen term $t$, da $A$ velja.
\end{definicija}
