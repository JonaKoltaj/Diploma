\begin{definicija}
	\emph{Univerzalni kvantifikator}, označen kot navadno s simbolom $\forall$, je definiran z naslednjima praviloma vpeljave. Tu $y$ ne sme biti prost v $\Gamma$ in $\Delta$.
	%notacija tuki bo t/x means substitute vsak x s tjem
	%also citiras simpsona as well
	\begin{center}
        \begin{bprooftree}
            \AxiomC{$\Gamma, A[t/x] \Rightarrow \Delta$}
            \pravilo{L$\forall$}
            \UnaryInfC{$\Gamma,\forall x A \Rightarrow \Delta$}
        \end{bprooftree}
        \begin{bprooftree}
            \AxiomC{$\Gamma \Rightarrow A[y/x],\Delta$}
            \pravilo{R$\forall$}
            \UnaryInfC{$\Gamma \Rightarrow \forall x A,\Delta$}
        \end{bprooftree}
	\end{center}
	Notacija $A[y/x]$ pomeni, da vsako instanco spremenljivke $x$ zamenjamo s spremenljivko $y$. Spremenljivka $t$ v definiciji označuje nek specifičen term $t$, ki si ga izberemo. Levo pravilo vpeljave torej pomeni, da če želimo iz dejstva, da za vsak $x$ velja formula $A$ dokazati $\Delta$, je dovolj, da spremenljivko $x$ v $A$ zamenjamo z nekim specifičnim termom in z njim dokažemo $\Delta$. Spremenljivka $y$ v definiciji pa označuje prosto spremenljivko. Desno pravilo vpeljave je ekvivalentno temu, da pri dokazovanju, da za vsak $x$ velja $A$, fiksiramo poljuben $y$ in dokazujemo $A$.
\end{definicija}

\begin{definicija}
    \emph{Eksistenčni kvantifikator} je spet brez sprememb označen s simbolom $\exists$. Spremenljivka $y$ spet ne sme  biti prosta v $\Gamma$ ter $\Delta$.
    \begin{center}
        \begin{bprooftree}
            \AxiomC{$\Gamma,A[y/x] \Rightarrow \Delta$}
            \pravilo{L$\exists$}
            \UnaryInfC{$\Gamma,\exists x A \Rightarrow \Delta$}
        \end{bprooftree}
        \begin{bprooftree}
            \AxiomC{$\Gamma \Rightarrow A[t/x],\Delta$}
            \pravilo{R$\exists$}
            \UnaryInfC{$\Gamma \Rightarrow \exists x A,\Delta$}
        \end{bprooftree}
	\end{center}
	Tokrat levo pravilo vsebuje prosto spremenjlivko $y$, desno pa specifičen term $t$. Če torej želimo uporabiti dejstvo, da obstaja $x$, da velja $A$, fiksiramo poljuben $y$ in dokazujemo $A$, če pa želimo dokazati, da obstaja $x$, da velja $A$, le poščemo nek specifičen term $t$, da $A$ velja.
\end{definicija}
