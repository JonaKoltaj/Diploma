Obravnavali bomo le eliminacijo pravila Rez!$_n$, saj je eliminacija Rez?$_n$ simetrična. Zopet bomo primere ločili glede na zadnje pravilo pred rezom, ki ne vpelje rezane formule. Za razliko od prejšnjega podpoglavja pa je tu pomembno, na kateri strani nad rezom se to pravilo zgodi, saj rez ni več simetričen. Denimo, da formula ni vpeljana na desni, torej:
\begin{prooftree}
    \derivation{0}{$\Gamma,(!A)^n \Rightarrow \Delta$}
    \AxiomC{$\D_1$}
    \pravilo{Pravilo, ki vpelje $C$}
    \UnaryInfC{$\Gamma' \Rightarrow \ !A,C,\Delta'$}
    \pravilo{Rez!$_n$}
    \BinaryInfC{$\Gamma,\Gamma' \Rightarrow C,\Delta,\Delta'$}
\end{prooftree}
V tem primeru pravzaprav ni važno, ali je bil obravnavani rez posplošen ali ne. Pri vseh pravilih iz prejšnjega poglavja, razen R! in L?, je bilo drugo poddrevo za korak indukcije nepomembno. Pravili R! in L? pa zopet ne moreta vpeljati formule C, saj sklepi $!A,\Delta$, zaradi formule $!A$ ne morejo biti predznačeni z veznikom ?.

Korak indukcije, ko formula $!A$ na desni \emph{ni} bila vpeljana, smo torej obdelali že v prejšnjem podpoglavju, zato nam ostane le še naslednji primer:
\begin{prooftree}
    \derivation{0}{$\Gamma,(!A)^n \Rightarrow \Delta$}
    \derivation{1}{$!\Gamma' \Rightarrow A,?\Delta'$}
    \pravilo{R!}
    \UnaryInfC{$!\Gamma' \Rightarrow \ !A,?\Delta'$}
    \pravilo{Rez!$_n$}
    \BinaryInfC{$\Gamma,\Gamma' \Rightarrow \Delta,\Delta'$}
\end{prooftree}
Sedaj moramo primere ločiti glede na zadnje pravilo nad sekventom $\Gamma,(!A)^n \Rightarrow \Delta$. Ti primeri so si med seboj spet zelo podobni, zato bomo obravnavali le pravili R$\sqcap$ ter R$\star$, začenši z desnim pravilom za veznik $\sqcap$:
\begin{prooftree}
    \derivation{0}{$\Gamma,(!A)^n \Rightarrow B,\Delta$}
    \derivation{1}{$\Gamma,(!A)^n \Rightarrow C,\Delta$}
    \levopravilo{R$\sqcap$}
    \BinaryInfC{$\Gamma,(!A)^n \Rightarrow B \sqcap C,\Delta$}

    \derivation{2}{$!\Gamma' \Rightarrow A,?\Delta'$}
    \pravilo{R!}
    \UnaryInfC{$!\Gamma' \Rightarrow \ !A,?\Delta'$}
    \pravilo{Rez!$_n$}
    \BinaryInfC{$\Gamma,!\Gamma' \Rightarrow B \sqcap C,\Delta,?\Delta'$}
\end{prooftree}
\dol
\begin{prooftree} \footnotesize
    \derivation{0}{$\Gamma,(!A)^n \Rightarrow B,\Delta$}
    \derivation{2}{$!\Gamma' \Rightarrow A,?\Delta'$}
    \pravilo{R!}
    \UnaryInfC{$!\Gamma' \Rightarrow \ !A,?\Delta'$}
    \levopravilo{Rez!$_n$}
    \BinaryInfC{$\Gamma,!\Gamma' \Rightarrow B,\Delta,?\Delta'$}


    \derivation{1}{$\Gamma,(!A)^n \Rightarrow C,\Delta$}
    \derivation{2}{$!\Gamma' \Rightarrow A,?\Delta'$}
    \pravilo{R!}
    \UnaryInfC{$!\Gamma' \Rightarrow \ !A,?\Delta'$}
    \pravilo{Rez!$_n$}
    \BinaryInfC{$\Gamma,!\Gamma' \Rightarrow C,\Delta,?\Delta'$}

    \levopravilo{R$\sqcap$}
    \BinaryInfC{$\Gamma,!\Gamma' \Rightarrow B \sqcap C,\Delta,?\Delta'$}
\end{prooftree}
Pri R$\sqcap$ predpostavke ostanejo v isti obliki, kar pomeni, da je korak indukcije enak, kot če bi obravnavali le navadni rez. Dejstvo, da je bilo rezanih več formul naenkrat namreč ne pride v poštev. Enako se seveda zgodi pri vsakem pravilu, ki predpostavk ne razdeli na dvoje. Pri R$\star$ pa se zgodi naslednje:
\begin{prooftree}
    \derivation{0}{$\Gamma,(!A)^p \Rightarrow B,\Delta$}
    \derivation{1}{$\Gamma',(!A)^q \Rightarrow C,\Delta'$}
    \levopravilo{R$\star$}
    \BinaryInfC{$\Gamma,\Gamma',(!A)^{p+q} \Rightarrow B \star C,\Delta,\Delta'$}

    \derivation{2}{$!\Gamma'' \Rightarrow A,?\Delta''$}
    \pravilo{R!}
    \UnaryInfC{$!\Gamma' \Rightarrow \ !A,?\Delta'$}
    \pravilo{Rez!$_{p+q}$}
    \BinaryInfC{$\Gamma,\Gamma',!\Gamma'' \Rightarrow B \star C,\Delta,\Delta',?\Delta''$}
\end{prooftree}
\dol
\begin{prooftree} \footnotesize
    \derivation{0}{$\Gamma,(!A)^p \Rightarrow B,\Delta$}
    \derivation{2}{$!\Gamma'' \Rightarrow A,?\Delta''$}
    \pravilo{R!}
    \UnaryInfC{$!\Gamma' \Rightarrow \ !A,?\Delta'$}
    \levopravilo{Rez!$_p$}
    \BinaryInfC{$\Gamma,!\Gamma'' \Rightarrow B,\Delta,?\Delta''$}

    \derivation{1}{$\Gamma',(!A)^q \Rightarrow C,\Delta'$}
    \derivation{2}{$!\Gamma'' \Rightarrow A,?\Delta''$}
    \pravilo{R!}
    \UnaryInfC{$!\Gamma' \Rightarrow \ !A,?\Delta'$}
    \pravilo{Rez!$_q$}
    \BinaryInfC{$\Gamma',!\Gamma'' \Rightarrow C,\Delta',?\Delta''$}

    \pravilo{R$\star$}
    \BinaryInfC{$\Gamma,\Gamma',!\Gamma'',!\Gamma'' \Rightarrow B \star C,\Delta,\Delta',?\Delta'',?\Delta''$}
    \pravilo{C!$\times|\Gamma''|$}
    \UnaryInfC{$\Gamma,\Gamma',!\Gamma'' \Rightarrow B \star C,\Delta,\Delta',?\Delta'',?\Delta''$}
    \pravilo{C?$\times|\Delta''|$}
    \UnaryInfC{$\Gamma,\Gamma',!\Gamma'' \Rightarrow B \star C,\Delta,\Delta',?\Delta''$}
\end{prooftree}
Pomembno je, da so predpostavke v desnem poddrevesu začetnega drevesa oblike $!\Gamma''$, sklepi pa oblike $?\Delta''$, saj smo jih zaradi tega lahko uporabili dvakrat. Enako se zgodi pri L+ ter R$\multimap$, saj ti dve pravili predpostavke prav tako razdelita na dva dela.
