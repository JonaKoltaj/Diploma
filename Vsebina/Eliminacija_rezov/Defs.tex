Dokaz zgornjega izreka poteka s pomočjo indukcije na drevesih izpeljave nad rezom ter na kompleksnosti formul, ki jih režemo. Zavoljo tega moramo tako na drevesih kot formulah vpeljati nekakšno mero, glede na katero bomo potem izvalali proces indukcije. Formula je induktivno definirana v definiciji \ref{formula}, zato je tudi kompleksnost formule definirana induktivno.

\begin{definicija}
    Naj bosta $A$ in $B$ poljubni formuli, $P$ pa naj bo neka osnovna formula. \emph{Rang formule}, označen s simbolom $\R$, je definiran sledeče:
    \begin{itemize}
        \item $\R(P) = \R(\top) = \R(\bot) = \R(\enota) = \R(\nicla) = 1$
        \item $\R(\negacija A) = \R(\forall x A) = \R(\exists x A) = \R(A) + 1$
        \item $\R(A*B) = \max\{\R(A),\R(B)\} + 1$ ; kjer je $*$ poljuben veznik, ki sprejme dve formuli.
    \end{itemize}
\end{definicija}

Podobno je definirana višina drevesa. Imejmo naslednji dve drevesi, levo označeno z $\D$, desno pa z $\D'$:
\begin{center}
    \begin{bprooftree}
        \AxiomC{$\D_0$}
        \AxiomC{$\D_1$}
        \AxiomC{$\ldots$}
        \AxiomC{$\D_n$}
        \pravilo{Pravilo}
        \QuaternaryInfC{$\mathcal{S}$}
    \end{bprooftree}
    \begin{bprooftree}
        \AxiomC{}
        \pravilo{Pravilo'}
        \UnaryInfC{$\mathcal{S}$'}
    \end{bprooftree}
\end{center}
Tu je $n\in\mathbb{N}$, Pravilo je poljubno pravilo, ki sprejme $n$ sekventov, Pravilo' pa je poljubno pravilo, ki jih ne sprejme nič. Slednje je na primer pravilo aksioma ali nekatera izmed pravil za konstante:

\begin{definicija}
    \emph{Višina drevesa izpeljave} je označena s simbolom $\h$ in je enaka:
    \begin{itemize}
        \item$\h(\D) = \max\{\h(\D_0),\h(\D_1),\ldots,\h(\D_n)\} + 1$
        \item$\h(\D') = 1$
    \end{itemize}
\end{definicija}

Sedaj ko imamo mero tako na formulah kot na drevesih izpeljave, lahko defniramo skupno mero vsakega reza. Oglejmo si poljubno pravilo reza:
\begin{prooftree} \label{rez general}
    \AxiomC{$\D_0$}
    \noLine
    \UnaryInfC{$\Gamma,A \Rightarrow \Delta$}

    \AxiomC{$\D_1$}
    \noLine
    \UnaryInfC{$\Gamma' \Rightarrow A,\Delta'$}

    \pravilo{Rez}
    \BinaryInfC{$\Gamma,\Gamma' \Rightarrow \Delta,\Delta'$}
\end{prooftree}
Tu $\D_0$ in $\D_1$ označujeta drevesi izpeljav, ki sta vodili do sekventov pod njima.

\begin{definicija} \label{stopnja}
    \emph{Stopnja reza} je par števil $(\R,\h)$, kjer je $\R$ rang rezane formule, $\h$ pa višina samega drevesa. Z zgornjimi oznakami je stopnja reza torej enaka:
    $$
    (\R,\h) = (\R(A), \max\{\h(\D_0),\h(\D_1)\} + 2)
    $$
    Stopnja reza je urejena \emph{leksikografsko}, kar pomeni, da relacijo < na paru $(\R,\h)$ definiramo sledeče:
    $$
    (\R,\h) < (\R',\h') \Leftrightarrow (\R < \R') \text{ ali } ((\R = \R') \text{ in } (\h < \h'))
    $$
\end{definicija}
\begin{definicija} \label{najnizji}
    Rezu bomo rekli \emph{najnižji rez}, če se nad njim ne pojavi nobeno drugo pravilo reza. To seveda ne pomeni nujno, da ima najnižjo stopnjo ali da je v drevesu izpeljave le en takšen rez.
\end{definicija}

Preden začnemo z dokazom je nazadnje potrebno definirati še eno vrsto reza, in sicer glavni rez.

\begin{definicija} \label{gl rez}
    Če je rezana formula vpeljana v obeh poddrevesih nad pravilom reza, to imenujemo \emph{glavni rez}.
\end{definicija}

\begin{primer*} \label{gl rez in}
    Glavni rez za formulo $A \sqcap B$.
    \begin{prooftree}
        \AxiomC{$\Gamma,A \Rightarrow \Delta$}
        \levopravilo{L$\sqcap$}
        \UnaryInfC{$\Gamma,A \sqcap B \Rightarrow \Delta$}

        \AxiomC{$\Gamma' \Rightarrow A,\Delta'$}
        \AxiomC{$\Gamma' \Rightarrow B,\Delta'$}
        \pravilo{R$\sqcap$}
        \BinaryInfC{$\Gamma' \Rightarrow A \sqcap B,\Delta'$}

        \pravilo{Rez}
        \BinaryInfC{$\Gamma,\Gamma' \Rightarrow \Delta,\Delta'$}
    \end{prooftree}
\end{primer*}
