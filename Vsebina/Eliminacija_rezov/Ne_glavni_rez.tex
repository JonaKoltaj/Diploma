V tem podpoglavju bomo obravnavali le eliminacijo navadnega reza, posplošeni rez si bomo ogledali posebej, v naslednjem podpoglavju.

Če rez ni bil glaven, je moralo biti na levi ali desni strani pravilo, ki ni vpeljalo rezane formule. Denimo, da je bil na levi z L$\sqcap$ pravkar vpeljan veznik $\sqcap$ , nato pa smo rezali neko drugo formulo $C$:
\begin{prooftree}
    \derivation{0}{$\Gamma,A,C \Rightarrow \Delta$}
    \levopravilo{L$\sqcap$}
    \UnaryInfC{$\Gamma,A \sqcap B,C \Rightarrow \Delta$}

    \derivation{1}{$\Gamma' \Rightarrow C,\Delta'$}
    \pravilo{Rez}
    \BinaryInfC{$\Gamma,\Gamma',A \sqcap B \Rightarrow \Delta,\Delta'$}
\end{prooftree}
Ne glede na to kakšna je formula $C$ in ne glede na to kaj je bilo zadnje pravilo v drevesu $\D_1$, lahko to drevo preobrazimo tako, da bo zadoščalo indukcijski predpostavki:
\begin{prooftree}
    \derivation{0}{$\Gamma,A,C \Rightarrow \Delta$}
    \derivation{1}{$\Gamma' \Rightarrow C,\Delta'$}
    \pravilo{Rez}
    \BinaryInfC{$\Gamma,\Gamma',A \Rightarrow \Delta,\Delta'$}
    \pravilo{L$\sqcap$}
    \UnaryInfC{$\Gamma,\Gamma',A \sqcap B \Rightarrow \Delta,\Delta'$}
\end{prooftree}
Višina reza je očitno nižja in kot lahko vidimo sama formula $C$ na indukcijski korak ni vplivala. Zato pri rezu, ki ni glaven, namesto glede na rezano formulo, ločimo primere glede na zadnje pravilo pred rezom, ki ne vpelje formule $C$. Seveda tudi ni pomembno, ali se ta vpeljava zgodi v levem ali desnem poddrevesu nad rezom, saj je to asimetrično le, če obravnavamo rezano formulo. Pomembno pa je, ali je bil veznik vpeljan z levim ali desnim pravilom, zato si oglejmo še drugi primer, ko je vpeljana formula $A \sqcap B$:
\begin{prooftree}
    \derivation{0}{$\Gamma,C \Rightarrow A,\Delta$}
    \derivation{1}{$\Gamma,C \Rightarrow B,\Delta$}
    \levopravilo{R$\sqcap$}
    \BinaryInfC{$\Gamma,C \Rightarrow A \sqcap B,\Delta$}

    \derivation{2}{$\Gamma' \Rightarrow C,\Delta'$}
    \pravilo{Rez}
    \BinaryInfC{$\Gamma,\Gamma' \Rightarrow A \sqcap B,\Delta,\Delta'$}
\end{prooftree}
\dol
\begin{prooftree}
    \derivation{0}{$\Gamma,C \Rightarrow A,\Delta$}
    \derivation{2}{$\Gamma' \Rightarrow C,\Delta'$}
    \levopravilo{Rez}
    \BinaryInfC{$\Gamma,\Gamma' \Rightarrow A,\Delta,\Delta'$}

    \derivation{1}{$\Gamma,C \Rightarrow B,\Delta$}
    \derivation{2}{$\Gamma' \Rightarrow C,\Delta'$}
    \pravilo{Rez}
    \BinaryInfC{$\Gamma,\Gamma' \Rightarrow B,\Delta,\Delta'$}

    \pravilo{R$\sqcap$}
    \BinaryInfC{$\Gamma,\Gamma' \Rightarrow A \sqcap B,\Delta,\Delta'$}
\end{prooftree}
Oba nova reza sta očitno nižje višine kot prvotni rez, zato sta tudi nižjih stopenj. Ker v obeh dosedanjih primerih -- in v vseh prihodnjih -- strukture rezane formule ne obravnavamo, ranga ne bomo mogli znižati. Eliminacija reza, ki ni glaven, torej temelji na tem, da znižamo višino reza. Večina teh primerov je trivialnih, zato si pri veznikih oglejmo le še L$\star$ in R$\star$:
\begin{prooftree}
    \derivation{0}{$\Gamma,A,B,C \Rightarrow \Delta$}
    \levopravilo{L$\star$}
    \UnaryInfC{$\Gamma,A \star B,C \Rightarrow \Delta$}

    \derivation{1}{$\Gamma' \Rightarrow C,\Delta'$}
    \pravilo{Rez}
    \BinaryInfC{$\Gamma,\Gamma',A \star B \Rightarrow \Delta,\Delta'$}
\end{prooftree}
\dol
\begin{prooftree}
    \derivation{0}{$\Gamma,A,B,C \Rightarrow \Delta$}
    \derivation{1}{$\Gamma' \Rightarrow C,\Delta'$}
    \pravilo{Rez}
    \BinaryInfC{$\Gamma,\Gamma',A,B \Rightarrow \Delta,\Delta'$}

    \pravilo{L$\star$}
    \UnaryInfC{$\Gamma,\Gamma',A \star B \Rightarrow \Delta,\Delta'$}
\end{prooftree}
Kot v primeru levega pravila za veznik $\sqcap$, pravilo reza ter logično pravilo le zamenjamo, zato enostavno zadostimo indukcijski predpostavki. Primeri R$\sqcup$, R+, L$\multimap$, L$\negacija$ in R$\negacija$ potekajo enako. Oglejmo si torej še desno pravilo za veznik $\star$:
\begin{prooftree}
    \derivation{0}{$\Gamma,C \Rightarrow A,\Delta$}
    \derivation{1}{$\Gamma' \Rightarrow B,\Delta'$}
    \levopravilo{R$\star$}
    \BinaryInfC{$\Gamma,\Gamma',C \Rightarrow A \star B,\Delta,\Delta'$}

    \derivation{2}{$\Gamma'' \Rightarrow C,\Delta''$}
    \pravilo{Rez}
    \BinaryInfC{$\Gamma,\Gamma',\Gamma'' \Rightarrow A \star B,\Delta,\Delta',\Delta''$}
\end{prooftree}
\dol
\begin{prooftree}
    \derivation{0}{$\Gamma,C \Rightarrow A,\Delta$}
    \derivation{2}{$\Gamma'' \Rightarrow C,\Delta''$}
    \levopravilo{Rez}
    \BinaryInfC{$\Gamma,\Gamma'' \Rightarrow A,\Delta,\Delta''$}

    \derivation{1}{$\Gamma' \Rightarrow B,\Delta'$}
    \pravilo{R$\star$}
    \BinaryInfC{$\Gamma,\Gamma',\Gamma'' \Rightarrow A \star B,\Delta,\Delta',\Delta''$}
\end{prooftree}
Pri tem koraku je pomembno, da se formula $C$, ki jo režemo, kot predpostavka dodatno pojavi le v enem izmed skventov $\Gamma \Rightarrow A,\Delta$ in $\Gamma' \Rightarrow B,\Delta'$, ne v obeh. Pravilo R$\star$ namreč združi konteksta, tako da bi drugače morali $C$ iz sekventa rezati dvakrat. To pa nam tudi omogoči zelo enostaven korak indukcije, kot je izveden zgoraj. Enako se obravnava primera L+ in R$\multimap$, primer L$\sqcup$, ki nam med izjavnimi vezniki še edini ostane, pa se obravnava enako kot primer R$\sqcap$.

Oglejmo si še konstante. Tokrat seveda obravnavamo tudi konstanti $\top$ ter $\bot$, saj rez ni glaven in nimamo več težave z dejstvom, da imata vsaka le po eno logično pravilo:
\begin{prooftree}
    \AxiomC{}
    \levopravilo{R$\top$}
    \UnaryInfC{$\Gamma,C \Rightarrow \top,\Delta$}
    \derivation{}{$\Gamma' \Rightarrow C,\Delta'$}
    \pravilo{Rez}
    \BinaryInfC{$\Gamma,\Gamma' \Rightarrow \top,\Delta,\Delta'$}
\end{prooftree}
\dol
\begin{prooftree}
    \AxiomC{}
    \levopravilo{R$\top$}
    \UnaryInfC{$\Gamma,\Gamma' \Rightarrow \top,\Delta,\Delta'$}
\end{prooftree}
Ker lahko $\top$ med sklepi vedno dokažemo, lahko končni sekvent dobimo tudi ne da bi rezali formulo $C$, saj je ta že od začetka med predpostavkami nastala ,,odvečno''. Korak indukcije za konstanto $\bot$ izvedemo simetrično, saj zanjo velja ista lastnost, tokrat med predpostavkami. Pri obravnavi konstante $\enota$ desno pravilo ne pride v poštev, saj v nastalem sekventu druga formula kot $\enota$ ne more nastopati. Oglejmo si raje L$\enota$:
\begin{prooftree}
    \derivation{0}{$\Gamma,C \Rightarrow \Delta$}
    \levopravilo{L$\enota$}
    \UnaryInfC{$\Gamma,C,\enota \Rightarrow \Delta$}

    \derivation{1}{$\Gamma' \Rightarrow C,\Delta'$}
    \pravilo{Rez}
    \BinaryInfC{$\Gamma,\Gamma',\enota \Rightarrow \Delta,\Delta'$}
\end{prooftree}
\dol
\begin{prooftree}
    \derivation{0}{$\Gamma,C \Rightarrow \Delta$}
    \derivation{1}{$\Gamma' \Rightarrow C,\Delta'$}
    \pravilo{Rez}
    \BinaryInfC{$\Gamma,\Gamma' \Rightarrow \Delta,\Delta'$}

    \pravilo{L$\enota$}
    \UnaryInfC{$\Gamma,\Gamma',\enota \Rightarrow \Delta,\Delta'$}
\end{prooftree}
Seveda je korak indukcije pri konstantni $\nicla$ simetričen. Pri kvantifikatorjih so vsi primeri zopet enaki, zato obravnavajmo le enega:
\begin{prooftree}
    \derivation{0}{$\Gamma,C,A[t/x] \Rightarrow \Delta$}
    \levopravilo{L$\forall$}
    \UnaryInfC{$\Gamma,C,\forall x A \Rightarrow \Delta$}

    \derivation{1}{$\Gamma' \Rightarrow C,\Delta'$}
    \pravilo{Rez}
    \BinaryInfC{$\Gamma,\Gamma',\forall x A \Rightarrow \Delta,\Delta'$}
\end{prooftree}
\dol
\begin{prooftree}
    \derivation{0}{$\Gamma,C,A[t/x] \Rightarrow \Delta$}
    \derivation{1}{$\Gamma' \Rightarrow C,\Delta'$}
    \pravilo{Rez}
    \BinaryInfC{$\Gamma,\Gamma',A[t/x] \Rightarrow \Delta,\Delta'$}

    \pravilo{L$\forall$}
    \UnaryInfC{$\Gamma,\Gamma',\forall x A \Rightarrow \Delta,\Delta'$}
\end{prooftree}
Lotimo se sedaj eksponentov. Zaradi simetričnosti bomo podrobno spet obravnavali le veznik !. Začnimo s primerom, ko je bila nad rezom vpeljana formula !A s skrčitvijo, ošibitvijo ali levim pravilom. Vsi ti primeri so si enaki, zato obravnavajmo le skrčitev:
\begin{prooftree}
    \derivation{0}{$\Gamma,C,!A,!A \Rightarrow \Delta$}
    \levopravilo{C!}
    \UnaryInfC{$\Gamma,C,!A \Rightarrow \Delta$}

    \derivation{1}{$\Gamma' \Rightarrow C,\Delta'$}
    \pravilo{Rez}
    \BinaryInfC{$\Gamma,\Gamma',!A \Rightarrow \Delta,\Delta'$}
\end{prooftree}
\dol
\begin{prooftree}
    \derivation{0}{$\Gamma,C,!A,!A \Rightarrow \Delta$}
    \derivation{1}{$\Gamma' \Rightarrow C,\Delta'$}
    \pravilo{Rez}
    \BinaryInfC{$\Gamma,\Gamma',!A,!A \Rightarrow \Delta,\Delta'$}
    \pravilo{C!}
    \UnaryInfC{$\Gamma,\Gamma',!A \Rightarrow \Delta,\Delta'$}
\end{prooftree}
Pri pravilu R? pa moramo biti bolj pozorni. Da smo lahko desno pravilo za veznik ? sploh uporabili, je morala biti formula $C$, ki jo režemo, oblike $!C$, če je v danem poddrevesu na desni strani sekventa, ali $?C$, če je na levi. Oba primera sta simetrična, zato si poglejmo rezanje formule oblike $!C$:
\begin{prooftree}
    \derivation{0}{$!\Gamma,!C \Rightarrow A,?\Delta$}
    \levopravilo{R!}
    \UnaryInfC{$!\Gamma,!C \Rightarrow \ !A,?\Delta$}

    \derivation{1}{$\Gamma' \Rightarrow \ !C,\Delta'$}
    \pravilo{Rez}
    \BinaryInfC{$!\Gamma,\Gamma' \Rightarrow \ !A,?\Delta,\Delta'$}
\end{prooftree}
Tu pa nastopi težava, saj drevesa izpeljave ne moremo preoblikovati, ne da bi vedeli kaj več o desnem poddrevesu nad rezom:
\begin{prooftree}
    \derivation{0}{$!\Gamma,!C \Rightarrow A,?\Delta$}
    \derivation{1}{$\Gamma' \Rightarrow \ !C,\Delta'$}
    \pravilo{Rez}
    \BinaryInfC{$!\Gamma,\Gamma' \Rightarrow A,?\Delta,\Delta'$}
    \UnaryInfC{?}
\end{prooftree}
Če namreč $\Gamma'$ ni oblike $!\Gamma'$ in $\Delta'$ ni oblike $?\Delta'$, desnega pravila za veznik ! ne moremo več uporabiti. Ločiti je treba tri primere, glede na zadnje pravilo pred sekventom $\Gamma' \Rightarrow \ !C,\Delta'$:
\begin{enumerate}
    \item zadnje pravilo je R!,
    \item zadnje pravilo je L?,
    \item zadnje je katerokoli drugo pravilo.
\end{enumerate}
V tretjem primeru nimamo težav, saj smo vsak tak primer že obravnavali tekom tega podpoglavja. Drugi primer se pravzaprav ne more zgoditi, prvi pa se ne more zgoditi, razen če R! vpelje ravno formulo $!C$:
\begin{center}
    \begin{bprooftree}
        \AxiomC{$!\Gamma' \Rightarrow \ !C,B,?\Delta'$}
        \levopravilo{\textcolor{red}{R!}}
        \UnaryInfC{$!\Gamma' \Rightarrow \ !C,!B,?\Delta'$}
    \end{bprooftree}\qquad
    ali \qquad
    \begin{bprooftree}
        \AxiomC{$!\Gamma',B \Rightarrow \ !C,?\Delta'$}
        \pravilo{\textcolor{red}{L?}}
        \UnaryInfC{$!\Gamma',?B \Rightarrow \ !C,?\Delta'$}
    \end{bprooftree}
\end{center}
Zaradi formule $!C$, ki jo želimo rezati, sklepi namreč niso v celoti predznačeni z veznikom ?. Ostane nam torej le primer, ko je formula $!C$ na desni vpeljana tik nad rezom:
\begin{prooftree}
    \derivation{0}{$!\Gamma,!C \Rightarrow A,?\Delta$}
    \levopravilo{R!}
    \UnaryInfC{$!\Gamma,!C \Rightarrow \ !A,?\Delta$}

    \derivation{1}{$!\Gamma' \Rightarrow C,?\Delta'$}
    \pravilo{R!}
    \UnaryInfC{$!\Gamma' \Rightarrow \ !C,?\Delta'$}
    \pravilo{Rez}
    \BinaryInfC{$!\Gamma,!\Gamma' \Rightarrow \ !A,?\Delta,?\Delta'$}
\end{prooftree}
\dol
\begin{prooftree}
    \derivation{0}{$!\Gamma,!C \Rightarrow A,?\Delta$}

    \derivation{1}{$!\Gamma' \Rightarrow C,?\Delta'$}
    \pravilo{R!}
    \UnaryInfC{$!\Gamma' \Rightarrow \ !C,?\Delta'$}
    \pravilo{Rez}
    \BinaryInfC{$!\Gamma,!\Gamma' \Rightarrow A,?\Delta,?\Delta'$}
    \pravilo{R!}
    \UnaryInfC{$!\Gamma,!\Gamma' \Rightarrow \ !A,?\Delta,?\Delta'$}
\end{prooftree}
