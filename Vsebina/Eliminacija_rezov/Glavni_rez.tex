Od tu naprej bomo zavoljo preglednosti nad sekventi označevali še drevesa izpeljave, ki so do sekventov vodila.

Oblika glavnega reza je odvisna od vsake rezane formule posebej, zato je njegovo eliminacijo potrebno ločiti glede na veznik, ki rezano formulo sestavlja. Začnimo kar z glavnim rezom veznika $\sqcap$, kot v primeru \ref{gl rez in}:
\begin{prooftree}
    \derivation{0}{$\Gamma,A \Rightarrow \Delta$}
    \levopravilo{L$\sqcap$}
    \UnaryInfC{$\Gamma,A \sqcap B \Rightarrow \Delta$}

    \derivation{1}{$\Gamma' \Rightarrow A,\Delta'$}
    \derivation{2}{$\Gamma' \Rightarrow B,\Delta'$}
    \pravilo{R$\sqcap$}
    \BinaryInfC{$\Gamma' \Rightarrow A \sqcap B,\Delta'$}

    \pravilo{Rez}
    \BinaryInfC{$\Gamma,\Gamma' \Rightarrow \Delta,\Delta'$}
\end{prooftree}
\dol
\begin{prooftree}
    \derivation{0}{$\Gamma,A \Rightarrow \Delta$}
    \derivation{1}{$\Gamma' \Rightarrow A,\Delta'$}
    \pravilo{Rez}
    \BinaryInfC{$\Gamma,\Gamma' \Rightarrow \Delta,\Delta'$}
\end{prooftree}
Sklep pred puščico in po njej je enak, saj iz istih poddreves dokažemo isti sekvent. Tako smo rez stopnje $(\R(A) + \R(B) + 1,\h)$ zamenjali z rezom stopnje $(\R(A),\h')$, ki je očitno manjša. Podobno lahko naredimo za glavni rez $A \star B$:
\begin{prooftree}
    \derivation{0}{$\Gamma,A,B \Rightarrow \Delta$}
    \levopravilo{L$\star$}
    \UnaryInfC{$\Gamma,A \star B \Rightarrow \Delta$}

    \derivation{1}{$\Gamma' \Rightarrow A,\Delta'$}
    \derivation{2}{$\Gamma'' \Rightarrow B,\Delta''$}
    \pravilo{R$\star$}
    \BinaryInfC{$\Gamma',\Gamma'' \Rightarrow A \star B,\Delta',\Delta''$}

    \pravilo{Rez}
    \BinaryInfC{$\Gamma,\Gamma',\Gamma'' \Rightarrow \Delta,\Delta',\Delta''$}
\end{prooftree}
\dol
\begin{prooftree}
    \derivation{0}{$\Gamma,A,B \Rightarrow \Delta$}
    \derivation{1}{$\Gamma' \Rightarrow A,\Delta'$}
    \levopravilo{Rez}
    \BinaryInfC{$\Gamma,\Gamma',B \Rightarrow \Delta,\Delta'$}

    \derivation{2}{$\Gamma'' \Rightarrow B,\Delta''$}
    \pravilo{Rez}
    \BinaryInfC{$\Gamma,\Gamma',\Gamma'' \Rightarrow \Delta,\Delta',\Delta''$}
\end{prooftree}
Tokrat smo prvotno drevo izpeljave prevedli na drevo z dvema rezoma, a imata oba nižjo stopnjo, saj je rang rezane formule očitno nižji. Zelo podobno kot zgornje dva primera izvedemo korak indukcije za $A \sqcup B$ ter $A+B$:
\begin{prooftree}
    \derivation{0}{$\Gamma,A \Rightarrow \Delta$}
    \derivation{1}{$\Gamma,B \Rightarrow \Delta$}
    \levopravilo{L$\sqcup$}
    \BinaryInfC{$\Gamma,A \sqcup B \Rightarrow \Delta$}

    \derivation{2}{$\Gamma' \Rightarrow A,\Delta'$}
    \pravilo{R$\sqcup$}
    \UnaryInfC{$\Gamma' \Rightarrow A \sqcup B,\Delta'$}

    \pravilo{Rez}
    \BinaryInfC{$\Gamma,\Gamma' \Rightarrow \Delta,\Delta'$}
\end{prooftree}
\dol
\begin{prooftree}
    \derivation{0}{$\Gamma,A \Rightarrow \Delta$}
    \derivation{2}{$\Gamma' \Rightarrow A,\Delta'$}
    \pravilo{Rez}
    \BinaryInfC{$\Gamma,\Gamma' \Rightarrow \Delta,\Delta'$}
\end{prooftree}
Kot vidimo je zgornji korak indukcije simetričen koraku indukcije za $A \sqcap B$, saj sta tudi veznika sama simetrična. Enako je korak indukcije za $A+B$ simetričen koraku indukcije za $A \star B$:
\begin{prooftree}
    \derivation{0}{$\Gamma,A \Rightarrow \Delta$}
    \derivation{1}{$\Gamma',B \Rightarrow \Delta'$}
    \levopravilo{L+}
    \BinaryInfC{$\Gamma,\Gamma',A + B \Rightarrow \Delta,\Delta'$}

    \derivation{2}{$\Gamma'' \Rightarrow A,B,\Delta''$}
    \pravilo{R+}
    \UnaryInfC{$\Gamma'' \Rightarrow A + B,\Delta''$}

    \pravilo{Rez}
    \BinaryInfC{$\Gamma,\Gamma',\Gamma'' \Rightarrow \Delta,\Delta',\Delta''$}
\end{prooftree}
\dol
\begin{prooftree}
    \derivation{0}{$\Gamma,A \Rightarrow \Delta$}
    \derivation{2}{$\Gamma'' \Rightarrow A,B,\Delta''$}
    \levopravilo{Rez}
    \BinaryInfC{$\Gamma,\Gamma'' \Rightarrow B,\Delta,\Delta''$}

    \derivation{1}{$\Gamma',B \Rightarrow \Delta'$}
    \pravilo{Rez}
    \BinaryInfC{$\Gamma,\Gamma',\Gamma'' \Rightarrow \Delta,\Delta',\Delta''$}
\end{prooftree}
Naslednji primer, ki ga obravnavamo, je veznik $\multimap$. Zopet se rez prevede na dva reza nižje stopnje, na enak način kot pri veznikih $A\star B$ in $A+B$:
\begin{prooftree}
	\derivation{0}{$\Gamma,A \Rightarrow \Delta$}
    \derivation{1}{$\Gamma',B \Rightarrow \Delta'$}
    \levopravilo{L$\multimap$}
    \BinaryInfC{$\Gamma,\Gamma',A \multimap B \Rightarrow \Delta,\Delta'$}

    \derivation{2}{$\Gamma'',A \Rightarrow B,\Delta''$}
    \pravilo{R$\multimap$}
    \UnaryInfC{$\Gamma'' \Rightarrow A \multimap B,\Delta''$}

    \pravilo{Rez}
    \BinaryInfC{$\Gamma,\Gamma',\Gamma'' \Rightarrow \Delta,\Delta',\Delta''$}
\end{prooftree}
\dol
\begin{prooftree}
	\derivation{0}{$\Gamma,A \Rightarrow \Delta$}
	\derivation{2}{$\Gamma'',A \Rightarrow B,\Delta''$}
    \levopravilo{Rez}
    \BinaryInfC{$\Gamma,\Gamma'' \Rightarrow B,\Delta,\Delta''$}

    \derivation{1}{$\Gamma',B \Rightarrow \Delta'$}
    \pravilo{Rez}
    \BinaryInfC{$\Gamma,\Gamma',\Gamma'' \Rightarrow \Delta,\Delta',\Delta''$}
\end{prooftree}
Poslednji izmed izjavnih veznikov, ki nam ga je potrebno obravnavati je negacija:
\begin{prooftree}
    \derivation{0}{$\Gamma \Rightarrow A,\Delta$}
	\levopravilo{L$\negacija$}
	\UnaryInfC{$\Gamma,\negacija A \Rightarrow \Delta$}

	\derivation{1}{$\Gamma',A \Rightarrow \Delta'$}
	\pravilo{R$\negacija$}
	\UnaryInfC{$\Gamma' \Rightarrow \negacija A,\Delta'$}

	\pravilo{Rez}
	\BinaryInfC{$\Gamma,\Gamma' \Rightarrow \Delta,\Delta'$}
\end{prooftree}
\dol
\begin{prooftree}
    \derivation{0}{$\Gamma \Rightarrow A,\Delta$}
	\derivation{1}{$\Gamma',A \Rightarrow \Delta'$}
	\pravilo{Rez}
	\BinaryInfC{$\Gamma,\Gamma' \Rightarrow \Delta,\Delta'$}
\end{prooftree}
Pri eliminaciji glavnega reza, kjer režemo izjavno konstanto, imamo moč obravnavati le konstanti $\enota$ in $\nicla$, saj za $\top$ levo pravilo ne obstaja, za $\bot$ pa ni desnega. Glavni rez, kjer režemo $\top$ ali $\bot$ se torej ne more zgoditi. Za konstanti $\enota$ in $\nicla$ eliminacija glavnega reza izgleda sledeče:
\begin{center}
    \begin{bprooftree}
        \derivation{}{$\Gamma \Rightarrow \Delta$}
        \levopravilo{L$\enota$}
        \UnaryInfC{$\Gamma,\enota \Rightarrow \Delta$}

        \AxiomC{}
        \pravilo{R$\enota$}
        \UnaryInfC{$ \Rightarrow \enota$}

        \pravilo{Rez}
        \BinaryInfC{$\Gamma \Rightarrow \Delta$}
    \end{bprooftree}
    \begin{bprooftree}
        \AxiomC{}
        \levopravilo{L$\nicla$}
        \UnaryInfC{$\nicla \Rightarrow$}

        \derivation{}{$\Gamma \Rightarrow \Delta$}
        \pravilo{R$\nicla$}
        \UnaryInfC{$\Gamma \Rightarrow \nicla,\Delta$}

        \pravilo{Rez}
        \BinaryInfC{$\Gamma \Rightarrow \Delta$}
    \end{bprooftree}
\end{center}
\begin{align*}
    &\downarrow & &\downarrow
\end{align*}
\begin{center}
    \begin{bprooftree}
        \derivation{}{$\Gamma \Rightarrow \Delta$}
    \end{bprooftree} \qquad \qquad \qquad \quad
    \begin{bprooftree}
        \derivation{}{$\Gamma \Rightarrow \Delta$}
    \end{bprooftree}
\end{center}
Kot lahko vidimo sta primera dokaj trivialna, saj je že sam rez take vrste trivialen. Ker smo rez popolnoma eliminirali je indukcijski predpostavki zadoščeno na prazno. Sedaj si oglejmo eliminacijo glavnega reza obeh kvantifikatorjev. Zopet s $t$ označimo specifičen term¸ z $y$ pa neko (svežo) prosto spremenljivko:
\begin{prooftree}
    \derivation{0}{$\Gamma,A[t/x] \Rightarrow \Delta$}
    \levopravilo{L$\forall$}
    \UnaryInfC{$\Gamma,\forall x A \Rightarrow \Delta$}

    \derivation{1}{$\Gamma' \Rightarrow A[y/x],\Delta'$}
    \pravilo{R$\forall$}
    \UnaryInfC{$\Gamma' \Rightarrow \forall x A,\Delta'$}

    \pravilo{Rez}
    \BinaryInfC{$\Gamma,\Gamma' \Rightarrow \Delta,\Delta'$}
\end{prooftree}
\dol
\begin{prooftree}
    \derivation{0}{$\Gamma,A[t/x] \Rightarrow \Delta$}

    \derivation{1}{$\Gamma' \Rightarrow A[y/x],\Delta'$}
    \pravilo{$y:=t$}
    \UnaryInfC{$\Gamma' \Rightarrow A[t/x],\Delta'$}

    \pravilo{Rez}
    \BinaryInfC{$\Gamma,\Gamma' \Rightarrow \Delta,\Delta'$}
\end{prooftree}
V desnem poddrevesu novonastalega drevesa izpeljave spremenljivko $y$ nadomestimo s specifičnim termom $t$. Ker je bil $y$ \emph{prosta} spremenljivka, to lahko naredimo. Tako dobimo formulo, ki je enaka formuli v levem poddrevesu, in jo zato lahko režemo. V definiciji ranga formule so pomembni le vezniki, ki formulo sestavljajo, ne pa tudi termi, ki se v njej pojavljajo, zato je $\R(A) = \R(A[t/x])$. Dalje je:
$$
\R(\forall x A) = \R(A) + 1 = \R(A[t/x]) + 1
$$
To pomeni, da je stopnja novega reza nižja od stopnje prvega, torej je indukcijski predpostavki zadoščeno. Postopek pri eliminaciji reza eksistenčnega kvantifikatorja je simetričen:
\begin{prooftree}
    \derivation{0}{$\Gamma,A[y/x] \Rightarrow \Delta$}
    \levopravilo{L$\exists$}
    \UnaryInfC{$\Gamma,\exists x A \Rightarrow \Delta$}

    \derivation{1}{$\Gamma' \Rightarrow A[t/x],\Delta'$}
    \pravilo{R$\exists$}
    \UnaryInfC{$\Gamma' \Rightarrow \exists x A,\Delta'$}

    \pravilo{Rez}
    \BinaryInfC{$\Gamma,\Gamma' \Rightarrow \Delta,\Delta'$}
\end{prooftree}
\dol
\begin{prooftree}
    \derivation{0}{$\Gamma,A[y/x] \Rightarrow \Delta$}
    \levopravilo{$y:=t$}
    \UnaryInfC{$\Gamma,A[t/x] \Rightarrow \Delta$}

    \derivation{1}{$\Gamma' \Rightarrow A[t/x],\Delta'$}
    \pravilo{Rez}
    \BinaryInfC{$\Gamma,\Gamma' \Rightarrow \Delta,\Delta'$}
\end{prooftree}
