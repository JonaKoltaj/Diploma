Začnimo z navadnim rezom. Kot omenjeno na začetku tega dokaza, je pri bazi indukcije potrebno preveriti, kaj se zgodi, ko je stopnja reza enaka $(1,\h)$, za nek $h\in\mathbb{N}_{\geq3}$. Denimo, da $\h\neq3$:
\begin{prooftree}
    \derivation{0}{$\Gamma,A \Rightarrow \Delta$}
    \derivation{1}{$\Gamma' \Rightarrow A,\Delta'$}
    \pravilo{Rez}
    \BinaryInfC{$\Gamma,\Gamma' \Rightarrow \Delta,\Delta'$}
\end{prooftree}
To pomeni, da $\h(\D_0) \neq 0$ ali $\h(\D_1) \neq 0$. Brez škode za splošnost lahko predpostavimo, da to velja prvo. Ker je $\R(A) = 1$, je $A$ po definiciji lahko le osnovna formula ali pa ena izmed konstant. Osnovna formula je lahko vpeljana le s strani pravila aksiom, zato je moralo zadnje pravilo vpeljati neko drugo formulo v $\Gamma$ ali $\Delta$. To pa pomeni, da se lahko skličemo na podpoglavje \ref{non principal}. Enako se zgodi, če je bila $A$ ena izmed konstant, ki na levi ni bila vpeljana.

Če pe je bila formula $A$ neka konstanta, ki jo je vpeljalo zadnje pravilo na levi, imamo tri možnosti. Če je bila vpeljana tudi na drugi strani, je to glavni rez konstante in lahko se skličemo na podpoglavje \ref{gl rez vezniki}. Če je bila na desni vpeljana neka druga formula, smo to zopet že obravnavali, tako da nam preostane le primer, ko je na desni pravilo aksioma:
\begin{prooftree}
    \derivation{0}{$\Gamma,A \Rightarrow \Delta$}
    \AxiomC{}
    \pravilo{Ax}
    \UnaryInfC{$A \Rightarrow A$}
    \pravilo{Rez}
    \BinaryInfC{$\Gamma,A \Rightarrow \Delta$}
\end{prooftree}
\dol
\begin{prooftree}
    \derivation{0}{$\Gamma,A \Rightarrow \Delta$}
\end{prooftree}
Kot lahko vidimo je to precej trivialen korak, saj rezanje formule, ki je bila ravnokar vpeljana z aksiomom, nima učinka.
Ostane nam le, da preverimo ali lahko rez eliminiramo, če $\h=3$. Drevo je moralo biti oblike:
\begin{prooftree}
    \AxiomC{}
    \levopravilo{Ax}
    \UnaryInfC{$A \Rightarrow A$}

    \AxiomC{}
    \pravilo{Ax}
    \UnaryInfC{$A \Rightarrow A$}

    \pravilo{Rez}
    \BinaryInfC{$A \Rightarrow A$}
\end{prooftree}
\dol
\begin{prooftree}
    \AxiomC{}
    \pravilo{Ax}
    \UnaryInfC{$A \Rightarrow A$}
\end{prooftree}

Baza indukcije pri posplošenem rezu se obravnava za $\R = 2$, saj je to najnižji možni rang rezane formule pri take vrste rezu. Višina reza ne more biti $3$, saj na levi ni moglo nastopati pravilo aksioma. Vsi argumenti, kako to prevesti na probleme prejšnjih podpoglavij, so enaki kot pri navadnem rezu, kjer $\h\neq3$. Oglejmo si le, kaj se zgodi, če je na desni strani tik nad rezom aksiom:

\begin{prooftree}
    \derivation{0}{$\Gamma,(!A)^n \Rightarrow \Delta$}
    \AxiomC{}
    \pravilo{Ax}
    \UnaryInfC{$!A \Rightarrow !A$}
    \pravilo{Rez!$_n$}
    \BinaryInfC{$\Gamma,!A \Rightarrow \Delta$}
\end{prooftree}
\dol
\begin{prooftree}
    \derivation{0}{$\Gamma,(!A)^n \Rightarrow \Delta$}
    \pravilo{C!$\times (n-1)$}
    \UnaryInfC{$\Gamma,!A \Rightarrow \Delta$}
\end{prooftree}

S tem smo dokaz izreka o eliminaciji rezov zaključili, saj smo obdelali vse primere koraka indukcije ter bazni primer za obe vrsti rezov.
