Lotimo se sedaj eksponentov. Ker imata oba eksponenta štiri pravila, moramo ločiti glavni rez na več primerov. Pri vezniku ! imamo tri pravila, ki veznik vpeljejo na levi ter eno pravilo, ki veznik vpelje na desni. Zato moramo ločiti glavni rez formule $!A$ na tri primere, po en primer za vsako levo pravilo. Ogljemo si naprej glavni rez, kjer je $!A$ na levi vpeljan s skrčitvijo, na desni pa z desnim pravilom vpeljave.
\begin{prooftree}
    \AxiomC{$\Gamma,!A,!A \Rightarrow \Delta$}
    \levopravilo{C!}
    \UnaryInfC{$\Gamma,!A \Rightarrow \Delta$}

    \AxiomC{$!\Gamma' \Rightarrow A,?\Delta'$}
    \pravilo{R!}
    \UnaryInfC{$!\Gamma' \Rightarrow \ !A,?\Delta'$}

    \pravilo{Rez}
    \BinaryInfC{$\Gamma,!\Gamma' \Rightarrow \Delta,?\Delta'$}
\end{prooftree}
Mikalo nas bi zgornje drevo izpeljave zamenjati s sledečim.
\begin{prooftree}
    \AxiomC{$\Gamma,!A,!A \Rightarrow \Delta$}

    \AxiomC{$!\Gamma' \Rightarrow A,?\Delta'$}
    \pravilo{R!}
    \UnaryInfC{$!\Gamma' \Rightarrow \ !A,?\Delta'$}

    \levopravilo{Rez}
    \BinaryInfC{$\Gamma,!\Gamma',!A \Rightarrow \Delta,?\Delta'$}

    \AxiomC{$!\Gamma' \Rightarrow A,?\Delta'$}
    \pravilo{R!}
    \UnaryInfC{$!\Gamma' \Rightarrow \ !A,?\Delta'$}

    \pravilo{Rez}
    \BinaryInfC{$\Gamma,!\Gamma',!\Gamma' \Rightarrow \Delta,?\Delta',?\Delta'$}
    \pravilo{C!$\times|\Gamma'|$}
    \UnaryInfC{$\Gamma,!\Gamma' \Rightarrow \Delta,?\Delta',?\Delta'$}
    \pravilo{C?$\times|\Delta'|$}
    \UnaryInfC{$\Gamma,!\Gamma' \Rightarrow \Delta,?\Delta'$}
\end{prooftree}
Vendar s tem ne zadostimo indukcijski predpostavki, saj nobeno izmed dreves nad spodnjim izmed novo pridelanih rezov ni poddrevo prvotnih dveh dreves nad rezom, poleg tega pa se je ohranila tudi kompleskonst rezane formule, saj še vedno režemo $!A$. Zavoljo tega, da lahko ta korak indukcije vseeno opravimo, potrebujemo pomožno razširjeno pravilo reza.
\begin{definicija}
    \emph{Posplošeni pravili reza}, označeni z Rez!$_n$ ter Rez?$_{n}$, sta definirani za vsak $n \in \mathbb{N}$.
    \begin{prooftree}
        \AxiomC{$\Gamma,(!A)^n \Rightarrow \Delta$}
        \AxiomC{$\Gamma' \Rightarrow \ !A,\Delta'$}
        \pravilo{Rez!$_n$}
        \BinaryInfC{$\Gamma,\Gamma' \Rightarrow \Delta,\Delta'$}
    \end{prooftree}
    \begin{prooftree}
        \AxiomC{$\Gamma,?A \Rightarrow \Delta$}
        \AxiomC{$\Gamma' \Rightarrow (?A)^n,\Delta'$}
        \pravilo{Rez?$_{n}$}
        \BinaryInfC{$\Gamma,\Gamma' \Rightarrow \Delta,\Delta'$}
    \end{prooftree}
\end{definicija}

\begin{opomba}
    Formula $(!A)^n$ v definiciji predstavjla $n$ formul $!A$, ločenih z vejico. Pravili Rez!$_{1}$ ter Rez?$_{1}$ sta torej le pravilo Rez, kjer režemo ali formulo $!A$, ali pa formulo $?A$.
\end{opomba}

\begin{lema}
    Pravili Rez!$_n$ ter Rez?$_{n}$ sta dopustni, kar pomeni, da ju lahko izpeljemo iz že definiranih pravil linearne logike.
\end{lema}

\begin{dokaz}
    Lemo dokažemo z indukcijo na številu $n$. Primer ko je $n=1$ je seveda le običajno pravilo reza, kot omenjeno že v zgornji opombi. Če predpostavimo, da pravilo Rez!$_n$ že znamo izpeljati, lahko izpeljemo Rez!$_{n+1}$ na naslednji način.
    \begin{prooftree}
        \AxiomC{$\Gamma,(!A)^{n+1} \Rightarrow \Delta$}
        \UnaryInfC{$\Gamma,(!A)^{n-1},!A,!A \Rightarrow \Delta$}
        \levopravilo{C!}
        \UnaryInfC{$\Gamma,(!A)^{n-1},!A \Rightarrow \Delta$}
        \UnaryInfC{$\Gamma,(!A)^n \Rightarrow \Delta$}

        \AxiomC{$\Gamma' \Rightarrow \ !A,\Delta'$}
        \pravilo{Rez!$_n$}
        \BinaryInfC{$\Gamma,\Gamma' \Rightarrow \Delta,\Delta'$}
    \end{prooftree}
    Pri indukcijskem koraku iz $n=1$ na $n=2$ moramo paziti, saj se v drugi vrstici dokaza pojavi izraz $(!A)^0$. To enostavno interpretiramo kot prazen nabor formul. Indukcijski korak za Rez?$_{n}$ je simetričen.
\end{dokaz}

Sedaj izrek \ref{izrek}, ki ga dokazujemo, preoblikujmo, tako da bo namesto le navadnega pravila reza vseboval tudi  posplošeno pravilo reza.
\begin{izrek}
    Vsak sekvent, dokazan z uporabo pravila reza ali posplošenega pravila reza, lahko dokažemo tudi brez uporabe pravila reza ali posplošenega pravila reza.
\end{izrek}

Ker je posplošeno pravilo reza dopustno, je novo formulirani izrek ekvivalenten izreku \ref{izrek}. Vendar si sedaj v dokazu lahko pomagamo s posplošenim pravilom reza. Med indukcijskim korakom torej lahko drevo izpeljave preobrazimo tako, da bo zadoščalo indukcijski predpostavki za kateregakoli izmed rezov, navadnega ali posplošenega. A sedaj je potrebno znati eliminirati tudi slednjega. To pomeni, da je potrebno obravnavati korak indukcije tako za pravilo Rez, kot za pravili Rez!$_n$ in Rez?$_n$, vsakič ko obravnavamo rez veznika ! ali ?. V določenih primerih lahko obravnavamo tako posplošeno pravilo reza kot navadno hkrati, saj korak indukcije za Rez!$_n$ lahko izgleda enako tudi pri $n=1$.

Lotimo se še enkrat drevesa izpeljave, kjer na levi $!A$ vpeljemo s skrčitvijo, tokrat s posplošenim pravilom reza v žepu.
\begin{prooftree}
    \AxiomC{$\Gamma,(!A)^{n+1} \Rightarrow \Delta$}
    \levopravilo{C!}
    \UnaryInfC{$\Gamma,(!A)^n \Rightarrow \Delta$}

    \AxiomC{$!\Gamma' \Rightarrow A,?\Delta'$}
    \pravilo{R!}
    \UnaryInfC{$!\Gamma' \Rightarrow \ !A,?\Delta'$}

    \pravilo{Rez!$_n$}
    \BinaryInfC{$\Gamma,!\Gamma' \Rightarrow \Delta,?\Delta'$}
\end{prooftree}
\dol
\begin{prooftree}
    \AxiomC{$\Gamma,(!A)^{n+1} \Rightarrow \Delta$}

    \AxiomC{$!\Gamma' \Rightarrow A,?\Delta'$}
    \pravilo{R!}
    \UnaryInfC{$!\Gamma' \Rightarrow \ !A,?\Delta'$}

    \pravilo{Rez!$_{n+1}$}
    \BinaryInfC{$\Gamma,!\Gamma' \Rightarrow \Delta,?\Delta'$}
\end{prooftree}
Zgornje seveda velja tudi za $n=1$, torej za navadno pravilo reza. Pri novem drevesu izpeljave je sedaj levo drevo nad rezom poddrevo levega drevesa nad prejšnjim rezom in korak indukcije je opravljen. Oglejmo si sedaj še glavni rez¸ kjer je formula $!A$ na levi vpeljana z ošibitvijo. Tu korak indukcije za posplošeni rez, kjer $n\neq1$, ter navadni rez ni združljiv, zato primera ločimo.
\begin{prooftree}
    \AxiomC{$\Gamma \Rightarrow \Delta$}
    \levopravilo{W!}
    \UnaryInfC{$\Gamma,!A \Rightarrow \Delta$}

    \AxiomC{$!\Gamma' \Rightarrow A,?\Delta'$}
    \pravilo{R!}
    \UnaryInfC{$!\Gamma' \Rightarrow \ !A,?\Delta'$}

    \pravilo{Rez}
    \BinaryInfC{$\Gamma,!\Gamma' \Rightarrow \Delta,?\Delta'$}
\end{prooftree}
\dol
\begin{prooftree}
	\AxiomC{$\Gamma \Rightarrow \Delta$}
    \levopravilo{W!$\times|\Gamma'|$}
    \UnaryInfC{$\Gamma,!\Gamma' \Rightarrow \Delta$}
    \pravilo{W?$\times|\Delta'|$}
    \UnaryInfC{$\Gamma,!\Gamma' \Rightarrow \Delta,?\Delta'$}
\end{prooftree}
Tako smo v tem primeru korak indukcije opravili za navadni rez, za Rez!$_n$, kjer je $n\geq2$ pa je postopek sledeč.
\begin{prooftree}
    \AxiomC{$\Gamma,(!A)^n \Rightarrow \Delta$}
    \levopravilo{W!}
    \UnaryInfC{$\Gamma,(!A)^{n+1} \Rightarrow \Delta$}

    \AxiomC{$!\Gamma' \Rightarrow A,?\Delta'$}
    \pravilo{R!}
    \UnaryInfC{$!\Gamma' \Rightarrow \ !A,?\Delta'$}

    \pravilo{Rez!$_{n+1}$}
    \BinaryInfC{$\Gamma,!\Gamma' \Rightarrow \Delta,?\Delta'$}
\end{prooftree}
\dol
\begin{prooftree}
    \AxiomC{$\Gamma,(!A)^n \Rightarrow \Delta$}

    \AxiomC{$!\Gamma' \Rightarrow A,?\Delta'$}
    \pravilo{R!}
    \UnaryInfC{$!\Gamma' \Rightarrow \ !A,?\Delta'$}

    \pravilo{Rez!$_n$}
    \BinaryInfC{$\Gamma,!\Gamma' \Rightarrow \Delta,?\Delta'$}
\end{prooftree}
Pri obravnavi glavnega reza formule $!A$ z levim pravilom vpeljave na levi je zopet potrebno ločiti Rez!$_n$ na primer ko je $n=1$ ter ko $n\neq1$. Ogljemo si zopet najprej navadno pravilo reza.
\begin{prooftree}
    \AxiomC{$\Gamma,A \Rightarrow \Delta$}
    \levopravilo{L!}
    \UnaryInfC{$\Gamma,!A \Rightarrow \Delta$}

    \AxiomC{$!\Gamma' \Rightarrow A,?\Delta'$}
    \pravilo{R!}
    \UnaryInfC{$!\Gamma' \Rightarrow \ !A,?\Delta'$}

    \pravilo{Rez}
    \BinaryInfC{$\Gamma,!\Gamma' \Rightarrow \Delta,?\Delta'$}
\end{prooftree}
\dol
\begin{prooftree}
    \AxiomC{$\Gamma,A \Rightarrow \Delta$}
    \AxiomC{$!\Gamma' \Rightarrow A,?\Delta'$}
    \pravilo{Rez}
    \BinaryInfC{$\Gamma,!\Gamma' \Rightarrow \Delta,?\Delta'$}
\end{prooftree}
Za eliminacijo pravila Rez!$_n$, ko je $n\geq2$, pa bo korak indukcije spet malce bolj zahteven.
\begin{prooftree}
    \AxiomC{$\Gamma,(!A)^{n-1},A \Rightarrow \Delta$}
    \levopravilo{L!}
    \UnaryInfC{$\Gamma,(!A)^n \Rightarrow \Delta$}

    \AxiomC{$!\Gamma' \Rightarrow A,?\Delta'$}
    \pravilo{R!}
    \UnaryInfC{$!\Gamma' \Rightarrow \ !A,?\Delta'$}

    \pravilo{Rez!$_n$}
    \BinaryInfC{$\Gamma,!\Gamma' \Rightarrow \Delta,?\Delta'$}
\end{prooftree}
\dol
\begin{prooftree}
    \AxiomC{$\Gamma,(!A)^{n-1},A \Rightarrow \Delta$}

    \AxiomC{$!\Gamma' \Rightarrow A,?\Delta'$}
    \pravilo{R!}
    \UnaryInfC{$!\Gamma' \Rightarrow \ !A,?\Delta'$}

    \levopravilo{Rez!$_{n-1}$}
    \BinaryInfC{$\Gamma,!\Gamma',A \Rightarrow \Delta,?\Delta'$}

    \AxiomC{$!\Gamma' \Rightarrow A,?\Delta'$}
    \pravilo{Rez}
    \BinaryInfC{$\Gamma,!\Gamma',!\Gamma' \Rightarrow \Delta,?\Delta',?\Delta'$}
    \pravilo{C!$\times|\Gamma'|$}
    \UnaryInfC{$\Gamma,!\Gamma' \Rightarrow \Delta,?\Delta',?\Delta'$}
    \pravilo{C?$\times|\Delta'|$}
    \UnaryInfC{$\Gamma,!\Gamma' \Rightarrow \Delta,?\Delta'$}
\end{prooftree}
Novonastalo drevo izpeljave bi nas lahko spominjalo na problem iz začetka tega podpoglavja, a smo tu za spremembo res zadostili indukcijski predpostavki, saj je desno poddrevo nad najnižjim izmed rezov res poddrevo desnega izmed dreves nad prvotnim rezom.

Eliminacija glavnega reza, ki reže formulo $?A$, je popolnoma simetrična in sledi natanko vsem korakom, opisanim v procesu eliminacije glavnega reza, ki reže formulo $!A$, zato je v tem delu ne bomo posebej obravnavali.
