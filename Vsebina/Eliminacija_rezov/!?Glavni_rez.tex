Eliminacija glavnega reza eksponentov zahteva posebno obravnavo, saj se dokazovanje tu nekoliko zaplete. Veznika ! in ? sta simetrična, zato bomo podrobno obravnavali le veznik !, bralec pa si lahko sam izpelje dokaze še za veznik ?.

Veznik ! ima štiri logična pravila, ki ga definirajo. Tri pravila veznik vpeljejo na levi strani sekventa, desno pravilo pa ga vpelje na desni. Zato moramo glavni rez formule $!A$ ločiti na tri primere, glede na to kako je bil vpeljan na levi strani. Ogljemo si naprej glavni rez, kjer je $!A$ na levi vpeljan s skrčitvijo:
\begin{prooftree}
    \derivation{0}{$\Gamma,!A,!A \Rightarrow \Delta$}
    \levopravilo{C!}
    \UnaryInfC{$\Gamma,!A \Rightarrow \Delta$}

    \derivation{1}{$!\Gamma' \Rightarrow A,?\Delta'$}
    \pravilo{R!}
    \UnaryInfC{$!\Gamma' \Rightarrow \ !A,?\Delta'$}

    \pravilo{Rez}
    \BinaryInfC{$\Gamma,!\Gamma' \Rightarrow \Delta,?\Delta'$}
\end{prooftree}
Mikalo bi nas zgornje drevo izpeljave zamenjati s sledečim:
\begin{prooftree}
    \derivation{0}{$\Gamma,!A,!A \Rightarrow \Delta$}

    \derivation{1}{$!\Gamma' \Rightarrow A,?\Delta'$}
    \pravilo{R!}
    \UnaryInfC{$!\Gamma' \Rightarrow \ !A,?\Delta'$}

    \levopravilo{Rez}
    \BinaryInfC{$\Gamma,!\Gamma',!A \Rightarrow \Delta,?\Delta'$}

    \derivation{1}{$!\Gamma' \Rightarrow A,?\Delta'$}
    \pravilo{R!}
    \UnaryInfC{$!\Gamma' \Rightarrow \ !A,?\Delta'$}

    \pravilo{Rez}
    \BinaryInfC{$\Gamma,!\Gamma',!\Gamma' \Rightarrow \Delta,?\Delta',?\Delta'$}
    \pravilo{C!$\times|\Gamma'|$}
    \UnaryInfC{$\Gamma,!\Gamma' \Rightarrow \Delta,?\Delta',?\Delta'$}
    \pravilo{C?$\times|\Delta'|$}
    \UnaryInfC{$\Gamma,!\Gamma' \Rightarrow \Delta,?\Delta'$}
\end{prooftree}
Če označimo z $(\R,\h)$ stopnjo prvotnega reza, z $(\R',\h')$ stopjo zgornjega izmed novih rezov, z $(\R'',\h'')$ pa stopnjo spodnjega, velja:
\begin{align*}
    (\R,\h) &= (\R(A) + 1,\h(\D_0) + \h(\D_1) + 5)\\
    (\R',\h') &= (\R(A) + 1,\h(\D_0) + \h(\D_1) + 4)\\
    (\R'',\h'') &= (\R(A) + 1,\h(\D_0) + 2*\h(\D_1) + 7)
\end{align*}
Takoj lahko vidimo, da je rang rezane formule v vseh primerih enak, zato bi bilo potrebno zmanjšati višino. Zgornji izmed novih rezov ima sicer nižjo višino kot prvotni rez, pri spodnjem pa višina znatno naraste. To pomeni, da koraku indukcije ne zadostimo. Da bi lahko ta korak indukcije vseeno opravili, potrebujemo pomožno (razširjeno) pravilo reza.

\begin{definicija}
    \emph{Posplošeni pravili reza}, označeni z Rez!$_n$ in Rez?$_{n}$, sta definirani za vsak $n\in\mathbb{N}_{>0}$;
    \begin{prooftree}
        \AxiomC{$\Gamma,(!A)^n \Rightarrow \Delta$}
        \AxiomC{$\Gamma' \Rightarrow \ !A,\Delta'$}
        \pravilo{Rez!$_n$}
        \BinaryInfC{$\Gamma,\Gamma' \Rightarrow \Delta,\Delta'$}
    \end{prooftree}
    \begin{prooftree}
        \AxiomC{$\Gamma,?A \Rightarrow \Delta$}
        \AxiomC{$\Gamma' \Rightarrow (?A)^n,\Delta'$}
        \pravilo{Rez?$_{n}$}
        \BinaryInfC{$\Gamma,\Gamma' \Rightarrow \Delta,\Delta'$}
    \end{prooftree}
\end{definicija}

\begin{opomba}
    Formula $(!A)^n$ v definiciji predstavlja $n$-kratno pojavitev formule $!A$. Pravili Rez!$_{1}$ ter Rez?$_{1}$ sta torej le pravilo Rez, kjer režemo ali formulo $!A$, ali pa formulo $?A$.
\end{opomba}

\begin{lema}
    Pravili Rez!$_n$ ter Rez?$_{n}$ sta dopustni, kar pomeni, da ju lahko izpeljemo iz že definiranih pravil linearne logike.
\end{lema}

\begin{dokaz}
    Lemo dokažemo z indukcijo na številu $n$. Primer pri $n=1$ je seveda le običajno pravilo reza, kot omenjeno že v zgornji opombi. Če predpostavimo, da pravilo Rez!$_n$ že znamo izpeljati, lahko izpeljemo Rez!$_{n+1}$ na naslednji način.
    \begin{prooftree}
        \AxiomC{$\Gamma,(!A)^{n+1} \Rightarrow \Delta$}
        \UnaryInfC{$\Gamma,(!A)^{n-1},!A,!A \Rightarrow \Delta$}
        \levopravilo{C!}
        \UnaryInfC{$\Gamma,(!A)^{n-1},!A \Rightarrow \Delta$}
        \UnaryInfC{$\Gamma,(!A)^n \Rightarrow \Delta$}

        \AxiomC{$\Gamma' \Rightarrow \ !A,\Delta'$}
        \pravilo{Rez!$_n$}
        \BinaryInfC{$\Gamma,\Gamma' \Rightarrow \Delta,\Delta'$}
    \end{prooftree}
    Pri indukcijskem koraku iz $n=1$ na $n=2$ moramo paziti, saj se v drugi vrstici dokaza pojavi izraz $(!A)^0$. To enostavno interpretiramo kot prazno multimnožico formul. Dokaz dopustnosti pravila Rez?$_{n}$ je simetričen.
\end{dokaz}

Če želimo uporabiti zgoraj definirano posplošeno pravilo reza, moramo izrek \ref{izrek}, ki ga dokazujemo, preoblikovati tako, da ga bo vseboval.
\begin{izrek}
    Vsak sekvent, izpeljan z uporabo pravila reza ali posplošenega pravila reza, lahko dokažemo tudi brez uporabe kateregakoli izmed njiju.
\end{izrek}

Zaradi dopustnosti posplošenega reza je ta izrek le posledica izreka \ref{izrek}. A ker novi izrek eliminira tako navadni kot posplošeni rez, je izrek \ref{izrek} prav tako le posledica tega, zato sta si izreka na nek način ekvivalentna. Kar je pomembno za nas je slednje; če dokažemo zgornji izrek, dokažemo tudi izrek \ref{izrek}.

Sedaj si lahko v dokazu pomagamo s posplošenim rezom. To pomeni, da lahko drevo izpeljave preobrazimo tako, da bo namesto prvotnega reza vsebovalo enega ali več rezov \emph{ali posplošenih rezov} nižje stopnje. A to pomeni, da moramo znati poleg navadnega reza sedaj eliminirati še posplošenega. V podpoglavju \ref{gl rez vezniki}, torej pri obravnavi glavnega reza vseh veznikov razen eksponentov, glavnega posplošenega reza niti ne moremo obravnavati. Ta namreč lahko nastopi le, če iz formule režemo eksponente. V nadaljevanju dokaza pa bomo morali biti pazljivi in obdelati še vse primere eliminacije posplošenega reza.

Ostane nam le še definicija stopnje posplošenega reza, saj ta trenutno velja le za navadni rez. V ta namen si oglejmo pravilo Rez!$_n$:
\begin{prooftree}
    \derivation{0}{$\Gamma,(!A)^n \Rightarrow \Delta$}
    \derivation{1}{$\Gamma' \Rightarrow \ !A,\Delta'$}
    \pravilo{Rez!$_n$}
    \BinaryInfC{$\Gamma,\Gamma' \Rightarrow \Delta,\Delta'$}
\end{prooftree}

\begin{definicija}
    \emph{Stopnja posplošenega reza} je par števil $(\R,\h)$, kjer je $\R$ rang formule $!A$ (ali $?A$, glede na vrsto posplošenega reza), $\h$ pa višina drevesa. Stopnja zgornjega posplošenega reza je torej enaka:
    $$
        (\R,\h) = (\R(A) + 1, \h(\D_0) + \h(\D_1)\} + 3)
    $$
\end{definicija}
\begin{opomba}
    Iz definicije je razvidno, da število rezanih formul ne vpliva na stopnjo reza. Če v indukcijskem koraku torej Rez!$_n$ zamenjamo z Rez!$_{n+1}$ na isti višini, režemo pa še zmeraj formulo $!A$, smo stopnjo reza ohranili.
\end{opomba}

Lotimo se še enkrat drevesa izpeljave, kjer na levi $!A$ vpeljemo s skrčitvijo, tokrat s posplošenim pravilom reza v žepu. Obravnavamo lahko kar posplošeni rez za poljuben $n\in\mathbb{N}_{>0}$, vključno z $n = 1$, torej navadnim pravilom reza:
\begin{prooftree}
    \derivation{0}{$\Gamma,(!A)^{n+1} \Rightarrow \Delta$}
    \levopravilo{C!}
    \UnaryInfC{$\Gamma,(!A)^n \Rightarrow \Delta$}

    \derivation{1}{$!\Gamma' \Rightarrow A,?\Delta'$}
    \pravilo{R!}
    \UnaryInfC{$!\Gamma' \Rightarrow \ !A,?\Delta'$}

    \pravilo{Rez!$_n$}
    \BinaryInfC{$\Gamma,!\Gamma' \Rightarrow \Delta,?\Delta'$}
\end{prooftree}
\dol
\begin{prooftree}
    \derivation{0}{$\Gamma,(!A)^{n+1} \Rightarrow \Delta$}

    \derivation{1}{$!\Gamma' \Rightarrow A,?\Delta'$}
    \pravilo{R!}
    \UnaryInfC{$!\Gamma' \Rightarrow \ !A,?\Delta'$}

    \pravilo{Rez!$_{n+1}$}
    \BinaryInfC{$\Gamma,!\Gamma' \Rightarrow \Delta,?\Delta'$}
\end{prooftree}
Zopet se rang formule ohrani, a tokrat je višina novega reza za $1$ nižja od višine prvotnega, torej smo indukcijski predpostavki zadostili. Oglejmo si sedaj glavni rez¸ kjer je formula $!A$ na levi vpeljana z ošibitvijo. Tu korak indukcije za posplošeni rez, kjer $n\neq1$, ter navadni rez ni združljiv, zato primera ločimo, začenši z navadnim pravilom reza:
\begin{prooftree}
    \derivation{0}{$\Gamma \Rightarrow \Delta$}
    \levopravilo{W!}
    \UnaryInfC{$\Gamma,!A \Rightarrow \Delta$}

    \derivation{1}{$!\Gamma' \Rightarrow A,?\Delta'$}
    \pravilo{R!}
    \UnaryInfC{$!\Gamma' \Rightarrow \ !A,?\Delta'$}

    \pravilo{Rez}
    \BinaryInfC{$\Gamma,!\Gamma' \Rightarrow \Delta,?\Delta'$}
\end{prooftree}
\dol
\begin{prooftree}
	\derivation{0}{$\Gamma \Rightarrow \Delta$}
    \levopravilo{W!$\times|\Gamma'|$}
    \UnaryInfC{$\Gamma,!\Gamma' \Rightarrow \Delta$}
    \pravilo{W?$\times|\Delta'|$}
    \UnaryInfC{$\Gamma,!\Gamma' \Rightarrow \Delta,?\Delta'$}
\end{prooftree}
Spet smo na prazno zadostili indukcijski predpostavki in se reza v celoti znebili. Za Rez!$_n$, kjer je $n\geq2$, pa je postopek sledeč:
\begin{prooftree}
    \derivation{0}{$\Gamma,(!A)^n \Rightarrow \Delta$}
    \levopravilo{W!}
    \UnaryInfC{$\Gamma,(!A)^{n+1} \Rightarrow \Delta$}

    \derivation{1}{$!\Gamma' \Rightarrow A,?\Delta'$}
    \pravilo{R!}
    \UnaryInfC{$!\Gamma' \Rightarrow \ !A,?\Delta'$}

    \pravilo{Rez!$_{n+1}$}
    \BinaryInfC{$\Gamma,!\Gamma' \Rightarrow \Delta,?\Delta'$}
\end{prooftree}
\dol
\begin{prooftree}
    \derivation{0}{$\Gamma,(!A)^n \Rightarrow \Delta$}

    \derivation{1}{$!\Gamma' \Rightarrow A,?\Delta'$}
    \pravilo{R!}
    \UnaryInfC{$!\Gamma' \Rightarrow \ !A,?\Delta'$}

    \pravilo{Rez!$_n$}
    \BinaryInfC{$\Gamma,!\Gamma' \Rightarrow \Delta,?\Delta'$}
\end{prooftree}
Stopnja reza je tu znižana na enak način, kot v primeru, ko na levi $!A$ vpeljemo s skrčitvijo. Pri obravnavi glavnega reza formule $!A$ z levim pravilom je zopet potrebno ločiti Rez!$_n$, kjer $n\geq2$, od navadnega reza:
\begin{prooftree}
    \derivation{0}{$\Gamma,A \Rightarrow \Delta$}
    \levopravilo{L!}
    \UnaryInfC{$\Gamma,!A \Rightarrow \Delta$}

    \derivation{1}{$!\Gamma' \Rightarrow A,?\Delta'$}
    \pravilo{R!}
    \UnaryInfC{$!\Gamma' \Rightarrow \ !A,?\Delta'$}

    \pravilo{Rez}
    \BinaryInfC{$\Gamma,!\Gamma' \Rightarrow \Delta,?\Delta'$}
\end{prooftree}
\dol
\begin{prooftree}
    \derivation{0}{$\Gamma,A \Rightarrow \Delta$}
    \derivation{1}{$!\Gamma' \Rightarrow A,?\Delta'$}
    \pravilo{Rez}
    \BinaryInfC{$\Gamma,!\Gamma' \Rightarrow \Delta,?\Delta'$}
\end{prooftree}
Uspelo nam je znižati rang rezane formule, zato je stopnja novega reza nižja. Pri eliminaciji pravila Rez!$_n$, ko je $n\geq2$ se je treba malce bolj potruditi:
\begin{prooftree}
    \derivation{0}{$\Gamma,(!A)^{n-1},A \Rightarrow \Delta$}
    \levopravilo{L!}
    \UnaryInfC{$\Gamma,(!A)^n \Rightarrow \Delta$}

    \derivation{1}{$!\Gamma' \Rightarrow A,?\Delta'$}
    \pravilo{R!}
    \UnaryInfC{$!\Gamma' \Rightarrow \ !A,?\Delta'$}

    \pravilo{Rez!$_n$}
    \BinaryInfC{$\Gamma,!\Gamma' \Rightarrow \Delta,?\Delta'$}
\end{prooftree}
\dol
\begin{prooftree}
    \derivation{0}{$\Gamma,(!A)^{n-1},A \Rightarrow \Delta$}

    \derivation{1}{$!\Gamma' \Rightarrow A,?\Delta'$}
    \pravilo{R!}
    \UnaryInfC{$!\Gamma' \Rightarrow \ !A,?\Delta'$}

    \levopravilo{Rez!$_{n-1}$}
    \BinaryInfC{$\Gamma,!\Gamma',A \Rightarrow \Delta,?\Delta'$}

    \derivation{1}{$!\Gamma' \Rightarrow A,?\Delta'$}
    \pravilo{Rez}
    \BinaryInfC{$\Gamma,!\Gamma',!\Gamma' \Rightarrow \Delta,?\Delta',?\Delta'$}
    \pravilo{C!$\times|\Gamma'|$}
    \UnaryInfC{$\Gamma,!\Gamma' \Rightarrow \Delta,?\Delta',?\Delta'$}
    \pravilo{C?$\times|\Delta'|$}
    \UnaryInfC{$\Gamma,!\Gamma' \Rightarrow \Delta,?\Delta'$}
\end{prooftree}
Novonastalo drevo izpeljave bi nas lahko spominjalo na problem iz začetka tega podpoglavja. Oglejmo si stopnje rezov, kjer prvotni rez zopet označimo z $(\R,\h)$, nova reza pa (po vrsti) z $(\R',\h')$ in $(\R'',\h'')$:
\begin{align*}
    (\R,\h) &= (\R(A) + 1,\h(\D_0) + \h(\D_1) + 5)\\
    (\R',\h') &= (\R(A) + 1,\h(\D_0) + \h(\D_1) + 4)\\
    (\R'',\h'') &= (\R(A),\h(\D_0) + 2*\h(\D_1) + 6)
\end{align*}
Pri prvem izmed novih dreves rang rezane formule ostane isti, vendar se višina zmanjša za $1$, torej je stopnja tega reza res nižja od stopnje prvotnega. Pri drugem rezu spet nastopi težava veliko večje višine, a se je, za razliko problema iz začetka podpoglavja, rang formule znižal. Ker je ureditev leksikografska, se lahko višina poljubno veča; čim je rang rezane formule nižji, bo stopnja reza nižja. Indukcijski predpostavki je torej zadoščeno in ta korak indukcije je opravljen.
