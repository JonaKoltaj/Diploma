Lotimo se sedaj eksponentov. Ker imata oba eksponenta štiri pravila, moramo ločiti glavni rez na več primerov. Pri vezniku ! imamo tri pravila, ki veznik vpeljejo na levi ter eno pravilo, ki veznik vpelje na desni. Zato moramo ločiti glavni rez formule $!A$ na tri primere, po en primer za vsako levo pravilo. Ogljemo si naprej glavni rez, kjer je $!A$ na levi vpeljan s skrčitvijo, na desni pa z desnim pravilom vpeljave.
\begin{prooftree}
    \AxiomC{$\Gamma,!A,!A \Rightarrow \Delta$}
    \levopravilo{C!}
    \UnaryInfC{$\Gamma,!A \Rightarrow \Delta$}

    \AxiomC{$!\Gamma' \Rightarrow A,?\Delta'$}
    \pravilo{R!}
    \UnaryInfC{$!\Gamma' \Rightarrow !A,?\Delta'$}

    \pravilo{Rez}
    \BinaryInfC{$\Gamma,!\Gamma' \Rightarrow \Delta,?\Delta'$}
\end{prooftree}
Mikalo nas bi zgornje drevo izpeljave zamenjati s sledečim.
\begin{prooftree}
    \AxiomC{$\Gamma,!A,!A \Rightarrow \Delta$}

    \AxiomC{$!\Gamma' \Rightarrow A,?\Delta'$}
    \pravilo{R!}
    \UnaryInfC{$!\Gamma' \Rightarrow !A,?\Delta'$}

    \levopravilo{Rez}
    \BinaryInfC{$\Gamma,!\Gamma',!A \Rightarrow \Delta,?\Delta'$}

    \AxiomC{$!\Gamma' \Rightarrow A,?\Delta'$}
    \pravilo{R!}
    \UnaryInfC{$!\Gamma' \Rightarrow !A,?\Delta'$}

    \pravilo{Rez}
    \BinaryInfC{$\Gamma,!\Gamma',!\Gamma' \Rightarrow \Delta,?\Delta',?\Delta'$}
    \pravilo{C!$\times|\Gamma'|$}
    \UnaryInfC{$\Gamma,!\Gamma' \Rightarrow \Delta,?\Delta',?\Delta'$}
    \pravilo{C?$\times|\Delta'|$}
    \UnaryInfC{$\Gamma,!\Gamma' \Rightarrow \Delta,?\Delta'$}
\end{prooftree}
Vendar s tem nismo zmanjšali niti velikosti poddreves nad rezom, saj je spodnji izmed novo pridelanih rezov na isti višini kot prvotni rez, niti kompleksnosti rezane formule, saj ohranimo rezanje formule $!A$. Zavoljo tega, da lahko ta korak indukcije vseeno opravimo, potrebujemo pomožno razširjeno pravilo reza.
\begin{definicija}
    \emph{Posplošeni pravili reza}, označeni z Rez!$_n$ ter Rez?$_{n}$, sta definirani za vsak $n \in \mathbb{N}$.
    \begin{prooftree}
        \AxiomC{$\Gamma,(!A)^n \Rightarrow \Delta$}
        \AxiomC{$\Gamma' \Rightarrow !A,\Delta'$}
        \pravilo{Rez!$_n$}
        \BinaryInfC{$\Gamma,\Gamma' \Rightarrow \Delta,\Delta'$}
    \end{prooftree}
    \begin{prooftree}
        \AxiomC{$\Gamma,?A \Rightarrow \Delta$}
        \AxiomC{$\Gamma' \Rightarrow (?A)^n,\Delta'$}
        \pravilo{Rez?$_{n}$}
        \BinaryInfC{$\Gamma,\Gamma' \Rightarrow \Delta,\Delta'$}
    \end{prooftree}
\end{definicija}

\begin{opomba}
    Formula $(!A)^n$ v definiciji predstavjla $n$ formul $!A$, ločenih z vejico. Pravili Rez!$_{1}$ ter Rez?$_{1}$ sta torej le pravilo Rez, kjer režemo ali formulo $!A$, ali pa formulo $?A$.
\end{opomba}

\begin{lema}
    Pravili Rez!$_n$ ter Rez?$_{n}$ sta dopustni, kar pomeni, da ju lahko izpeljemo iz že definiranih pravil linearne logike.
\end{lema}

\begin{dokaz}
    Lemo dokažemo z indukcijo na številu $n$. Primer ko je $n=1$ je seveda le običajno pravilo reza, kot omenjeno že v zgornji opombi. Če predpostavimo, da pravilo Rez!$_n$ že znamo izpeljati, lahko izpeljemo Rez!$_{n+1}$ na naslednji način.
    \begin{prooftree}
        \AxiomC{$\Gamma,(!A)^{n+1} \Rightarrow \Delta$}
        \UnaryInfC{$\Gamma,(!A)^{n-1},!A,!A \Rightarrow \Delta$}
        \levopravilo{C!}
        \UnaryInfC{$\Gamma,(!A)^{n-1},!A \Rightarrow \Delta$}
        \UnaryInfC{$\Gamma,(!A)^n \Rightarrow \Delta$}

        \AxiomC{$\Gamma' \Rightarrow !A,\Delta'$}
        \pravilo{Rez!$_n$}
        \BinaryInfC{$\Gamma,\Gamma' \Rightarrow \Delta,\Delta'$}
    \end{prooftree}
    Pri indukcijskem koraku iz $n=1$ na $n=2$ moramo paziti, saj se v drugi vrstici dokaza pojavi izraz $(!A)^0$. To enostavno interpretiramo kot prazen nabor formul. Indukcijski korak za Rez?$_{n}$ je simetričen.
\end{dokaz}

Sedaj izrek \ref{izrek}, ki ga dokazujemo, preoblikujmo, tako da bo namesto le navadnega pravila reza vseboval tudi  posplošeno pravilo reza.
\begin{izrek}
    Vsak sekvent, dokazan z uporabo pravila reza ali posplošenega pravila reza, lahko dokažemo tudi brez uporabe pravila reza ali posplošenega pravila reza.
\end{izrek}
%TODO popravi ta godforsaken odstavek
Ker je posplošeni rez izpeljan iz vseh pravil, ki smo jih vpeljali že pred izrekom \ref{izrek}, mu je novo formulirani izrek ekvivalenten. Vendar si sedaj lahko pomagamo s posplošenim pravilom reza in med indukcijskim korakom preobrazimo drevo tako, da bo indukcijski predpostavki zadoščeno za kateregakoli izmed rezov, posplošenega ali navadnega. Vendar, ko rečemo, da želimo drevo izpeljave preobraziti tako da ne vključuje pravila reza, to sedaj seveda pomeni tudi, da ne vključuje posplošenega pravila reza. To pomeni, da je v vsakem koraku indukcije potrebno obravnati tudi eliminacijo posplošenega reza. Seveda pa slednje obravnava le rezanje eksponentov, zato ni relevantno pri primerih koraka indukcije, kjer eksponentov ni, kot na primer pri vseh primerih eliminacije glavnega reza iz podpoglavja \ref{gl rez vezniki}.

Lotimo se še enkrat drevesa izpeljave, kjer na levi $!A$ vpeljemo s skrčitvijo, tokrat s posplošenim pravilom reza v žepu.
\begin{prooftree}
    \AxiomC{$\Gamma,(!A)^{n+1} \Rightarrow \Delta$}
    \levopravilo{C!}
    \UnaryInfC{$\Gamma,(!A)^n \Rightarrow \Delta$}

    \AxiomC{$!\Gamma' \Rightarrow A,?\Delta'$}
    \pravilo{R!}
    \UnaryInfC{$!\Gamma' \Rightarrow !A,?\Delta'$}

    \pravilo{Rez!$_n$}
    \BinaryInfC{$\Gamma,!\Gamma' \Rightarrow \Delta,?\Delta'$}
\end{prooftree}
\dol
\begin{prooftree}
    \AxiomC{$\Gamma,(!A)^{n+1} \Rightarrow \Delta$}

    \AxiomC{$!\Gamma' \Rightarrow A,?\Delta'$}
    \pravilo{R!}
    \UnaryInfC{$!\Gamma' \Rightarrow !A,?\Delta'$}

    \pravilo{Rez!$_{n+1}$}
    \BinaryInfC{$\Gamma,!\Gamma' \Rightarrow \Delta,?\Delta'$}
\end{prooftree}
Zgornje seveda velja tudi za $n=1$, torej za navadno pravilo reza. Pri novem drevesu izpeljave sta sedaj drevesi nad rezom poddrevesi dreves nad rezom pri prejšnjem drevesu izpeljave in korak indukcije je opravljen. Oglejmo si sedaj še glavni rez¸ kjer je formula $!A$ na levi vpeljana z ošibitvijo. Tu korak indukcije za posplošeni, kjer $n\neq1$, ter navadni rez ni združljiv, zato primera ločimo.
\begin{prooftree}
    \AxiomC{$\Gamma \Rightarrow \Delta$}
    \levopravilo{W!}
    \UnaryInfC{$\Gamma,!A \Rightarrow \Delta$}

    \AxiomC{$!\Gamma' \Rightarrow A,?\Delta'$}
    \pravilo{R!}
    \UnaryInfC{$!\Gamma' \Rightarrow !A,?\Delta'$}

    \pravilo{Rez}
    \BinaryInfC{$\Gamma,!\Gamma' \Rightarrow \Delta,?\Delta'$}
\end{prooftree}
\dol
\begin{prooftree}
	\AxiomC{$\Gamma \Rightarrow \Delta$}
    \levopravilo{W!$\times|\Gamma'|$}
    \UnaryInfC{$\Gamma,!\Gamma' \Rightarrow \Delta$}
    \pravilo{W?$\times|\Delta'|$}
    \UnaryInfC{$\Gamma,!\Gamma' \Rightarrow \Delta,?\Delta'$}
\end{prooftree}
Tako smo v tem primeru korak indukcije opravili za navadni rez, za Rez!$_n$, kjer $n\geq2$ pa je postopek sledeč.
\begin{prooftree}
    \AxiomC{$\Gamma,(!A)^n \Rightarrow \Delta$}
    \levopravilo{W!}
    \UnaryInfC{$\Gamma,(!A)^{n+1} \Rightarrow \Delta$}

    \AxiomC{$!\Gamma' \Rightarrow A,?\Delta'$}
    \pravilo{R!}
    \UnaryInfC{$!\Gamma' \Rightarrow !A,?\Delta'$}

    \pravilo{Rez!$_{n+1}$}
    \BinaryInfC{$\Gamma,!\Gamma' \Rightarrow \Delta,?\Delta'$}
\end{prooftree}
\dol
\begin{prooftree}
    \AxiomC{$\Gamma,(!A)^n \Rightarrow \Delta$}

    \AxiomC{$!\Gamma' \Rightarrow A,?\Delta'$}
    \pravilo{R!}
    \UnaryInfC{$!\Gamma' \Rightarrow !A,?\Delta'$}

    \pravilo{Rez!$_n$}
    \BinaryInfC{$\Gamma,!\Gamma' \Rightarrow \Delta,?\Delta'$}
\end{prooftree}
Pri obravnavi glavnega reza formule $!A$ z levim pravilom vpeljave na levi je zopet potrebno ločiti Rez!$_n$ na primer ko je $n=1$ ter ko $n\neq1$. Ogljemo si zopet najprej navadno pravilo reza.
\begin{prooftree}
    \AxiomC{$\Gamma,A \Rightarrow \Delta$}
    \levopravilo{L!}
    \UnaryInfC{$\Gamma,!A \Rightarrow \Delta$}

    \AxiomC{$!\Gamma' \Rightarrow A,?\Delta'$}
    \pravilo{R!}
    \UnaryInfC{$!\Gamma' \Rightarrow !A,?\Delta'$}

    \pravilo{Rez}
    \BinaryInfC{$\Gamma,!\Gamma' \Rightarrow \Delta,?\Delta'$}
\end{prooftree}
\dol
\begin{prooftree}
    \AxiomC{$\Gamma,A \Rightarrow \Delta$}
    \AxiomC{$!\Gamma' \Rightarrow A,?\Delta'$}
    \pravilo{Rez}
    \BinaryInfC{$\Gamma,!\Gamma' \Rightarrow \Delta,?\Delta'$}
\end{prooftree}
%anyway pac na desni je eno nizje aka level je nizji
%TODO dej si od odstavka naprej poglej kako si stvari formulirala in jih spremeni, da bo smiselno kaj delas
