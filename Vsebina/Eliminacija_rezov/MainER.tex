Poslednje pravilo, ki ga je potrebno vpeljati, je \emph{pravilo reza}. To pravilo obstaja tudi v običajnem sekventnem računu in formalizira koncept dokazovanja s pomočjo leme.

\begin{definicija}
	\emph{Pravilo reza} pravi, da če znamo pod določenimi predpostavkami dokazati formulo $A$, potem pa iz te formule dokažemo nekaj drugega, lahko $A$ enostavno režemo iz procesa.
	\begin{prooftree}
        \AxiomC{$\Gamma \Rightarrow A,\Delta$}
        \AxiomC{$\Gamma',A \Rightarrow \Delta'$}
        \pravilo{Rez}
        \BinaryInfC{$\Gamma,\Gamma' \Rightarrow \Delta,\Delta'$}
	\end{prooftree}
\end{definicija}

\begin{opomba}
	Vsa dosedanja pravila v linearni logiki so bila na nek način deterministična. Če smo imeli v sekventu določen veznik, smo lahko uporabili pravilo vpeljave tega veznika, če pa v sekvetnu ni nastopal, tega nismo mogli narediti. Pravilo reza pa je v tem smislu drugačno, saj lahko drevo izpeljave poljubno razvejamo z novimi ,,vrinjenimi'' sekventi.
\end{opomba}
Preden smo v naš sekventni račun vpeljali pravilo reza, je bilo torej dokaj enostavno videti, če sekvent ne velja. Oglejmo si na primer iz podpoglavja \ref{ill}, kjer trdimo, da pricip izključene tretje možnosti ne velja za veznik $\sqcup$, tudi pri klasični linearni logiki.
\begin{center}
    \begin{bprooftree}
        \AxiomC{$\Rightarrow A$}
        \pravilo{R$\sqcup$}
        \UnaryInfC{$\Rightarrow A \sqcup (\negacija A)$}
    \end{bprooftree}
    \begin{bprooftree}
        \AxiomC{$\Rightarrow \negacija A$}
        \pravilo{R$\sqcup$}
        \UnaryInfC{$\Rightarrow A \sqcup (\negacija A)$}
    \end{bprooftree}
\end{center}
Ko dokazujemo sekvent $\Rightarrow A \sqcup (\negacija A)$ brez uporabe pravila reza, imamo na voljo le desno pravilo vpeljave veznika $\sqcup$ in nič drugega. Edina dva možna koraka sta torej prikazana zgoraj, sekvent $\Rightarrow A$ ali $\Rightarrow \negacija A$ pa ne bo veljal za poljubno formulo $A$, torej lahko trdimo, da sekvent, ki ga želimo dokazati, ne velja.

Če pa dopustimo uporabo pravila reza, lahko drevo neskončno razvejamo z vrivanjem poljubnih formul, na primer:
\begin{prooftree}
	\AxiomC{.}
	\noLine
	\UnaryInfC{.}
	\noLine
	\UnaryInfC{.}
	\noLine
	\UnaryInfC{$\Rightarrow B$}

	\AxiomC{.}
	\noLine
	\UnaryInfC{.}
	\noLine
	\UnaryInfC{.}
	\noLine
	\UnaryInfC{$B \Rightarrow A \sqcup (\negacija A)$}

	\pravilo{Rez}
	\BinaryInfC{$\Rightarrow A \sqcup (\negacija A)$}
\end{prooftree}
Drevo izpeljave se lahko razvejuje v neskončnost, zato ne moremo nikoli z gotovostjo trditi, da sekventa ne moremo izpeljati. A če pravilo reza res interpretiramo kot dokaz z uporabo leme, bi bilo smiselno, da če sekventa ne moremo dokazati brez uporabe rezov, ga tudi z rezi ne moremo dokazati. Tu nastopi naslednji izrek.

\begin{izrek}[Izrek o eliminaciji rezov] \label{izrek}
    Vsak sekvent, dokazan z uporabo reza, lahko dokažemo tudi brez uporabe reza.
\end{izrek}

\begin{posledica} %lah pises se o gentzenu etc
    Problem, opisan zgoraj, se torej ne pojavi. Ker sekventa brez reza ne moremo dokazati, ga tudi z rezom ne moremo, ne glede na to, kako razvejamo drevo izpeljave. Izrek nam torej zagotavlja konsistentnost sistema dokazovanja.
\end{posledica}

\subsection{Potrebne definicije} \label{defs}
Dokaz zgornjega izreka poteka s pomočjo indukcije na drevesih izpeljave nad rezom ter na kompleksnosti formul, ki jih režemo. Zavoljo tega moramo tako na drevesih kot formulah vpeljati nekakšno mero, glede na katero bomo potem izvalali proces indukcije. Formula je induktivno definirana v definiciji \ref{formula}, zato je tudi kompleksnost formule definirana induktivno.

\begin{definicija}
    Naj bosta $A$ in $B$ poljubni formuli, $P$ pa naj bo neka osnovna formula. \emph{Rang formule}, označen s simbolom $\R$, je definiran sledeče:
    \begin{itemize}
        \item $\R(P) = \R(\top) = \R(\bot) = \R(\enota) = \R(\nicla) = 1$
        \item $\R(\negacija A) = \R(\forall x A) = \R(\exists x A) = \R(A) + 1$
        \item $\R(A*B) = \max\{\R(A),\R(B)\} + 1$ ; kjer je $*$ poljuben veznik, ki sprejme dve formuli.
    \end{itemize}
\end{definicija}

Podobno je definirana višina drevesa. Imejmo naslednji dve drevesi, levo označeno z $\D$, desno pa z $\D'$:
\begin{center}
    \begin{bprooftree}
        \AxiomC{$\D_0$}
        \AxiomC{$\D_1$}
        \AxiomC{$\ldots$}
        \AxiomC{$\D_n$}
        \pravilo{Pravilo}
        \QuaternaryInfC{$\mathcal{S}$}
    \end{bprooftree}
    \begin{bprooftree}
        \AxiomC{}
        \pravilo{Pravilo'}
        \UnaryInfC{$\mathcal{S}$'}
    \end{bprooftree}
\end{center}
Tu je $n\in\mathbb{N}$, Pravilo je poljubno pravilo, ki sprejme $n$ sekventov, Pravilo' pa je poljubno pravilo, ki jih ne sprejme nič. Slednje je na primer pravilo aksioma ali nekatera izmed pravil za konstante:

\begin{definicija}
    \emph{Višina drevesa izpeljave} je označena s simbolom $\h$ in je enaka:
    \begin{itemize}
        \item$\h(\D) = \max\{\h(\D_0),\h(\D_1),\ldots,\h(\D_n)\} + 1$
        \item$\h(\D') = 1$
    \end{itemize}
\end{definicija}

Sedaj ko imamo mero tako na formulah kot na drevesih izpeljave, lahko defniramo skupno mero vsakega reza. Oglejmo si poljubno pravilo reza:
\begin{prooftree} \label{rez general}
    \AxiomC{$\D_0$}
    \noLine
    \UnaryInfC{$\Gamma,A \Rightarrow \Delta$}

    \AxiomC{$\D_1$}
    \noLine
    \UnaryInfC{$\Gamma' \Rightarrow A,\Delta'$}

    \pravilo{Rez}
    \BinaryInfC{$\Gamma,\Gamma' \Rightarrow \Delta,\Delta'$}
\end{prooftree}
Tu $\D_0$ in $\D_1$ označujeta drevesi izpeljav, ki sta vodili do sekventov pod njima.

\begin{definicija} \label{stopnja}
    \emph{Stopnja reza} je par števil $(\R,\h)$, kjer je $\R$ rang rezane formule, $\h$ pa višina samega drevesa. Z zgornjimi oznakami je stopnja reza torej enaka:
    $$
    (\R,\h) = (\R(A), \max\{\h(\D_0),\h(\D_1)\} + 2)
    $$
    Stopnja reza je urejena \emph{leksikografsko}, kar pomeni, da relacijo < na paru $(\R,\h)$ definiramo sledeče:
    $$
    (\R,\h) < (\R',\h') \Leftrightarrow (\R < \R') \text{ ali } ((\R = \R') \text{ in } (\h < \h'))
    $$
\end{definicija}
\begin{definicija} \label{najnizji}
    Rezu bomo rekli \emph{najnižji rez}, če se nad njim ne pojavi nobeno drugo pravilo reza. To seveda ne pomeni nujno, da ima najnižjo stopnjo ali da je v drevesu izpeljave le en takšen rez.
\end{definicija}

Preden začnemo z dokazom je nazadnje potrebno definirati še eno vrsto reza, in sicer glavni rez.

\begin{definicija} \label{gl rez}
    Če je rezana formula vpeljana v obeh poddrevesih nad pravilom reza, to imenujemo \emph{glavni rez}.
\end{definicija}

\begin{primer*} \label{gl rez in}
    Glavni rez za formulo $A \sqcap B$.
    \begin{prooftree}
        \AxiomC{$\Gamma,A \Rightarrow \Delta$}
        \levopravilo{L$\sqcap$}
        \UnaryInfC{$\Gamma,A \sqcap B \Rightarrow \Delta$}

        \AxiomC{$\Gamma' \Rightarrow A,\Delta'$}
        \AxiomC{$\Gamma' \Rightarrow B,\Delta'$}
        \pravilo{R$\sqcap$}
        \BinaryInfC{$\Gamma' \Rightarrow A \sqcap B,\Delta'$}

        \pravilo{Rez}
        \BinaryInfC{$\Gamma,\Gamma' \Rightarrow \Delta,\Delta'$}
    \end{prooftree}
\end{primer*}


\subsection{Dokaz izreka o eliminaciji rezov}

Razlog, da izrek \ref{izrek} imenujemo izrek o \emph{eliminaciji rezov}, leži v tem, da dokaz poteka s postopno eliminacijo rezov iz poljubnega drevesa izpeljave, dokler na koncu ne dobimo drevesa izpeljave brez kakeršnegakoli pravila reza. Kot smo omenili na začetku podpoglavja \ref{defs}, dokaz poteka z dvojno indukcijo. Zunanja indukcija, je strukturna indukcija na drevesoma izpeljave nad rezom, notranja pa je strukturna indukcija na rezani formuli. Označimo z $\mathcal{D}_0$ ter $\mathcal{D}_1$ drevesi izpeljave nad rezom, torej:
\begin{prooftree}
    \AxiomC{$\mathcal{D}_0$}
    \AxiomC{$\mathcal{D}_1$}
    \pravilo{Rez}
    \BinaryInfC{$\mathcal{S}$}
\end{prooftree}

Začnimo z indukcijskim korakom. Predpostavljamo torej, da iz poddreves $\mathcal{D}_0$ in $\mathcal{D}_1$ znamo eliminirati reze. Da se izognemo problemu, kjer je bilo na primer zadnje pravilo v $\mathcal{D}_0$ ali $\mathcal{D}_1$ rez, predpostavimo, da smo reze tudi že eliminirali iz poddreves. Sedaj dokazujemo, da lahko zgornje drevo izpeljave preobrazimo v drevo izpeljave, kjer se rez pojavi višje v drevesu, torej v poddrevesu tega drevesa. To ustreza indukcijski predpostavki, zato je korak indukcije opravljen.

Znotraj tega koraka, pa seveda delamo tudi indukcijo na strukturi formule. Če torej zgornjega drevesa izpeljave ne uspemo prevesti na drevo, kjer se pravilo reza sedaj pojavi višje, ga lahko prevedemo na rez, ki reže podformulo prvotne formule. Baza notranje indukcije je, da pravilo reza, ki reže osnovno formulo, res prestavimo višje.

\begin{opomba}
    Zgoraj opisano indukcijo si lahko predstavljamo kot algoritem, ki sproti eliminira reze iz drevesa izpeljave. Začnemo z poddrevesom, ki kot zadnje pravilo uporabi rez, v nobenem izmed poddreves pa reza ne uporablja več, torej začnemo z ,,najvišjim'' rezom. Če je takih poddreves več, si arbitrarno izberemo enega izmed njih. V tem poddrevesu postopoma potiskamo rez višje ali pa vsaj znižujemo njegovo kompleksnost, dokler reza ne eliminiramo iz poddrevesa. To ponovimo za vsak rez v drevesu izpeljave. Ker je slednje končno, se proces ustavi in drevo izpeljave ne vključuje več rezov.
\end{opomba}

Vrnimo se h koraku indukcije. Pri obravnavi eliminacije reza je potrebno ločiti, ali je rez, ki ga eliminiramo \emph{glaven}, kot ga opiše definicija \ref{gl rez}, ali pa ni.

\subsubsection{Eliminacija glavnega reza} \label{gl rez vezniki}
Od tu naprej bomo zavoljo preglednosti nad sekventi označevali še drevesa izpeljave, ki so do sekventov vodila.

Oblika glavnega reza je odvisna od vsake rezane formule posebej, zato je njegovo eliminacijo potrebno ločiti glede na veznik, ki rezano formulo sestavlja. Začnimo kar z glavnim rezom veznika $\sqcap$, kot v primeru \ref{gl rez in}:
\begin{prooftree}
    \derivation{0}{$\Gamma,A \Rightarrow \Delta$}
    \levopravilo{L$\sqcap$}
    \UnaryInfC{$\Gamma,A \sqcap B \Rightarrow \Delta$}

    \derivation{1}{$\Gamma' \Rightarrow A,\Delta'$}
    \derivation{2}{$\Gamma' \Rightarrow B,\Delta'$}
    \pravilo{R$\sqcap$}
    \BinaryInfC{$\Gamma' \Rightarrow A \sqcap B,\Delta'$}

    \pravilo{Rez}
    \BinaryInfC{$\Gamma,\Gamma' \Rightarrow \Delta,\Delta'$}
\end{prooftree}
\dol
\begin{prooftree}
    \derivation{0}{$\Gamma,A \Rightarrow \Delta$}
    \derivation{1}{$\Gamma' \Rightarrow A,\Delta'$}
    \pravilo{Rez}
    \BinaryInfC{$\Gamma,\Gamma' \Rightarrow \Delta,\Delta'$}
\end{prooftree}
Sklep pred puščico in po njej je enak, saj iz istih poddreves dokažemo isti sekvent. Tako smo rez stopnje $(\R(A) + \R(B) + 1,\h)$ zamenjali z rezom stopnje $(\R(A),\h')$, ki je očitno manjša. Podobno lahko naredimo za glavni rez $A \star B$:
\begin{prooftree}
    \derivation{0}{$\Gamma,A,B \Rightarrow \Delta$}
    \levopravilo{L$\star$}
    \UnaryInfC{$\Gamma,A \star B \Rightarrow \Delta$}

    \derivation{1}{$\Gamma' \Rightarrow A,\Delta'$}
    \derivation{2}{$\Gamma'' \Rightarrow B,\Delta''$}
    \pravilo{R$\star$}
    \BinaryInfC{$\Gamma',\Gamma'' \Rightarrow A \star B,\Delta',\Delta''$}

    \pravilo{Rez}
    \BinaryInfC{$\Gamma,\Gamma',\Gamma'' \Rightarrow \Delta,\Delta',\Delta''$}
\end{prooftree}
\dol
\begin{prooftree}
    \derivation{0}{$\Gamma,A,B \Rightarrow \Delta$}
    \derivation{1}{$\Gamma' \Rightarrow A,\Delta'$}
    \levopravilo{Rez}
    \BinaryInfC{$\Gamma,\Gamma',B \Rightarrow \Delta,\Delta'$}

    \derivation{2}{$\Gamma'' \Rightarrow B,\Delta''$}
    \pravilo{Rez}
    \BinaryInfC{$\Gamma,\Gamma',\Gamma'' \Rightarrow \Delta,\Delta',\Delta''$}
\end{prooftree}
Tokrat smo prvotno drevo izpeljave prevedli na drevo z dvema rezoma, a imata oba nižjo stopnjo, saj je rang rezane formule očitno nižji. Zelo podobno kot zgornje dva primera izvedemo korak indukcije za $A \sqcup B$ ter $A+B$:
\begin{prooftree}
    \derivation{0}{$\Gamma,A \Rightarrow \Delta$}
    \derivation{1}{$\Gamma,B \Rightarrow \Delta$}
    \levopravilo{L$\sqcup$}
    \BinaryInfC{$\Gamma,A \sqcup B \Rightarrow \Delta$}

    \derivation{2}{$\Gamma' \Rightarrow A,\Delta'$}
    \pravilo{R$\sqcup$}
    \UnaryInfC{$\Gamma' \Rightarrow A \sqcup B,\Delta'$}

    \pravilo{Rez}
    \BinaryInfC{$\Gamma,\Gamma' \Rightarrow \Delta,\Delta'$}
\end{prooftree}
\dol
\begin{prooftree}
    \derivation{0}{$\Gamma,A \Rightarrow \Delta$}
    \derivation{2}{$\Gamma' \Rightarrow A,\Delta'$}
    \pravilo{Rez}
    \BinaryInfC{$\Gamma,\Gamma' \Rightarrow \Delta,\Delta'$}
\end{prooftree}
Kot vidimo je zgornji korak indukcije simetričen koraku indukcije za $A \sqcap B$, saj sta tudi veznika sama simetrična. Enako je korak indukcije za $A+B$ simetričen koraku indukcije za $A \star B$:
\begin{prooftree}
    \derivation{0}{$\Gamma,A \Rightarrow \Delta$}
    \derivation{1}{$\Gamma',B \Rightarrow \Delta'$}
    \levopravilo{L+}
    \BinaryInfC{$\Gamma,\Gamma',A + B \Rightarrow \Delta,\Delta'$}

    \derivation{2}{$\Gamma'' \Rightarrow A,B,\Delta''$}
    \pravilo{R+}
    \UnaryInfC{$\Gamma'' \Rightarrow A + B,\Delta''$}

    \pravilo{Rez}
    \BinaryInfC{$\Gamma,\Gamma',\Gamma'' \Rightarrow \Delta,\Delta',\Delta''$}
\end{prooftree}
\dol
\begin{prooftree}
    \derivation{0}{$\Gamma,A \Rightarrow \Delta$}
    \derivation{2}{$\Gamma'' \Rightarrow A,B,\Delta''$}
    \levopravilo{Rez}
    \BinaryInfC{$\Gamma,\Gamma'' \Rightarrow B,\Delta,\Delta''$}

    \derivation{1}{$\Gamma',B \Rightarrow \Delta'$}
    \pravilo{Rez}
    \BinaryInfC{$\Gamma,\Gamma',\Gamma'' \Rightarrow \Delta,\Delta',\Delta''$}
\end{prooftree}
Naslednji primer, ki ga obravnavamo, je veznik $\multimap$. Zopet se rez prevede na dva reza nižje stopnje, na enak način kot pri veznikih $A\star B$ in $A+B$:
\begin{prooftree}
	\derivation{0}{$\Gamma,A \Rightarrow \Delta$}
    \derivation{1}{$\Gamma',B \Rightarrow \Delta'$}
    \levopravilo{L$\multimap$}
    \BinaryInfC{$\Gamma,\Gamma',A \multimap B \Rightarrow \Delta,\Delta'$}

    \derivation{2}{$\Gamma'',A \Rightarrow B,\Delta''$}
    \pravilo{R$\multimap$}
    \UnaryInfC{$\Gamma'' \Rightarrow A \multimap B,\Delta''$}

    \pravilo{Rez}
    \BinaryInfC{$\Gamma,\Gamma',\Gamma'' \Rightarrow \Delta,\Delta',\Delta''$}
\end{prooftree}
\dol
\begin{prooftree}
	\derivation{0}{$\Gamma,A \Rightarrow \Delta$}
	\derivation{2}{$\Gamma'',A \Rightarrow B,\Delta''$}
    \levopravilo{Rez}
    \BinaryInfC{$\Gamma,\Gamma'' \Rightarrow B,\Delta,\Delta''$}

    \derivation{1}{$\Gamma',B \Rightarrow \Delta'$}
    \pravilo{Rez}
    \BinaryInfC{$\Gamma,\Gamma',\Gamma'' \Rightarrow \Delta,\Delta',\Delta''$}
\end{prooftree}
Poslednji izmed izjavnih veznikov, ki nam ga je potrebno obravnavati je negacija:
\begin{prooftree}
    \derivation{0}{$\Gamma \Rightarrow A,\Delta$}
	\levopravilo{L$\negacija$}
	\UnaryInfC{$\Gamma,\negacija A \Rightarrow \Delta$}

	\derivation{1}{$\Gamma',A \Rightarrow \Delta'$}
	\pravilo{R$\negacija$}
	\UnaryInfC{$\Gamma' \Rightarrow \negacija A,\Delta'$}

	\pravilo{Rez}
	\BinaryInfC{$\Gamma,\Gamma' \Rightarrow \Delta,\Delta'$}
\end{prooftree}
\dol
\begin{prooftree}
    \derivation{0}{$\Gamma \Rightarrow A,\Delta$}
	\derivation{1}{$\Gamma',A \Rightarrow \Delta'$}
	\pravilo{Rez}
	\BinaryInfC{$\Gamma,\Gamma' \Rightarrow \Delta,\Delta'$}
\end{prooftree}
Pri eliminaciji glavnega reza, kjer režemo izjavno konstanto, imamo moč obravnavati le konstanti $\enota$ in $\nicla$, saj za $\top$ levo pravilo ne obstaja, za $\bot$ pa ni desnega. Glavni rez, kjer režemo $\top$ ali $\bot$ se torej ne more zgoditi. Za konstanti $\enota$ in $\nicla$ eliminacija glavnega reza izgleda sledeče:
\begin{center}
    \begin{bprooftree}
        \derivation{}{$\Gamma \Rightarrow \Delta$}
        \levopravilo{L$\enota$}
        \UnaryInfC{$\Gamma,\enota \Rightarrow \Delta$}

        \AxiomC{}
        \pravilo{R$\enota$}
        \UnaryInfC{$ \Rightarrow \enota$}

        \pravilo{Rez}
        \BinaryInfC{$\Gamma \Rightarrow \Delta$}
    \end{bprooftree}
    \begin{bprooftree}
        \AxiomC{}
        \levopravilo{L$\nicla$}
        \UnaryInfC{$\nicla \Rightarrow$}

        \derivation{}{$\Gamma \Rightarrow \Delta$}
        \pravilo{R$\nicla$}
        \UnaryInfC{$\Gamma \Rightarrow \nicla,\Delta$}

        \pravilo{Rez}
        \BinaryInfC{$\Gamma \Rightarrow \Delta$}
    \end{bprooftree}
\end{center}
\begin{align*}
    &\downarrow & &\downarrow
\end{align*}
\begin{center}
    \begin{bprooftree}
        \derivation{}{$\Gamma \Rightarrow \Delta$}
    \end{bprooftree} \qquad \qquad \qquad \quad
    \begin{bprooftree}
        \derivation{}{$\Gamma \Rightarrow \Delta$}
    \end{bprooftree}
\end{center}
Kot lahko vidimo sta primera dokaj trivialna, saj je že sam rez take vrste trivialen. Ker smo rez popolnoma eliminirali je indukcijski predpostavki zadoščeno na prazno. Sedaj si oglejmo eliminacijo glavnega reza obeh kvantifikatorjev. Zopet s $t$ označimo specifičen term¸ z $y$ pa neko (svežo) prosto spremenljivko:
\begin{prooftree}
    \derivation{0}{$\Gamma,A[t/x] \Rightarrow \Delta$}
    \levopravilo{L$\forall$}
    \UnaryInfC{$\Gamma,\forall x A \Rightarrow \Delta$}

    \derivation{1}{$\Gamma' \Rightarrow A[y/x],\Delta'$}
    \pravilo{R$\forall$}
    \UnaryInfC{$\Gamma' \Rightarrow \forall x A,\Delta'$}

    \pravilo{Rez}
    \BinaryInfC{$\Gamma,\Gamma' \Rightarrow \Delta,\Delta'$}
\end{prooftree}
\dol
\begin{prooftree}
    \derivation{0}{$\Gamma,A[t/x] \Rightarrow \Delta$}

    \derivation{1}{$\Gamma' \Rightarrow A[y/x],\Delta'$}
    \pravilo{$y:=t$}
    \UnaryInfC{$\Gamma' \Rightarrow A[t/x],\Delta'$}

    \pravilo{Rez}
    \BinaryInfC{$\Gamma,\Gamma' \Rightarrow \Delta,\Delta'$}
\end{prooftree}
V desnem poddrevesu novonastalega drevesa izpeljave spremenljivko $y$ nadomestimo s specifičnim termom $t$. Ker je bil $y$ \emph{prosta} spremenljivka, to lahko naredimo. Tako dobimo formulo, ki je enaka formuli v levem poddrevesu, in jo zato lahko režemo. V definiciji ranga formule so pomembni le vezniki, ki formulo sestavljajo, ne pa tudi termi, ki se v njej pojavljajo, zato je $\R(A) = \R(A[t/x])$. Dalje je:
$$
\R(\forall x A) = \R(A) + 1 = \R(A[t/x]) + 1
$$
To pomeni, da je stopnja novega reza nižja od stopnje prvega, torej je indukcijski predpostavki zadoščeno. Postopek pri eliminaciji reza eksistenčnega kvantifikatorja je simetričen:
\begin{prooftree}
    \derivation{0}{$\Gamma,A[y/x] \Rightarrow \Delta$}
    \levopravilo{L$\exists$}
    \UnaryInfC{$\Gamma,\exists x A \Rightarrow \Delta$}

    \derivation{1}{$\Gamma' \Rightarrow A[t/x],\Delta'$}
    \pravilo{R$\exists$}
    \UnaryInfC{$\Gamma' \Rightarrow \exists x A,\Delta'$}

    \pravilo{Rez}
    \BinaryInfC{$\Gamma,\Gamma' \Rightarrow \Delta,\Delta'$}
\end{prooftree}
\dol
\begin{prooftree}
    \derivation{0}{$\Gamma,A[y/x] \Rightarrow \Delta$}
    \levopravilo{$y:=t$}
    \UnaryInfC{$\Gamma,A[t/x] \Rightarrow \Delta$}

    \derivation{1}{$\Gamma' \Rightarrow A[t/x],\Delta'$}
    \pravilo{Rez}
    \BinaryInfC{$\Gamma,\Gamma' \Rightarrow \Delta,\Delta'$}
\end{prooftree}


\subsubsection{Glavni rez eksponentov}
Eliminacija glavnega reza eksponentov zahteva posebno obravnavo, saj se dokazovanje tu nekoliko zaplete. Veznika ! in ? sta simetrična, zato bomo podrobno obravnavali le veznik !, bralec pa si lahko sam izpelje dokaze še za veznik ?.

Veznik ! ima štiri logična pravila, ki ga definirajo. Tri pravila veznik vpeljejo na levi strani sekventa, desno pravilo pa ga vpelje na desni. Zato moramo glavni rez formule $!A$ ločiti na tri primere, glede na to kako je bil vpeljan na levi strani. Ogljemo si naprej glavni rez, kjer je $!A$ na levi vpeljan s skrčitvijo:
\begin{prooftree}
    \derivation{0}{$\Gamma,!A,!A \Rightarrow \Delta$}
    \levopravilo{C!}
    \UnaryInfC{$\Gamma,!A \Rightarrow \Delta$}

    \derivation{1}{$!\Gamma' \Rightarrow A,?\Delta'$}
    \pravilo{R!}
    \UnaryInfC{$!\Gamma' \Rightarrow \ !A,?\Delta'$}

    \pravilo{Rez}
    \BinaryInfC{$\Gamma,!\Gamma' \Rightarrow \Delta,?\Delta'$}
\end{prooftree}
Mikalo bi nas zgornje drevo izpeljave zamenjati s sledečim:
\begin{prooftree}
    \derivation{0}{$\Gamma,!A,!A \Rightarrow \Delta$}

    \derivation{1}{$!\Gamma' \Rightarrow A,?\Delta'$}
    \pravilo{R!}
    \UnaryInfC{$!\Gamma' \Rightarrow \ !A,?\Delta'$}

    \levopravilo{Rez}
    \BinaryInfC{$\Gamma,!\Gamma',!A \Rightarrow \Delta,?\Delta'$}

    \derivation{1}{$!\Gamma' \Rightarrow A,?\Delta'$}
    \pravilo{R!}
    \UnaryInfC{$!\Gamma' \Rightarrow \ !A,?\Delta'$}

    \pravilo{Rez}
    \BinaryInfC{$\Gamma,!\Gamma',!\Gamma' \Rightarrow \Delta,?\Delta',?\Delta'$}
    \pravilo{C!$\times|\Gamma'|$}
    \UnaryInfC{$\Gamma,!\Gamma' \Rightarrow \Delta,?\Delta',?\Delta'$}
    \pravilo{C?$\times|\Delta'|$}
    \UnaryInfC{$\Gamma,!\Gamma' \Rightarrow \Delta,?\Delta'$}
\end{prooftree}
Če označimo z $(\R,\h)$ stopnjo prvotnega reza, z $(\R',\h')$ stopjo zgornjega izmed novih rezov, z $(\R'',\h'')$ pa stopnjo spodnjega, velja:
\begin{align*}
    (\R,\h) &= (\R(A) + 1,\h(\D_0) + \h(\D_1) + 5)\\
    (\R',\h') &= (\R(A) + 1,\h(\D_0) + \h(\D_1) + 4)\\
    (\R'',\h'') &= (\R(A) + 1,\h(\D_0) + 2*\h(\D_1) + 7)
\end{align*}
Takoj lahko vidimo, da je rang rezane formule v vseh primerih enak, zato bi bilo potrebno zmanjšati višino. Zgornji izmed novih rezov ima sicer nižjo višino kot prvotni rez, pri spodnjem pa višina znatno naraste. To pomeni, da koraku indukcije ne zadostimo. Da bi lahko ta korak indukcije vseeno opravili, potrebujemo pomožno (razširjeno) pravilo reza.

\begin{definicija}
    \emph{Posplošeni pravili reza}, označeni z Rez!$_n$ in Rez?$_{n}$, sta definirani za vsak $n\in\mathbb{N}_{>0}$;
    \begin{prooftree}
        \AxiomC{$\Gamma,(!A)^n \Rightarrow \Delta$}
        \AxiomC{$\Gamma' \Rightarrow \ !A,\Delta'$}
        \pravilo{Rez!$_n$}
        \BinaryInfC{$\Gamma,\Gamma' \Rightarrow \Delta,\Delta'$}
    \end{prooftree}
    \begin{prooftree}
        \AxiomC{$\Gamma,?A \Rightarrow \Delta$}
        \AxiomC{$\Gamma' \Rightarrow (?A)^n,\Delta'$}
        \pravilo{Rez?$_{n}$}
        \BinaryInfC{$\Gamma,\Gamma' \Rightarrow \Delta,\Delta'$}
    \end{prooftree}
\end{definicija}

\begin{opomba}
    Formula $(!A)^n$ v definiciji predstavlja $n$-kratno pojavitev formule $!A$. Pravili Rez!$_{1}$ ter Rez?$_{1}$ sta torej le pravilo Rez, kjer režemo ali formulo $!A$, ali pa formulo $?A$.
\end{opomba}

\begin{lema}
    Pravili Rez!$_n$ ter Rez?$_{n}$ sta dopustni, kar pomeni, da ju lahko izpeljemo iz že definiranih pravil linearne logike.
\end{lema}

\begin{dokaz}
    Lemo dokažemo z indukcijo na številu $n$. Primer pri $n=1$ je seveda le običajno pravilo reza, kot omenjeno že v zgornji opombi. Če predpostavimo, da pravilo Rez!$_n$ že znamo izpeljati, lahko izpeljemo Rez!$_{n+1}$ na naslednji način.
    \begin{prooftree}
        \AxiomC{$\Gamma,(!A)^{n+1} \Rightarrow \Delta$}
        \UnaryInfC{$\Gamma,(!A)^{n-1},!A,!A \Rightarrow \Delta$}
        \levopravilo{C!}
        \UnaryInfC{$\Gamma,(!A)^{n-1},!A \Rightarrow \Delta$}
        \UnaryInfC{$\Gamma,(!A)^n \Rightarrow \Delta$}

        \AxiomC{$\Gamma' \Rightarrow \ !A,\Delta'$}
        \pravilo{Rez!$_n$}
        \BinaryInfC{$\Gamma,\Gamma' \Rightarrow \Delta,\Delta'$}
    \end{prooftree}
    Pri indukcijskem koraku iz $n=1$ na $n=2$ moramo paziti, saj se v drugi vrstici dokaza pojavi izraz $(!A)^0$. To enostavno interpretiramo kot prazno multimnožico formul. Dokaz dopustnosti pravila Rez?$_{n}$ je simetričen.
\end{dokaz}

Če želimo uporabiti zgoraj definirano posplošeno pravilo reza, moramo izrek \ref{izrek}, ki ga dokazujemo, preoblikovati tako, da ga bo vseboval.
\begin{izrek}
    Vsak sekvent, izpeljan z uporabo pravila reza ali posplošenega pravila reza, lahko dokažemo tudi brez uporabe kateregakoli izmed njiju.
\end{izrek}

Zaradi dopustnosti posplošenega reza je ta izrek le posledica izreka \ref{izrek}. A ker novi izrek eliminira tako navadni kot posplošeni rez, je izrek \ref{izrek} prav tako le posledica tega, zato sta si izreka na nek način ekvivalentna. Kar je pomembno za nas je slednje; če dokažemo zgornji izrek, dokažemo tudi izrek \ref{izrek}.

Sedaj si lahko v dokazu pomagamo s posplošenim rezom. To pomeni, da lahko drevo izpeljave preobrazimo tako, da bo namesto prvotnega reza vsebovalo enega ali več rezov \emph{ali posplošenih rezov} nižje stopnje. A to pomeni, da moramo znati poleg navadnega reza sedaj eliminirati še posplošenega. V podpoglavju \ref{gl rez vezniki}, torej pri obravnavi glavnega reza vseh veznikov razen eksponentov, glavnega posplošenega reza niti ne moremo obravnavati. Ta namreč lahko nastopi le, če iz formule režemo eksponente. V nadaljevanju dokaza pa bomo morali biti pazljivi in obdelati še vse primere eliminacije posplošenega reza.

Ostane nam le še definicija stopnje posplošenega reza, saj ta trenutno velja le za navadni rez. V ta namen si oglejmo pravilo Rez!$_n$:
\begin{prooftree}
    \derivation{0}{$\Gamma,(!A)^n \Rightarrow \Delta$}
    \derivation{1}{$\Gamma' \Rightarrow \ !A,\Delta'$}
    \pravilo{Rez!$_n$}
    \BinaryInfC{$\Gamma,\Gamma' \Rightarrow \Delta,\Delta'$}
\end{prooftree}

\begin{definicija}
    \emph{Stopnja posplošenega reza} je par števil $(\R,\h)$, kjer je $\R$ rang formule $!A$ (ali $?A$, glede na vrsto posplošenega reza), $\h$ pa višina drevesa. Stopnja zgornjega posplošenega reza je torej enaka:
    $$
        (\R,\h) = (\R(A) + 1, \h(\D_0) + \h(\D_1)\} + 3)
    $$
\end{definicija}
\begin{opomba}
    Iz definicije je razvidno, da število rezanih formul ne vpliva na stopnjo reza. Če v indukcijskem koraku torej Rez!$_n$ zamenjamo z Rez!$_{n+1}$ na isti višini, režemo pa še zmeraj formulo $!A$, smo stopnjo reza ohranili.
\end{opomba}

Lotimo se še enkrat drevesa izpeljave, kjer na levi $!A$ vpeljemo s skrčitvijo, tokrat s posplošenim pravilom reza v žepu. Obravnavamo lahko kar posplošeni rez za poljuben $n\in\mathbb{N}_{>0}$, vključno z $n = 1$, torej navadnim pravilom reza:
\begin{prooftree}
    \derivation{0}{$\Gamma,(!A)^{n+1} \Rightarrow \Delta$}
    \levopravilo{C!}
    \UnaryInfC{$\Gamma,(!A)^n \Rightarrow \Delta$}

    \derivation{1}{$!\Gamma' \Rightarrow A,?\Delta'$}
    \pravilo{R!}
    \UnaryInfC{$!\Gamma' \Rightarrow \ !A,?\Delta'$}

    \pravilo{Rez!$_n$}
    \BinaryInfC{$\Gamma,!\Gamma' \Rightarrow \Delta,?\Delta'$}
\end{prooftree}
\dol
\begin{prooftree}
    \derivation{0}{$\Gamma,(!A)^{n+1} \Rightarrow \Delta$}

    \derivation{1}{$!\Gamma' \Rightarrow A,?\Delta'$}
    \pravilo{R!}
    \UnaryInfC{$!\Gamma' \Rightarrow \ !A,?\Delta'$}

    \pravilo{Rez!$_{n+1}$}
    \BinaryInfC{$\Gamma,!\Gamma' \Rightarrow \Delta,?\Delta'$}
\end{prooftree}
Zopet se rang formule ohrani, a tokrat je višina novega reza za $1$ nižja od višine prvotnega, torej smo indukcijski predpostavki zadostili. Oglejmo si sedaj glavni rez¸ kjer je formula $!A$ na levi vpeljana z ošibitvijo. Tu korak indukcije za posplošeni rez, kjer $n\neq1$, ter navadni rez ni združljiv, zato primera ločimo, začenši z navadnim pravilom reza:
\begin{prooftree}
    \derivation{0}{$\Gamma \Rightarrow \Delta$}
    \levopravilo{W!}
    \UnaryInfC{$\Gamma,!A \Rightarrow \Delta$}

    \derivation{1}{$!\Gamma' \Rightarrow A,?\Delta'$}
    \pravilo{R!}
    \UnaryInfC{$!\Gamma' \Rightarrow \ !A,?\Delta'$}

    \pravilo{Rez}
    \BinaryInfC{$\Gamma,!\Gamma' \Rightarrow \Delta,?\Delta'$}
\end{prooftree}
\dol
\begin{prooftree}
	\derivation{0}{$\Gamma \Rightarrow \Delta$}
    \levopravilo{W!$\times|\Gamma'|$}
    \UnaryInfC{$\Gamma,!\Gamma' \Rightarrow \Delta$}
    \pravilo{W?$\times|\Delta'|$}
    \UnaryInfC{$\Gamma,!\Gamma' \Rightarrow \Delta,?\Delta'$}
\end{prooftree}
Spet smo na prazno zadostili indukcijski predpostavki in se reza v celoti znebili. Za Rez!$_n$, kjer je $n\geq2$, pa je postopek sledeč:
\begin{prooftree}
    \derivation{0}{$\Gamma,(!A)^n \Rightarrow \Delta$}
    \levopravilo{W!}
    \UnaryInfC{$\Gamma,(!A)^{n+1} \Rightarrow \Delta$}

    \derivation{1}{$!\Gamma' \Rightarrow A,?\Delta'$}
    \pravilo{R!}
    \UnaryInfC{$!\Gamma' \Rightarrow \ !A,?\Delta'$}

    \pravilo{Rez!$_{n+1}$}
    \BinaryInfC{$\Gamma,!\Gamma' \Rightarrow \Delta,?\Delta'$}
\end{prooftree}
\dol
\begin{prooftree}
    \derivation{0}{$\Gamma,(!A)^n \Rightarrow \Delta$}

    \derivation{1}{$!\Gamma' \Rightarrow A,?\Delta'$}
    \pravilo{R!}
    \UnaryInfC{$!\Gamma' \Rightarrow \ !A,?\Delta'$}

    \pravilo{Rez!$_n$}
    \BinaryInfC{$\Gamma,!\Gamma' \Rightarrow \Delta,?\Delta'$}
\end{prooftree}
Stopnja reza je tu znižana na enak način, kot v primeru, ko na levi $!A$ vpeljemo s skrčitvijo. Pri obravnavi glavnega reza formule $!A$ z levim pravilom je zopet potrebno ločiti Rez!$_n$, kjer $n\geq2$, od navadnega reza:
\begin{prooftree}
    \derivation{0}{$\Gamma,A \Rightarrow \Delta$}
    \levopravilo{L!}
    \UnaryInfC{$\Gamma,!A \Rightarrow \Delta$}

    \derivation{1}{$!\Gamma' \Rightarrow A,?\Delta'$}
    \pravilo{R!}
    \UnaryInfC{$!\Gamma' \Rightarrow \ !A,?\Delta'$}

    \pravilo{Rez}
    \BinaryInfC{$\Gamma,!\Gamma' \Rightarrow \Delta,?\Delta'$}
\end{prooftree}
\dol
\begin{prooftree}
    \derivation{0}{$\Gamma,A \Rightarrow \Delta$}
    \derivation{1}{$!\Gamma' \Rightarrow A,?\Delta'$}
    \pravilo{Rez}
    \BinaryInfC{$\Gamma,!\Gamma' \Rightarrow \Delta,?\Delta'$}
\end{prooftree}
Uspelo nam je znižati rang rezane formule, zato je stopnja novega reza nižja. Pri eliminaciji pravila Rez!$_n$, ko je $n\geq2$ se je treba malce bolj potruditi:
\begin{prooftree}
    \derivation{0}{$\Gamma,(!A)^{n-1},A \Rightarrow \Delta$}
    \levopravilo{L!}
    \UnaryInfC{$\Gamma,(!A)^n \Rightarrow \Delta$}

    \derivation{1}{$!\Gamma' \Rightarrow A,?\Delta'$}
    \pravilo{R!}
    \UnaryInfC{$!\Gamma' \Rightarrow \ !A,?\Delta'$}

    \pravilo{Rez!$_n$}
    \BinaryInfC{$\Gamma,!\Gamma' \Rightarrow \Delta,?\Delta'$}
\end{prooftree}
\dol
\begin{prooftree}
    \derivation{0}{$\Gamma,(!A)^{n-1},A \Rightarrow \Delta$}

    \derivation{1}{$!\Gamma' \Rightarrow A,?\Delta'$}
    \pravilo{R!}
    \UnaryInfC{$!\Gamma' \Rightarrow \ !A,?\Delta'$}

    \levopravilo{Rez!$_{n-1}$}
    \BinaryInfC{$\Gamma,!\Gamma',A \Rightarrow \Delta,?\Delta'$}

    \derivation{1}{$!\Gamma' \Rightarrow A,?\Delta'$}
    \pravilo{Rez}
    \BinaryInfC{$\Gamma,!\Gamma',!\Gamma' \Rightarrow \Delta,?\Delta',?\Delta'$}
    \pravilo{C!$\times|\Gamma'|$}
    \UnaryInfC{$\Gamma,!\Gamma' \Rightarrow \Delta,?\Delta',?\Delta'$}
    \pravilo{C?$\times|\Delta'|$}
    \UnaryInfC{$\Gamma,!\Gamma' \Rightarrow \Delta,?\Delta'$}
\end{prooftree}
Novonastalo drevo izpeljave bi nas lahko spominjalo na problem iz začetka tega podpoglavja. Oglejmo si stopnje rezov, kjer prvotni rez zopet označimo z $(\R,\h)$, nova reza pa (po vrsti) z $(\R',\h')$ in $(\R'',\h'')$:
\begin{align*}
    (\R,\h) &= (\R(A) + 1,\h(\D_0) + \h(\D_1) + 5)\\
    (\R',\h') &= (\R(A) + 1,\h(\D_0) + \h(\D_1) + 4)\\
    (\R'',\h'') &= (\R(A),\h(\D_0) + 2*\h(\D_1) + 6)
\end{align*}
Pri prvem izmed novih dreves rang rezane formule ostane isti, vendar se višina zmanjša za $1$, torej je stopnja tega reza res nižja od stopnje prvotnega. Pri drugem rezu spet nastopi težava veliko večje višine, a se je, za razliko problema iz začetka podpoglavja, rang formule znižal. Ker je ureditev leksikografska, se lahko višina poljubno veča; čim je rang rezane formule nižji, bo stopnja reza nižja. Indukcijski predpostavki je torej zadoščeno in ta korak indukcije je opravljen.


% Dej bazo na konc, i think da leps pase
% Baza indukcije na $\mathcal{D}_0$ in $\mathcal{D}_1$ pomeni, da je vsaj eden izmed $\mathcal{D}_0$ ter $\mathcal{D}_1$ list, torej pravilo aksioma, ter da tisto poddrevo, ki ni list, ne vsebuje pravila rez. Potrebno je dokazati, da lahko take vrste drevo izpeljave preobrazimo v drevo izpeljave brez rezov. Oglejmo si torej primer, ko je $\mathcal{D}_0$ list, $\mathcal{D}_1$ pa ni. Obratno je seveda simetrično.
%
% \begin{center}
%     \begin{bprooftree}
%         \AxiomC{}
%         \pravilo{Ax}
%         \UnaryInfC{$A \Rightarrow A$}
%
%         \AxiomC{$\Gamma,A \Rightarrow \Delta$}
%         \pravilo{Rez}
%         \BinaryInfC{$\Gamma,A \Rightarrow \Delta$}
%     \end{bprooftree}\qquad
%     $\rightarrow$ \qquad
%     \begin{bprooftree}
%         \AxiomC{$\Gamma,A \Rightarrow \Delta$}
%     \end{bprooftree}
% \end{center}
% Kot lahko vidimo,

% TODO baza notranje indukcije (aka osnovno formulo lah gor fuknemo, figure it out)
% TODO baza indukcije za Cut!?
