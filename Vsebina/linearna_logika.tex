Linearna logika je podzvrst logike sekventnega računa, kjer zavržemo pravili ošibitve in skrčitve iz definicij \ref{weakening} in \ref{contraction}. To pomeni, da moramo vsako predpostavko uporabiti natanko enkrat ter da ne smemo imeti odvečnih predpostavk. Prav tako moramo vsak sklep dokazati natanko enkrat, brez odvečnih sklepov.

Za primer si oglejmo še eno možno definicijo veznika $\land$, poleg \ref{inl} ter \ref{inr} , označimo ga v tem primeru z $\land'$.
\begin{definicija} \label{in'l}
    \emph{Levo pravilo vpeljave veznika $\land'$} pravi, če želimo iz $A \land' B$ dokazati $\Delta$, lahko veznik na levi enostavno prevedemo nazaj v vejico in iz $A$ ter $B$ dokazujemo $\Delta$.
    \begin{prooftree}
        \AxiomC{$\Gamma,A,B \Rightarrow \Delta$}
        \pravilo{L$\land'$}
        \UnaryInfC{$\Gamma,A \land' B \Rightarrow \Delta$}
    \end{prooftree}
\end{definicija}

\begin{definicija} \label{in'r}
	\emph{Desno pravilo vpeljave veznika $\land'$} pa pravi, da če želimo $A \land' B$ dokazati, lahko predpostavke (in preostale sklepe, ki niso povezani z $A$ in $B$) ločimo na dva dela ter z enim delom dokažemo $A$, z drugim pa $B$. Obratno gledano, če znamo iz $\Gamma$ dokazati $A$ ter iz $\Gamma'$ dokazati $B$, lahko predpostavke združimo in dokažemo $A \land' B$.
    \begin{prooftree}
        \AxiomC{$\Gamma \Rightarrow A,\Delta$}
        \AxiomC{$\Gamma' \Rightarrow B,\Delta'$}
        \pravilo{R$\land'$}
        \BinaryInf$\Gamma,\Gamma' \fCenter A \land' B,\Delta,\Delta'$
    \end{prooftree}
\end{definicija}

\begin{lema}
    Če dopustimo uporabo ošibitve in skrčitve, sta si levi in desni pravili vpeljave za $\land$ ter $\land'$ ekvivalentni.
\end{lema}
\begin{dokaz}
    Začnimo z dokazom ekvivalence levih pravil za veznika $\land$ ter $\land'$. Dokaz ekvivalence v tem kontekstu pomeni, da sta pravili medsebojno izpeljivi.
    Izpeljava levega pravila za $\land'$ iz definicije \ref{in'l} na podlagi levega pravila za $\land$ iz definicije \ref{inl} poteka s pomočjo skrčitve.
    \begin{prooftree}
        \AxiomC{$\Gamma,A,B \Rightarrow \Delta$}
        \pravilo{L$\land$}
        \UnaryInfC{$\Gamma,A \land B,B \Rightarrow \Delta$}
        \pravilo{L$\land$}
        \UnaryInfC{$\Gamma,A \land B, A \land B \Rightarrow \Delta$}
        \pravilo{C}
        \UnaryInfC{$\Gamma,A \land B \Rightarrow \Delta$}
    \end{prooftree}
    Najprej torej predpostavko $A \land B$ ,,podvojimo'', nato pa dvakrat uporabimo levo pravilo za $\land$, vsakič ne eni izmed podvojenih predpostavk.

    Obratna izpeljava pa poteka s pomočjo ošibitve. Tu najprej uporabimo levo pravilo za $\land'$, nato pa odvečno izmed predpostavk odstranimo s pomočjo ošibitve.
    \begin{prooftree}
        \AxiomC{$\Gamma,A \Rightarrow \Delta$}
        \pravilo{W}
        \UnaryInfC{$\Gamma,A,B \Rightarrow \Delta$}
        \pravilo{L$\land'$}
        \UnaryInfC{$\Gamma,A \land' B \Rightarrow \Delta$}
    \end{prooftree}

    Podobno dokažemo ekvivalenco desnega pravila za $\land$ iz definicije \ref{inr} ter desnega pravila za $\land'$ iz definicije \ref{in'r}.
    \begin{prooftree}
        \AxiomC{$\Gamma \Rightarrow A,\Delta$}
        \levopravilo{W$\times |\Gamma' \cup \Delta'|$}
        \UnaryInfC{$\Gamma,\Gamma' \Rightarrow A,\Delta,\Delta'$}

        \AxiomC{$\Gamma' \Rightarrow B,\Delta'$}
        \pravilo{W$\times |\Gamma \cup \Delta|$}
        \UnaryInfC{$\Gamma,\Gamma' \Rightarrow B,\Delta,\Delta'$}

        \pravilo{R$\land$}
        \BinaryInf$\Gamma,\Gamma' \fCenter A \land B,\Delta,\Delta'$
    \end{prooftree}
    Ko zgoraj iz desnega pravila za $\land$ izpeljujemo desno pravilo za $\land'$, najprej uporabimo desno pravilo za $\land$, torej predpostavk (in sklepov) ne razdelimo na dva dela, zato se s pomočjo ošibitve na levi ,,znebimo'' (saj pravila beremo od spodaj navzgor) predpostavk $\Gamma'$ in sklepov $\Delta'$. To naredimo tako, da ošibitev iteriramo tolikokrat, kolikor je velikost multimnožice $\Gamma'\cup\Delta'$. Podobno naredimo na desni strani.

    Ko pa iz desnega pravila za $\land'$ izpeljujemo desno pravilo za $\land$, najprej ,,podvojimo'' vse predpostavke v $\Gamma$ in vse sklepe v $\Delta$, nato pa uporabimo desno pravilo za $\land'$ in podvojene predpostavke spet razpolovimo.
    \begin{prooftree}
        \AxiomC{$\Gamma \Rightarrow A,\Delta$}
        \AxiomC{$\Gamma \Rightarrow B,\Delta$}
        \pravilo{R$\land'$}
        \BinaryInf$\Gamma, \Gamma \fCenter A \land' B, \Delta, \Delta$
        \pravilo{C$\times |\Gamma \cup \Delta|$}
        \UnaryInfC{$\Gamma \Rightarrow A \land' B, \Delta$}
    \end{prooftree}
\end{dokaz}
\begin{opomba}
	V zgornjem dokazu smo se malce podrobneje spustili v intuicijo posameznega dela dokaza, saj je to prvi formalen dokaz v sekventnem računu v tem delu. V nadaljnem je intuicija za posamezne vrstice dokaza načeloma prepuščena bralcu.
\end{opomba}

Kot smo v zgornjem dokazu lahko videli, se za dokaz ekvivalence $\land$ ter $\land'$ na bistven način uporabi tako skrčitev kot ošibitev. Slutimo lahko, da brez teh dveh pravil veznika pravzaprav nista ekvivalentna, kar se tudi izkaže za resnično ??cite??. Zato sta sta ta dva veznika v linearni logiki dva različna veznika

\subsection{Propozicijski vezniki}

\begin{definicija} \label{in}
    Veznik $\land$, s pravili iz definicij \ref{inl} in \ref{inr}, se še vedno glasi \emph{in}, zapišemo pa ga s simbolom $\sqcap$. Zapišimo še enkrat njegovo levo in desno pravilo, tokrat s pravilnim pojmovanjem.
    \begin{center}
        \begin{bprooftree}
            \AxiomC{$\Gamma, A \Rightarrow \Delta$}
            \pravilo{L$\sqcap$}
            \UnaryInfC{$\Gamma,A \sqcap B \Rightarrow \Delta$}
        \end{bprooftree}
        \begin{bprooftree}
            \AxiomC{$\Gamma, B \Rightarrow \Delta$}
            \pravilo{L$\sqcap$}
            \UnaryInfC{$\Gamma,A \sqcap B \Rightarrow \Delta$}
        \end{bprooftree}
        \begin{bprooftree}
            \AxiomC{$\Gamma \Rightarrow A,\Delta$}
            \AxiomC{$\Gamma \Rightarrow B,\Delta$}
            \pravilo{R$\sqcap$}
            \BinaryInf$\Gamma \fCenter A \sqcap B,\Delta$
        \end{bprooftree}
    \end{center}
\end{definicija}

\begin{definicija} \label{tenzor}
    Veznik $\land'$, s pravili iz definicij \ref{in'l} in \ref{in'r} pa preimenujemo v \emph{tenzor} ter ga zapišemo s simbolom $\star$.
    \begin{center}
        \begin{bprooftree}
            \AxiomC{$\Gamma,A,B \Rightarrow \Delta$}
            \pravilo{L$\star$}
            \UnaryInfC{$\Gamma,A \star B \Rightarrow \Delta$}
        \end{bprooftree}
        \begin{bprooftree}
            \AxiomC{$\Gamma \Rightarrow A,\Delta$}
            \AxiomC{$\Gamma' \Rightarrow B,\Delta'$}
            \pravilo{R$\star$}
            \BinaryInf$\Gamma,\Gamma' \fCenter A \star B,\Delta,\Delta'$
        \end{bprooftree}
    \end{center}
\end{definicija}

Zakaj te dva veznika v kontekstu linearne logike nista enaka je razvidno že če primerjamo njuni levi in desni pravili. Kot smo omenili na začetku tega poglavja je pomembno, da vsako predpostavko uporabimo natanko enkrat. Veznik $\sqcap$ med predpostavkami na nek način vsebuje le eno izmed predpostavk, ki ju združuje, medtem ko veznik $\star$ vsebuje obe. Ko torej uporabimo $A \sqcap B$¸ da dokažemo neki $\Delta$, uporabimo le $A$ ali $B$, medtem ko pri $A \star B$ uporabimo tako $A$ kot $B$. Če pa želimo dokazati, da velja $A \sqcap B$, pa je dovolj, da iz istih predpostavk dokažemo $A$ ter $B$, prav tako ostale sklepe na desni strani sekventa pustimo pri miru. To spet implicira, da vsebuje $A\sqcap B$ enako število informacij kot le $A$ ali $B$. Če pa dokazujemo $A\star B$, pa moramo posebej dokazati $A$, nato pa iz ločenega sklopa predpostavk dokazati $B$. Ostale sklepe poleg $A\star B$ je tudi potrebno posebej dokazati. Vse to spet implicira, da vsebuje tenzor informacij tako za $A$ kot $B$.

Oglejmo si sedaj še preostale veznike, začenši z veznikom $\lor$. V linearni logiki se ta spet razdeli na dvoje, intuicija za to pa je simetična intuiciji za veznik $\land$, zato jo prepustimo bralcu.

\begin{definicija} \label{ali}
	Veznik \emph{ali}, označen z $\sqcup$, ima sledeče levo in desno pravilo vpeljave.
	\begin{center}
        \begin{bprooftree}
            \AxiomC{$\Gamma,A \Rightarrow \Delta$}
            \AxiomC{$\Gamma,B \Rightarrow \Delta$}
            \pravilo{L$\sqcup$}
            \BinaryInf$\Gamma,A \sqcup B \fCenter \Delta$
        \end{bprooftree}
        \begin{bprooftree}
            \AxiomC{$\Gamma \Rightarrow A,\Delta$}
            \pravilo{R$\sqcup$}
            \UnaryInfC{$\Gamma \Rightarrow A \sqcup B,\Delta$}
        \end{bprooftree}
        \begin{bprooftree}
            \AxiomC{$\Gamma \Rightarrow B, \Delta$}
            \pravilo{R$\sqcup$}
            \UnaryInfC{$\Gamma \Rightarrow A \sqcup B,\Delta$}
        \end{bprooftree}
    \end{center}
    Kot vidimo sta obe pravili popolnoma simetrični praviloma za veznik $\sqcap$.
\end{definicija}

\begin{definicija} \label{plus}
	Veznik \emph{plus}, označen z + pa je analogno simetričen vezniku $\star$.
	\begin{center}
        \begin{bprooftree}
            \AxiomC{$\Gamma,A \Rightarrow \Delta$}
            \AxiomC{$\Gamma',B \Rightarrow \Delta'$}
            \pravilo{L+}
            \BinaryInf$\Gamma,\Gamma',A + B \fCenter \Delta,\Delta'$
        \end{bprooftree}
        \begin{bprooftree}
            \AxiomC{$\Gamma \Rightarrow A,B,\Delta$}
            \pravilo{R+}
            \UnaryInfC{$\Gamma \Rightarrow A + B,\Delta$}
        \end{bprooftree}
    \end{center}
\end{definicija}

Vsi nadaljni vezniki imajo v linearni logiki enaka pravila vpeljave kot v navadnem sekventnem računu in se ne delijo na dva dela, še vseeno pa so to \emph{linearni} vezniki, že samo zaradi pogojev pod katerimi so vpeljani. Če na primer $A$ linearno implicira $B$ to pomeni, da natanko en $A$ implicira natanko en $B$, linearna negacija formule $A$ pa negira natanko en $A$.

Za vpeljavo implikacije zopet potrebujemo drugačen simbol kot smo ga vajeni, saj se $\Rightarrow$ že uporablja v strukturi sekventa samega. Običajni sekventni račun v ta namen uporablja $\rightarrow$, linearna implikacija pa, da se loči od nelinearne, spet uporabi svoj simbol.

\begin{definicija}
	\emph{Implikacija}, označena s simbolom $\multimap$, je vpeljana z naslednjimi pravili.
    \begin{center}
        \begin{bprooftree}
            \AxiomC{$\Gamma \Rightarrow A,\Delta$}
            \AxiomC{$\Gamma',B \Rightarrow \Delta'$}
            \pravilo{L$\multimap$}
            \BinaryInf$\Gamma,\Gamma',A \multimap B \fCenter \Delta,\Delta'$
        \end{bprooftree}
        \begin{bprooftree}
            \AxiomC{$\Gamma,A \Rightarrow B,\Delta$}
            \pravilo{R$\multimap$}
            \UnaryInfC{$\Gamma \Rightarrow A \multimap B,\Delta$}
        \end{bprooftree}
    \end{center}
    Kot lahko vidimo je desno pravilo vpeljave dokaj jasno za interpretacijo. Če dokazujemo $A \multimap B$, je dovolj da pod predpostavko $A$ dokažemo $B$. Levo pravilo pa je morda lažje brati od zgoraj navzdol. Če torej z $\Gamma$ dokažemo $A$ ter neke druge sklepe $\Delta$, iz $\Gamma'$ in $B$ pa dokažemo $\Delta'$, lahko iz združenih predpostavk $\Gamma,\Gamma'$ ter dejstva, da iz $A$ sledi $B$ dokažemo združene sklepe $\Delta,\Delta'$.
\end{definicija}

\begin{definicija}
    \emph{Negacija}, označena s simbolom $\negacija$, ima naslednji pravili vpeljave.
    \begin{center}
        \begin{bprooftree}
            \AxiomC{$\Gamma \Rightarrow A,\Delta$}
            \pravilo{L$\negacija$}
            \UnaryInfC{$\Gamma,\negacija A \Rightarrow \Delta$}
        \end{bprooftree}
        \begin{bprooftree}
            \AxiomC{$\Gamma,A \Rightarrow \Delta$}
            \pravilo{R$\negacija$}
            \UnaryInfC{$\Gamma \Rightarrow \negacija A,\Delta$}
        \end{bprooftree}
    \end{center}
    ?? Kaj je tuki sploh intuicija lmao??
\end{definicija}

\subsection{Propozicijske konstante}

V običajnem sekventnem računu imamo dvoje konstant; resnico in neresnico, ki pa se v linearni logiki spet vsaka razdelita na dvoje. Resnici se delita na enoto za $\star$ ter enoto za $\sqcap$, neresnici pa na enoto za + ter enoto za $\sqcup$.

\begin{definicija}
    \emph{Enota}, označena z $\enota$, ima levo in desno pravilo vpeljave:
    \begin{center}
        \begin{bprooftree}
            \AxiomC{$\Gamma \Rightarrow \Delta$}
            \pravilo{L$\enota$}
            \UnaryInfC{$\Gamma,\enota \Rightarrow \Delta$}
        \end{bprooftree}
        \begin{bprooftree}
            \AxiomC{}
            \pravilo{R$\enota$}
            \UnaryInfC{$\Rightarrow \enota$}
        \end{bprooftree}
    \end{center}
    Enoto torej lahko brez predpostavk vedno dokažemo, kar nam pove desno pravilo, če pa vemo da enota velja je to trivialna informacija, kar nam pove levo pravilo.
\end{definicija}

\begin{definicija}
    \emph{Resnica}, označena z $\top$, ima le desno pravilo vpeljave. Ne moremo je torej uporabiti kot sklep.
    \begin{prooftree}
        \AxiomC{}
        \pravilo{R$\top$}
        \UnaryInfC{$\Gamma \Rightarrow \top,\Delta$}
    \end{prooftree}
    Kar nam to pove je, da resnica vedno velja.
\end{definicija}

\begin{lema} \label{enoti}
	Enota $\enota$ je enota za za $\star$, resnica $\top$ pa je enota za $\sqcap$.
\end{lema}
\begin{dokaz}
    Za dokaz leme potrebujemo izpeljati sekvente $A \star \enota \Rightarrow A$, $A \Rightarrow A \star \enota$, $A \sqcap \top \Rightarrow A$ ter $A \Rightarrow A \sqcap \top$.
    \begin{center}
        \vskip 10pt
        \begin{bprooftree}
            \AxiomC{}
            \pravilo{Ax}
            \UnaryInfC{$A \Rightarrow A$}
            \pravilo{L$\enota$}
            \UnaryInfC{$A,\enota \Rightarrow A$}
            \pravilo{L$\star$}
            \UnaryInfC{$A \star \enota \Rightarrow A$}
        \end{bprooftree}
        \begin{bprooftree}
            \AxiomC{}
            \levopravilo{Ax}
            \UnaryInfC{$A \Rightarrow A$}

            \AxiomC{}
            \pravilo{R$\enota$}
            \UnaryInfC{$\Rightarrow \enota$}

            \pravilo{R$\star$}
            \BinaryInf$A \fCenter A \star \enota$
        \end{bprooftree}
    \end{center}
    \vskip 10pt
    \begin{center}
        \begin{bprooftree}
            \AxiomC{}
            \pravilo{Ax}
            \UnaryInfC{$A \Rightarrow A$}
            \pravilo{L$\sqcap$}
            \UnaryInfC{$A \sqcap \top \Rightarrow A$}
        \end{bprooftree}
        \begin{bprooftree}
            \AxiomC{}
            \levopravilo{Ax}
            \UnaryInfC{$A \Rightarrow A$}

            \AxiomC{}
            \pravilo{R$\top$}
            \UnaryInfC{$A \Rightarrow \top$}

            \pravilo{R$\sqcap$}
            \BinaryInf$A \fCenter A \sqcap \top$
        \end{bprooftree}
    \end{center}
\end{dokaz}

Pri neresnici je razlog za razdvojitev enak, pravila vpeljave pa so simetrična, zato interpretacijo prepustimo bralcu.

\begin{definicija}
	\emph{Ničla}, označena z $\nicla$ ima levo in desno pravilo vpeljave:
	 \begin{center}
        \begin{bprooftree}
            \AxiomC{}
            \pravilo{L$\nicla$}
            \UnaryInfC{$\nicla \Rightarrow$}
        \end{bprooftree}
        \begin{bprooftree}
            \AxiomC{$\Gamma \Rightarrow \Delta$}
            \pravilo{R$\nicla$}
            \UnaryInfC{$\Gamma \Rightarrow \nicla,\Delta$}
        \end{bprooftree}
    \end{center}
\end{definicija}
\begin{definicija}
    \emph{Neresnica}, označena z $\bot$, ima le levo pravilo vpeljave. Ne moremo je torej uporabiti kot predpostavko.
    \begin{prooftree}
        \AxiomC{}
        \pravilo{L$\bot$}
        \UnaryInfC{$\Gamma, \bot \Rightarrow \Delta$}
    \end{prooftree}
\end{definicija}

Dokaz naslednje leme bomo opustili, saj je simetričen dokazu leme \ref{enoti}.
\begin{lema}
	Ničla $\nicla$ je enota za +, neresnica $\bot$ pa je enota za $\sqcup$.
\end{lema}

\subsection{Kvantifikatorja}

\begin{definicija}
	\emph{Univerzalni kvantifikator}, označen kot navadno s simbolom $\forall$, je definiran z naslednjima praviloma vpeljave. Tu $y$ ne sme biti prost v $\Gamma$ in $\Delta$.
	%notacija tuki bo t/x means substitute vsak x s tjem
	%also citiras simpsona as well
	\begin{center}
        \begin{bprooftree}
            \AxiomC{$\Gamma, A[t/x] \Rightarrow \Delta$}
            \pravilo{L$\forall$}
            \UnaryInfC{$\Gamma,\forall x A \Rightarrow \Delta$}
        \end{bprooftree}
        \begin{bprooftree}
            \AxiomC{$\Gamma \Rightarrow A[y/x],\Delta$}
            \pravilo{R$\forall$}
            \UnaryInfC{$\Gamma \Rightarrow \forall x A,\Delta$}
        \end{bprooftree}
	\end{center}
	Notacija $A[y/x]$ pomeni, da vsako instanco spremenljivke $x$ zamenjamo s spremenljivko $y$. Spremenljivka $t$ v definiciji označuje nek specifičen term $t$, ki si ga izberemo. Levo pravilo vpeljave torej pomeni, da če želimo iz dejstva, da za vsak $x$ velja formula $A$ dokazati $\Delta$, je dovolj, da spremenljivko $x$ v $A$ zamenjamo z nekim specifičnim termom in z njim dokažemo $\Delta$. Spremenljivka $y$ v definiciji pa označuje prosto spremenljivko. Desno pravilo vpeljave je ekvivalentno temu, da pri dokazovanju, da za vsak $x$ velja $A$, fiksiramo poljuben $y$ in dokazujemo $A$.
\end{definicija}

\begin{definicija}
    \emph{Eksistenčni kvantifikator} je spet brez sprememb označen s simbolom $\exists$. Spremenljivka $y$ spet ne sme  biti prosta v $\Gamma$ ter $\Delta$.
    \begin{center}
        \begin{bprooftree}
            \AxiomC{$\Gamma,A[y/x] \Rightarrow \Delta$}
            \pravilo{L$\exists$}
            \UnaryInfC{$\Gamma,\exists x A \Rightarrow \Delta$}
        \end{bprooftree}
        \begin{bprooftree}
            \AxiomC{$\Gamma \Rightarrow A[t/x],\Delta$}
            \pravilo{R$\exists$}
            \UnaryInfC{$\Gamma \Rightarrow \exists x A,\Delta$}
        \end{bprooftree}
	\end{center}
	Tokrat levo pravilo vsebuje prosto spremenjlivko $y$, desno pa specifičen term $t$. Če torej želimo uporabiti dejstvo, da obstaja $x$, da velja $A$, fiksiramo poljuben $y$ in dokazujemo $A$, če pa želimo dokazati, da obstaja $x$, da velja $A$, le poščemo nek specifičen term $t$, da $A$ velja.
\end{definicija}

\subsection{Eksponenta}

Včasih si želimo v linearni logiki emulirati tudi običajen sekventni račun, vključno torej z ošibitvijo ter skrčitvijo, a želimo to nelinearnost omejiti na specifične formule. Namen teh ,,nelinearnih'' formul je, da nam dovolijo v linearni logiki dokazati vse, kar je možno dokazati tudi v običajnem sekventnem računu, a da je iz dokaza takoj razvidno, kateri deli sekventa so bili dokazani linearno in kateri ne. Ker želimo označiti formulo kot nelinearno, jo modificiramo v novo formulo z veznikom. Imamo dva takšna veznika, ki ju imenujemo \emph{eksponenta}.

\begin{definicija}
    Veznik \emph{seveda} je označen s simbolom !, poleg levega in desnega pravila vpeljave pa zanj veljata še ošibitev in skrčitev na levi strani sekventa. Veznik \emph{zakaj ne} pa je označen s simbolom ?, zanj pa prav tako veljajo štiri pravila; levo in desno pravilo vpeljave ter skrčitev in ošibitev na desni strani sekventa.
    \begin{center}
        \begin{bprooftree}
            \AxiomC{$\Gamma,A \Rightarrow \Delta$}
            \pravilo{L!}
            \UnaryInfC{$\Gamma,!A \Rightarrow \Delta$}
        \end{bprooftree}
        \begin{bprooftree}
            \AxiomC{$!\Gamma \Rightarrow A,?\Delta$}
            \pravilo{R!}
            \UnaryInfC{$!\Gamma \Rightarrow \ !A,?\Delta$}
        \end{bprooftree}
        \begin{bprooftree}
            \AxiomC{$\Gamma \Rightarrow \Delta$}
            \pravilo{W!}
            \UnaryInfC{$\Gamma,!A \Rightarrow \Delta$}
        \end{bprooftree}
        \begin{bprooftree}
            \AxiomC{$\Gamma,!A,!A \Rightarrow \Delta$}
            \pravilo{C!}
            \UnaryInfC{$\Gamma,!A \Rightarrow \Delta$}
        \end{bprooftree}
    \end{center}
    \begin{center}
        \begin{bprooftree}
            \AxiomC{$!\Gamma,A \Rightarrow ?\Delta$}
            \pravilo{L?}
            \UnaryInfC{$!\Gamma,?A \Rightarrow ?\Delta$}
        \end{bprooftree}
        \begin{bprooftree}
            \AxiomC{$\Gamma \Rightarrow A,\Delta$}
            \pravilo{R?}
            \UnaryInfC{$\Gamma \Rightarrow \ !A,\Delta$}
        \end{bprooftree}
        \begin{bprooftree}
            \AxiomC{$\Gamma \Rightarrow A,\Delta$}
            \pravilo{W?}
            \UnaryInfC{$\Gamma \Rightarrow \ ?A,\Delta$}
        \end{bprooftree}
        \begin{bprooftree}
            \AxiomC{$\Gamma \Rightarrow \ ?A,?A,\Delta$}
            \pravilo{C?}
            \UnaryInfC{$\Gamma \Rightarrow \ ?A,\Delta$}
        \end{bprooftree}
    \end{center}
    Notacija $!\Gamma$ (ali $?\Delta$) označuje, da je vsaka formula v $\Gamma$ (ali $\Delta$) predznačena z veznikom ! (ali ?).
\end{definicija}

Kar želimo z eksponenti označiti je, da imamo ,,poljubno mnogo'' označene formule na voljo. Ker vejice na levi beremo kot ,,in'', vejice na desni pa kot ,,ali'', je treba podati dva različna veznika. Formula $!A$ torej označuje ,,$A$ in $A$ in $\ldots$ $A$'', kolikor kopij pač potrebujemo, formula $?A$ pa označuje ,,$A$ ali $A$ ali $\ldots$ $A$''.

Interpretacija levega pravila vpeljave za ! je dokaj enostavna. Pove le, da se kadarkoli v procesu dokazovanja lahko odločimo formulo označiti kot nelinearno. Desno pravilo pa je malce bolj komplicirano, saj veznika ! ni tako lahko razumeti na desni strani sekventa. Rabimo, da je že celoten sekvent nelinearen, da lahko ! vpeljemo na desni. Interpretacija levega in desnega pravila za ? je simetrična.

\subsection{Notacija in druge formalnosti}

Na koncu tega poglavja je potrebnih še nekaj opomb glede zapisa veznikov, strukture sekventov ter strukture linearne logike.

\subsubsection{Poimenovanje veznikov}

Notacija, uporabljena v tem diplomskem delu, je črpana iz vira ??daj vir??, ni pa najbolj standardna. Jean-Yves Girard, ki se je prvi ukvarjal z linearno logiko je veznike in konstante označil drugače, ta notacija pa se je tudi ohranila. Razlog za spremembo notacije v mojem delu je bolj slogoven kot zgodovinski. Načini kako je notacija spremenjena ter standardna imena veznikov v angleščini so prikazana v spodnji tabeli. ??Daj also vir??
\begin{center}
\begin{tabular}{||c|c|c||}
    \hline
    Simbol veznika & Simbol v standardni notaciji & Ime \\
    \hline\hline
    $\sqcap$ & $\&$ & with \\
    \hline
    $\star$ & $\otimes$ & tensor \\
    \hline
    $\sqcup$ & $\oplus$ & plus \\
    \hline
    + & $\parr$ & par \\
    \hline
    $\top$ & $\top$ & top \\
    \hline
    $\enota$ & $\enota$ & one \\
    \hline
    $\bot$ & $\nicla$ & zero \\
    \hline
    $\nicla$ & $\bot$ & bottom \\
    \hline
    \end{tabular}
\end{center}

??tudi tukej navedi troelstra vir kjer razlozi notacijo??
Kot lahko vidimo so v standardni notaciji parni drugačni vezniki kot v naši notaciji. Razlog za to je, da veznik + distribuira čez veznik $\sqcap$, veznik $\star$ pa distribuira čez $\sqcup$. Velja torej:
\begin{align*}
    A + (B \sqcap C) &\equiv (A + B) \sqcap (A + C)\\
    A \star (B \sqcup C) &\equiv (A \star B) \sqcup (A \star C)
\end{align*}
Toda če veznik + negiramo, ne dobimo veznika $\sqcap$, ampak $\star$, če negiramo veznik $\sqcap$ pa dobimo $\sqcup$ in seveda obratno. Dualna para sta torej $(\star,+)$ ter $(\sqcap,\sqcup)$, kar je veliko bolj razvidno pri naših oznakah. Poleg tega sta si že sami pravili za veznika $\star$ ter +, iz definicij \ref{tenzor} in \ref{plus}, simetrični, kot sta si pravili za $\sqcap$ ter $\sqcup$, iz definicij \ref{in} in \ref{ali}. Zdi se mi, da sta ta dva razloga dokaj pomembna za razumevanje veznikov, zato sem se odločila za takšno notacijo, kot jo imam.

\subsubsection{Intuicionistična linearna logika}

Ker je pogosto, da članki, ki govorijo o linearni logiki, govorijo specifično o intuicionistični linearni logiki, se mi zdi pomembno predstaviti slednje tudi v mojem delu.

Intuicionistična logika je logika brez principa izključene tretje možnosti. To je aksiom, ki pravi, da za poljubno trditev $P$ velja $P\lor\neg P$. V sekventnem računu kot smo ga predstavili do sedaj, se da to pravilo izpeljati iz pravila aksioma. V linearni logiki ta princip velja le za veznik +, ne pa za veznik $\sqcup$.??citiraj??
\begin{prooftree}
	\AxiomC{}
	\pravilo{Ax}
	\UnaryInfC{$A \Rightarrow A$}
	\pravilo{R$\negacija$}
	\UnaryInfC{$\Rightarrow \negacija A,A$}
	\pravilo{R+}
	\UnaryInfC{$\Rightarrow \negacija A + A$}
\end{prooftree}
V običajnem sekventnem računu je pravilo za negacijo enako, linearnost je ne spremeni, veznika + in $\sqcap$ pa sta si ekvivalnetna in sta oba reprezentaciji veznika $\lor$. To pomeni, da izpeljava principa izključene tretje možnosti poteka enako kot zgoraj.

Če torej želimo delati z intuicionistično linearno logiko, moramo strukturo sekventnega računa nekoliko spremeniti. Izkaže se, ??citiraj?? da je dovolj, da na desni strani sekventov ne dopustimo več kot ene formule. Kot lahko hitro vidimo, zgornja izpeljava ne deluje več, saj sekvent $\Rightarrow \negacija A,A$ ne more obstajati.

Bistvo intuicionistične logike je poudarek na tem \emph{kako} dokazujemo izreke in trditve, ne le da jih dokažemo. A ker je linearna logika že zelo strukturirana, poleg tega pa bi dopuščanje le ene formule na desni strani sekventa uničilo simetrijo pravil, saj na primer ne bi smeli imeti desnega pravila za +, sem se odločila v tem delu uporabljati klasično -- torej neintuicionistično -- logiko.

\subsubsection{Naravna dedukcija}

%??lahko kle se o Gentzenu pises + glej simpsona??
Poleg sekventnega računa je v matematiki zelo pogost sistem dokazovanja \emph{naravna dedukcija}. Naravna dedukcija je načeloma zelo podobna sekventnemu računu, glavna razlika je, da imajo namesto levega in desnega pravila vpeljave vezniki \emph{pravilo vpeljave} in \emph{pravilo eliminacije}??. Tudi naravna dedukcija ima -- prirejeni -- pravili ošibitve in skrčitve, ki ju lahko odstranimo iz sistema, kar pomeni, da sekventni račun ni edini način zapisovanja linearne logike. Za tiste, ki jih zanima linearna logika v naravni dedukciji, si več lahko preberejo ??tukaj??.
%tuki ne dat not troelstre k tbh ne razlozi kaj je natural deduction, bauer!

% \subsection{Pravilo reza}
%
% Poslednje pravilo, ki ga bomo v tem poglavju vpeljali, je \emph{pravilo reza}. To pravilo obstaja tudi v običajnem sekventnem računu, a nam je relevantno v kontekstu linearne logike, zato je vpeljano v tem poglavju.
%
% \begin{definicija}
% 	\emph{Pravilo reza} je formalizacija koncepta dokazovanja z lemo. To pravilo pravi, da če znamo pod določenimi predpostavkami dokazati formulo $A$, potem pa iz te formule dokažemo nekaj drugega, lahko $A$ enostavno režemo iz procesa.
% 	\begin{prooftree}
%         \AxiomC{$\Gamma \Rightarrow A,\Delta$}
%         \AxiomC{$\Gamma',A \Rightarrow \Delta'$}
%         \pravilo{Rez}
%         \BinaryInf$\Gamma,\Gamma' \fCenter \Delta,\Delta'$
% 	\end{prooftree}
% \end{definicija}
