Ko preučujemo trditve in izreke, nas velikokrat bolj od izrekov samih zanimajo njihovi dokazi, saj ti držijo bistvo izreka. Ker bi torej želeli bolje preučiti strukturo teh dokazov, bi pomagalo, če bi lahko kako formalizirali to dokazovanje. Tu nastopijo formalni sistemi dokazovanja, ki -- kot pove že ime -- formalizirajo dokaze. Takih sistemov je mnogo, najbolj poznani izmed njih pa se delijo na tako imenovani Hilbertov stil dokazovanja in Gentzenov stil dokazovanja. Prvi sistem dokazovanja je bil predstavljen v letu 1879 s strani Gottloba Frege \cite{proof-herstory}, nemškega filozofa in matematika. Spada pod Hilbertov stil dokazovanja, saj je v drugem desetletju dvajsetega stoletja podoben sistem opisal David Hilbert \cite{proof-herstory} in o formalizaciji matematike odprl več pomembnih vprašanj. Tak sistem je karakteriziran s tem, da je vsak korak dokaza ali aksiom, ali pa je dobljen iz aksioma z enim izmed dveh pravil sklepanja Sistem tudi uporablja le dva veznika, namreč negacijo in implikacijo. V Gentzenovem stilu sta najbolj znana sekventni račun in naravna dedukcija, ki ju je prvi opisal nemški matematik Gerhard Gentzen, v članku iz leta 1934 \cite{proof-herstory} Oba sistema vključujeta več veznikov in tudi več pravil sklepanja kot sistemi v Hilbertovem stilu. Sekventni račun temelji na levih in desnih pravilih, naravna dedukcija pa ima pravila vpeljave ter pravila eliminacije. Načeloma sta sekventni račun in naravna dedukcija bolj ,,pregledna'' in se v določeni meri lepše prilagodita sklepanju, ki smo ga vajeni iz drugih vej matematike. A preglednost niti ni bila cilj Hilbertovih sistemov, saj so bile motivacije za uvedbo obeh vrst dokazovalnih sistemov pač različne. V tem delu bomo preučevali sekventni račun, ki ga bomo bolj natančno vpeljali v poglavju \ref{seq_calc}.

V članku, kjer sta bila sekventni račun in naravna dedukcija vpeljana, je poleg tega Gentzen dokazal enega izmed pomembnejših izrekov, ki se tiče sistemov dokazovanja, namreč izrek o eliminaciji rezov. Ta sistemu dokazovanja na formalen način zagotavlja konsistentnost. V poglavju \ref{ier} bomo tudi mi ta izrek formulirali ter dokazali, le da bomo to naredili za različico sekventnega računa, imenovano linearna logika. Slednjo je prvič v članku iz leta 1987 opisal Jean-Yves Girard \cite{ll-herstory}, francoski matematik, ki je ugotovil, da z omejitvijo določenih strukturnih pravil v formalnem sistemu dokazovanja lahko bolj natančno preučujemo, \emph{koliko} predpostavk smo porabili v dokazu. Linearno logiko je možno obravnavati tako v naravni dedukciji kot sekventnem računu, a se bomo v tem delu omejili na sekventni račun.

Glavna motivacija za linearno logiko je zavedanje, koliko ,,surovin'' smo porabili in pridelali, torej kolikokrat smo tekom dokaza predpostavko uporabili, katere predpostavke smo zavrgli ter katere sklepe smo dokazali večkrat. Omejitev podvajanja in odvečnih predpostavk pomembno vpliva tudi na veznike, ki jih uporabljamo. Več o tem bomo povedali v poglavju \ref{ll}.
