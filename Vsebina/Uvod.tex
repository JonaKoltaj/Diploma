Ko preučujemo trditve in izreke, nas velikokrat bolj od izrekov samih zanimajo njihovi dokazi, saj ti držijo bistvo izreka samega. Ker bi torej želeli bolje preučiti strukturo teh dokazov, bi pomagalo, če bi lahko kako formalizirali to dokazovanje. Tu nastopijo formalni sistemi dokazovanja, ki -- kot pove že ime -- formalizirajo dokaze. Takih sistemov je več, najbolj poznani izmed njih pa so Hilbertov sistem, naravna dedukcija ter sekventni račun. Prvi Hilbertov sistem dokazovanja je bil predstavljen v letu 1879 s strani Gottloba Frege, nemškega filozofa in matematika. V tem sistemu je vsak korak dokaza ali aksiom, ali pa je dobljen iz aksioma z enim izmed dveh pravil sklepanja. Karakteriziran je tudi s tem, da uporablja le dva veznika, namreč implikacijo in negacijo, kar ga sicer naredi minimalističnega in neodvečnega, a je velikokrat bistvo dokaza težko izluščiti. Sekventni račun in naravna dedukcija pa sta bila vpeljana v istem članku, leta 1934, ki ga je objavil nemški matematik Gerhard Gentzen. Te dva sistema vključujeta več veznikov in tudi več pravil sklepanja. Sekventni račun temelji na levih in desnih pravilih vpeljave veznikov, naravna dedukcija pa ima pravila vpeljave ter pravila eliminacije. Oba sistema sta torej veliko manj minimalistična od Hilbertovega sistema, a sta v nekaterih pogledih bolj berljiva. V tem delu bomo preučevali sekventni račun, ki ga bomo bolj natančno vpeljali v poglavju \ref{seq_calc}.

V članku, kjer sta bila sekventni račun in naravna dedukcija vpeljana, je poleg tega Gentzen dokazal enega izmed pomembnejših izrekov, kar se tiče sistemov dokazovanja, namreč izrek o eliminaciji rezov. Ta sistemu dokazovanja na formalen način zagotavlja konsistentnost. V poglavju \ref{ier} bomo tudi mi ta izrek formulirali ter dokazali, le da bomo to naredili za podzvrst sekventnega računa, imenovano linearna logika. Slednjo je prvič v članku iz leta 1987 opisal Jean-Yves Girard, francoski matematik, ki je ugotovil, da z omejitvijo določenih strukturnih pravil v formalnem sistemu dokazovanja lahko bolj natančno preučujemo \emph{koliko} predpostavk smo porabili v dokazu. Linearno logiko je možno obravnavati tako v naravni dedukciji kot sekventnem računu, a se bomo v tem delu omejili na sekventni račun.

Glavna motivacija za linearno logiko je zavedanje, koliko ,,surovin'' smo porabili in pridelali, torej kolikokrat smo tekom dokaza predpostavko uporabili, katere predpostavke smo zavrgli ter katere sklepe smo dokazali večkrat. Omejitev podvajanja in odvečnih predpostavk pomembno vpliva tudi na veznike, ki jih uporabljamo. Več o tem bomo povedali v poglavju \ref{ll}. Želimo pa tudi vedeti, kako linearna logika modelira nelinarno logiko in kako se ti dve primerjata med seboj, kar pa bomo nazadnje obravnavali v poglavju \ref{cl v cll}.
