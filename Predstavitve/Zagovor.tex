\documentclass[xcolor=dvipsnames]{beamer}
\usepackage{bussproofs}
\usepackage{changepage}
\usepackage{amsfonts}
\usepackage{amsmath}
\usepackage{amssymb}

\usetheme{Szeged}
\useinnertheme{circles}

\beamertemplatenavigationsymbolsempty
\setbeamertemplate{footline}{}

\addtobeamertemplate{block begin}{}{\begin{adjustwidth}{10pt}{10pt} \begin{flushleft}}
\addtobeamertemplate{block end}{\end{flushleft} \end{adjustwidth}}{}

\definecolor{svetlozelena}{RGB}{204,255,153}
\definecolor{temnozelena}{RGB}{0,102,0}

\usecolortheme[named=temnozelena]{structure}
\setbeamercolor{block title}{bg=svetlozelena!60}
\setbeamercolor{block body}{bg=svetlozelena!40}
\setbeamercolor{block title alerted}{bg=svetlozelena!50, fg=temnozelena}
\setbeamercolor{block body alerted}{bg=svetlozelena!20}

\newenvironment{bprooftree}{\leavevmode\hbox\bgroup}{\DisplayProof\egroup}

\newcommand{\oranzna}[1]{\textcolor{RedOrange}{\textit{#1}}}
\newcommand{\pravilo}[1]{\RightLabel{\footnotesize \textcolor{temnozelena}{#1}}}
\newcommand{\levopravilo}[1]{\LeftLabel{\footnotesize \textcolor{temnozelena}{#1}}}
\newcommand{\dol}{\begin{center}$\downarrow$\end{center}}
\newcommand{\R}{\mathfrak{R}}
\newcommand{\h}{\mathfrak{h}}
\newcommand{\D}{\mathcal{D}}
\newcommand{\derivation}[2]{\AxiomC{$\D_{#1}$}\noLine\UnaryInfC{{#2}}}


\title{Eliminacija rezov v linearni logiki}
\author{Jona Koltaj\\{\small Mentor: prof. dr. Andrej Bauer}}
\date{26.9.2025}

\begin{document}

\begin{frame}
  \titlepage
\end{frame}

\begin{frame}{Sekvent}
    \begin{block}{Definicija}
        Naj bodo $A_0,\dots,A_n,B_0,\dots,B_m$ logične formule. \oranzna{Sekvent} je izraz oblike:
        $$
        A_0,\dots,A_n \Rightarrow B_0,\dots,B_m
        $$
        Formulam na levi pravimo \oranzna{predpostavke}, označimo jih z $\Gamma$, formulam na desni pa \oranzna{sklepi}, označimo jih z $\Delta$.
    \end{block}
    \pause
    \begin{block}{Opomba}
        Če $\Gamma$ in $\Delta$ definiramo kot multimnožici nam vrstni red predpostavk in sklepov ni pomemben.
    \end{block}
\end{frame}


\begin{frame}{Logična pravila}
    \begin{alertblock}{Logični pravili za $\land$}
        \begin{center}
            \begin{bprooftree}
                \AxiomC{$\Gamma, A_0 \Rightarrow \Delta$}
                \levopravilo{L$\land$}
                \UnaryInfC{$\Gamma,A_0 \land A_1 \Rightarrow \Delta$}
            \end{bprooftree}
            ali
            \begin{bprooftree}
                \AxiomC{$\Gamma, A_1 \Rightarrow \Delta$}
                \pravilo{L$\land$}
                \UnaryInfC{$\Gamma,A_0 \land A_1 \Rightarrow \Delta$}
            \end{bprooftree}
        \end{center}
        \begin{prooftree}
            \AxiomC{$\Gamma \Rightarrow A_0,\Delta$}
            \AxiomC{$\Gamma \Rightarrow A_1,\Delta$}
            \pravilo{R$\land$}
            \BinaryInfC{$\Gamma \Rightarrow A_0 \land A_1,\Delta$}
        \end{prooftree}
    \end{alertblock}
\end{frame}

\begin{frame}{Strukturna pravila}
     \begin{block}{Pravilo aksioma}
        \oranzna{Aksiom} je sekvent oblike $A \Rightarrow A$. Pravilo aksioma nam pove:
        \vspace{10pt}
        \begin{prooftree}
            \AxiomC{}
            \pravilo{Ax}
            \UnaryInfC{$A \Rightarrow A$}
        \end{prooftree}
    \end{block}
    \pause
    \begin{alertblock}{Pravili \oranzna{ošibitve} in \oranzna{skrčitve}}
        \begin{center}
            \begin{bprooftree}
                \AxiomC{$\Gamma \Rightarrow \Delta$}
                \levopravilo{W}
                \UnaryInfC{$\Gamma,A \Rightarrow \Delta$}
            \end{bprooftree}
            ali
             \begin{bprooftree}
                \AxiomC{$\Gamma \Rightarrow \Delta$}
                \pravilo{W}
                \UnaryInfC{$\Gamma \Rightarrow A,\Delta$}
            \end{bprooftree}\\
            \medskip
            \begin{bprooftree}
                \AxiomC{$\Gamma,A,A \Rightarrow \Delta$}
                \levopravilo{C}
                \UnaryInfC{$\Gamma,A \Rightarrow \Delta$}
            \end{bprooftree}
            ali
            \begin{bprooftree}
                \AxiomC{$\Gamma \Rightarrow A,A,\Delta$}
                \pravilo{C}
                \UnaryInfC{$\Gamma \Rightarrow A,\Delta$}
            \end{bprooftree}
        \end{center}
    \end{alertblock}
\end{frame}

\begin{frame}{Linearna Logika}
    \begin{block}{Definicija}
        V \oranzna{linearni logiki} zavržemo pravili skrčitve in ošibitve.
    \end{block}
    \pause
    \begin{alertblock}{Veznika $\star$ (\oranzna{tenzor}) ter $\sqcap$ (\oranzna{in})}
        \begin{prooftree} \hskip -130pt
            \AxiomC{$\Gamma,A_0,A_1 \Rightarrow \Delta$}
            \levopravilo{L$\star$}
            \UnaryInfC{$\Gamma,A_0 \star A_1 \Rightarrow \Delta$}
        \end{prooftree}
        \begin{prooftree} \vskip -41pt \hskip 130pt
            \AxiomC{$\Gamma, A_i \Rightarrow \Delta$}
            \pravilo{L$\sqcap$}
            \UnaryInfC{$\Gamma,A_0 \sqcap A_1 \Rightarrow \Delta$}
        \end{prooftree}
        \begin{prooftree} \hskip -160pt
            \AxiomC{$\Gamma \Rightarrow A_0,\Delta$}
            \AxiomC{$\Gamma' \Rightarrow A_1,\Delta'$}
            \levopravilo{R$\star$}
            \BinaryInfC{$\Gamma,\Gamma' \Rightarrow A_0 \star A_1,\Delta,\Delta'$}
        \end{prooftree}
        \begin{prooftree} \vskip -41pt \hskip 160pt
            \AxiomC{$\Gamma \Rightarrow A_0,\Delta$}
            \AxiomC{$\Gamma \Rightarrow A_1,\Delta$}
            \pravilo{R$\sqcap$}
            \BinaryInfC{$\Gamma \Rightarrow A_0 \sqcap A_1,\Delta$}
        \end{prooftree}
    \end{alertblock}
\end{frame}

\begin{frame}{Eksponenti}
    \begin{alertblock}{Logična pravila za ! (\oranzna{seveda}) in ? (\oranzna{zakaj ne})}
        \begin{prooftree} \hskip -130pt
            \AxiomC{$\Gamma,A \Rightarrow \Delta$}
            \pravilo{L!}
            \UnaryInfC{$\Gamma,!A \Rightarrow \Delta$}
        \end{prooftree}
        \begin{prooftree} \vskip -40pt \hskip 130pt
            \AxiomC{$\Gamma \Rightarrow A,\Delta$}
            \pravilo{R?}
            \UnaryInfC{$\Gamma \Rightarrow \ ?A,\Delta$}
        \end{prooftree}
        \begin{prooftree} \hskip -130pt
            \AxiomC{$\Gamma \Rightarrow \Delta$}
            \pravilo{W!}
            \UnaryInfC{$\Gamma,!A \Rightarrow \Delta$}
        \end{prooftree}
        \begin{prooftree} \vskip -40pt \hskip 130pt
            \AxiomC{$\Gamma \Rightarrow \Delta$}
            \pravilo{W?}
            \UnaryInfC{$\Gamma \Rightarrow \ ?A,\Delta$}
        \end{prooftree}
        \begin{prooftree} \hskip -130pt
            \AxiomC{$\Gamma,!A,!A \Rightarrow \Delta$}
            \pravilo{C!}
            \UnaryInfC{$\Gamma,!A \Rightarrow \Delta$}
        \end{prooftree}
        \begin{prooftree} \vskip -40pt \hskip 130pt
            \AxiomC{$\Gamma \Rightarrow ?A,?A,\Delta$}
            \pravilo{C?}
            \UnaryInfC{$\Gamma \Rightarrow \ ?A,\Delta$}
        \end{prooftree}
        \begin{prooftree} \hskip -130pt
            \AxiomC{$!\Gamma \Rightarrow A,?\Delta$}
            \pravilo{R!}
            \UnaryInfC{$!\Gamma \Rightarrow \ !A,?\Delta$}
        \end{prooftree}
        \begin{prooftree} \vskip -40pt\hskip 130pt
            \AxiomC{$!\Gamma, A \Rightarrow ?\Delta$}
            \pravilo{R?}
            \UnaryInfC{$!\Gamma,?A \Rightarrow \ ?\Delta$}
        \end{prooftree}
    \end{alertblock}
\end{frame}

\begin{frame}{Rez}
    \begin{alertblock}{Logično pravilo \oranzna{reza}}
        \begin{prooftree}
            \AxiomC{$\Gamma,A \Rightarrow \Delta$}
            \AxiomC{$\Gamma' \Rightarrow A, \Delta'$}
            \pravilo{Rez}
            \BinaryInfC{$\Gamma,\Gamma' \Rightarrow \Delta,\Delta'$}
        \end{prooftree}
    \end{alertblock}
    \pause
    \begin{block}{Izrek}
        Izrek o \oranzna{eliminaciji reza} nam pove, da lahko vsak sekvent, izpeljan z uporabo reza, izpeljemo tudi brez uporabe reza.
    \end{block}
\end{frame}

\begin{frame}{Indukcija}
    \begin{block}{Zunanja indukcija}
        Začnemo z indukcijo na številu rezov v drevesu izpeljave. Če rezov ni, smo opravili, drugače si izberemo vrhnjega.
    \end{block}
    \pause
    \begin{block}{Notranja indukcija}
        Znotraj zgornjega koraka indukcije delamo indukcijo na stopnji reza. (Pod)drevo izpeljave preobrazimo tako da vsebuje rez nižje stopnje.
    \end{block}
    \pause
    \begin{block}{Definicija}
        \oranzna{Stopnja reza} je par $(\R,\h)$, kjer je $\R$ rang rezane formule, $\h$ pa višina drevesa, ki se konča z danim rezom. Stopnje so urejene leksikografsko.
    \end{block}
\end{frame}

\begin{frame}{Eliminacija glavnega reza}
    \begin{alertblock}{Glavni rez veznika $\sqcap$}
        \begin{prooftree}
            \derivation{0}{$\Gamma,A_0 \Rightarrow \Delta$}
            \levopravilo{L$\sqcap$}
            \UnaryInfC{$\Gamma,A_0 \sqcap A_1 \Rightarrow \Delta$}

            \derivation{1}{$\Gamma' \Rightarrow A_0,\Delta'$}
            \derivation{2}{$\Gamma' \Rightarrow A_1,\Delta'$}
            \pravilo{R$\sqcap$}
            \BinaryInfC{$\Gamma' \Rightarrow A_0 \sqcap A_1,\Delta'$}

            \pravilo{Rez}
            \BinaryInfC{$\Gamma,\Gamma' \Rightarrow \Delta,\Delta'$}
        \end{prooftree}
        \dol
        \begin{prooftree}
            \derivation{0}{$\Gamma,A_0 \Rightarrow \Delta$}
            \derivation{1}{$\Gamma' \Rightarrow A_0,\Delta'$}
            \pravilo{Rez}
            \BinaryInfC{$\Gamma,\Gamma' \Rightarrow \Delta,\Delta'$}
        \end{prooftree}
    \end{alertblock}
\end{frame}

\begin{frame}{Eliminacija reza, ki ni glaven}
    \begin{alertblock}{Rez formule $C$}
        \begin{prooftree} \vskip -10pt
            \derivation{0}{$\Gamma, A_0 \Rightarrow C,\Delta$}
            \levopravilo{L$\sqcap$}
            \UnaryInfC{$\Gamma, A_0 \sqcap A_1 \Rightarrow C,\Delta$}

            \derivation{1}{$\Gamma',C \Rightarrow \Delta'$}
            \pravilo{Rez}
            \BinaryInfC{$\Gamma,\Gamma', A_0 \sqcap A_1 \Rightarrow \Delta,\Delta'$}
        \end{prooftree}
        \dol
        \begin{prooftree} \vskip -20pt
            \derivation{0}{$\Gamma,A_0 \Rightarrow C,\Delta$}
            \derivation{1}{$\Gamma',C \Rightarrow \Delta'$}
            \pravilo{Rez}
            \BinaryInfC{$\Gamma,\Gamma',A_0 \Rightarrow \Delta,\Delta'$}
            \pravilo{L$\sqcap$}
            \UnaryInfC{$\Gamma,\Gamma', A_0 \sqcap A_1 \Rightarrow \Delta,\Delta'$}
        \end{prooftree}
    \end{alertblock}
\end{frame}

\begin{frame}{Eliminacija glavnega reza eksponenta}
    \begin{block}{Opomba}
        Pri eliminaciji glavnega reza z eksponenti (specifično !) imamo tri možnosti; pare (C!,R!), (L!,R!) in (W!,R!).
    \end{block}
    \pause
    \begin{alertblock}{Posplošeno pravilo reza}
        \begin{prooftree}
            \AxiomC{$\Gamma, (!A)^n \Rightarrow \Delta$}
            \AxiomC{$\Gamma' \Rightarrow !A,\Delta'$}
            \pravilo{Rez!$_n$}
            \BinaryInfC{$\Gamma,\Gamma' \Rightarrow \Delta,\Delta'$}
        \end{prooftree}
    \end{alertblock}
\end{frame}

\begin{frame}{Eliminacija glavnega reza eksponentov}
    \begin{alertblock}{Primer (C!,R!)}
        \begin{prooftree} \vskip -10pt
            \derivation{0}{$\Gamma,(!A)^{n+1} \Rightarrow \Delta$}
            \levopravilo{C!}
            \UnaryInfC{$\Gamma,(!A)^n \Rightarrow \Delta$}

            \derivation{1}{$!\Gamma' \Rightarrow A,?\Delta'$}
            \pravilo{R!}
            \UnaryInfC{$!\Gamma' \Rightarrow !A,?\Delta'$}

            \pravilo{Rez!$_n$}
            \BinaryInfC{$\Gamma,!\Gamma' \Rightarrow \Delta,?\Delta'$}
        \end{prooftree}
        \dol
        \begin{prooftree} \vskip - 20pt
            \derivation{0}{$\Gamma,(!A)^{n+1} \Rightarrow \Delta$}

            \derivation{1}{$!\Gamma' \Rightarrow A,?\Delta'$}
            \pravilo{R!}
            \UnaryInfC{$!\Gamma' \Rightarrow !A,?\Delta'$}

            \pravilo{Rez!$_{n+1}$}
            \BinaryInfC{$\Gamma,!\Gamma' \Rightarrow \Delta,?\Delta'$}
        \end{prooftree}
    \end{alertblock}
\end{frame}

\begin{frame}{Konec}
    \centering \oranzna{HVALA ZA POZORNOST:)}
\end{frame}


\end{document}
